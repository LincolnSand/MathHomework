\documentclass{article}

\usepackage{amsfonts}
\usepackage{graphicx}
\usepackage{amssymb}
\usepackage{amsmath}


\setlength\parindent{18pt}

\begin{document}

Homework Excercises:

1) Beck Excercise 2.2.
Let $a \in \mathbb{R}^n$ be a non-zero vector. Show that the maximum of $a^{T}x$
over $B[0, 1] = \{x \in \mathbb{R}^n : \lVert x \rVert \leq 1\}$ is attained at
$x = \frac{1}{\lVert a \rVert}a$ and that the maximal value is $\lVert a \rVert$.

Let us first consider the case where $x = \frac{1}{\lVert a \rVert}a$,
if the norm here is taken to be the euclidean norm (p = 2 norm), then
$a^{T} x = a^{T} \frac{1}{\lVert a \rVert}a = \frac{1}{\lVert a \rVert} a^{T}a$

$a^{T} a = a \cdot a = {\lVert a \rVert}^2$

$\implies \frac{1}{\lVert a \rVert} a^{T}a =
\frac{1}{\lVert a \rVert} {\lVert a \rVert}^2$
$ = \lVert a \rVert$.

So, we've shown that if $x = \frac{1}{\lVert a \rVert} a$, then the maximum value
is indeed $\lVert a \rVert$. Now we need to show that the maximum truly is obtained
at $x = \frac{1}{\lVert a \rVert} a$.


First of all, $|x|$ must be a vector with a norm of exactly 1.
This is the case because if it had a norm below 1, it would be an internal point.
And if we take a vector of norm < 1 and take its dot product with a non-zero real vector,
then its value will necessarily be less than the same vector if its norm was 1.
For instance, let's consider a vector of norm $b$ where $b < 1$. Let $a = \frac{1}{b}$
so that $a \cdot b = 1$. Since $b$ is strictly less than 1,
$a$ will always be strictly greater than 1. And when we do a dot product,
we would end up with:
$v_1 \cdot v_2$
vs $a \cdot v_1 \cdot v_2$
where $v_1$ is the vector with norm of $b$ and $v_2$ is any non-zero real vector.
Since we showed that $a$ is greater than 1 and $v_1 \cdot v_2$ is a real number,
then $a \cdot v_1 \cdot v_2$ will always be a larger number if $v_1 \cdot v_2$
is positive.

And for obvious reasons we can discard all of the points that result in a negative value
when the dot product is taken with $a$ (since $\exists$ a
point in the unit ball where it will be positive).

Since the norm of the boundary points of the unit ball is exactly 1,
the best we can hope for is a maximum value of ${\lVert a \rVert}$
since if all the values are added together to make a positive number,
it won't affect the norm of $a$, only the sign.

And since $a \cdot a$ will always result in a positive value,
then we can conclude that $a = \frac{1}{\lVert a \rVert}a$
is in fact the value that yields the maximum value in the closed unit ball
when taking the dot product with some non-zero real vector $a$.


2) Beck Excercise 2.6. Let $B \in \mathbb{R}^{n \times k}$
and let $A = BB^{T}$.

i) Prove that $A$ is positive semidefinite.

$BB^{T}$ will necessarily be positive semidefinite because of the fact
that the entries of the resulting matrix is made up of the squares of the
input matrices. This is the case since matrix multiplication is
made by multiplying the entries of the columns in the rhs matrix by the
entries of the rows of the lhs matrix and a transpose of a matrix
turns all the columns into rows (and vice versa). And since a
square of a real number is always postitive,
then all of the entries of the resulting matrix are positive.
This is similar to how $x^2 \geq 0$.


ii) Prove that $A$ is positive defininite if and only if
B has a full row rank.

Since we are required to have all positive eigenvalues,
the only way we wouldn't in this case is if we have eigenvalues of $0$.
This is only possible if the matrix is not linearly independent.

$\therefore$ $A$ is positive defininite if and only if
B has a full row rank.


3) Beck Excercise 2.10. Let $A^{\alpha}$ be the $n \times n$
matrix ($n > 1$) defined by

\[A_{ij}^{\alpha} = \begin{cases}
    \alpha & i = j \\
    1 & i \neq j
\end{cases}.\]

Show that $A^{\alpha}$ is positive semidefinite if
and only if $\alpha \geq 1$.

This forms a matrix that has entries of $1$ everywhere except for the main diagonal,
which has entries of $\alpha$. If the main diagonal entries are $1$ (equal to all
other entries in the matrix), we will get eigenvalues of $0$. But if we have entries
greater than $1$, they will be positive, and if they are less than $1$, then they must
be negative.

And since positive semidefinite requires us to have all non-negative eigenvalues,
this can only occur if the main diagonal is greater than or equal to all the
other entries of the matrix. Thus, the main diagonal entries must
be greater than or equal to $1$. Which means that $\alpha \geq 1$.


4) Beck Excercise 2.14 (Schur complement lemma). Let

\[D = \begin{pmatrix}
    A & b \\
    b^{T} & c
\end{pmatrix},\]

where $A \in \mathbb{R}^{n \times n}, b \in \mathbb{R}^{n}, c \in \mathbb{R}.$
Suppose that $A \succ 0$. Prove that $D \succeq 0$ if and only if
$c - b^{T}A^{-1}b \geq 0$.

$A \succ 0 \implies b^{T}A^{-1}b > 0$

$c - b^{T}A^{-1}b \geq 0 \implies c \geq b^{T}A^{-1}b$

$\implies c > 0$

That means the main diagonal of $D$ must be positive or
part of an already positive definite matrix's main diagonal.
This means that $D$ must be positive semidefinite since the newly
introduced eigenvalue not present in $A$ has to be greater than or equal to $0$.


5) Beck Excercise 2.17 (iv and v). For each of the following functions,
find all the stationary points and classify them according to whether
they are saddle points, strict/non-strict local/global minimum/maximum points:

iv) $f(x_1, x_2) = x_1^4 + 2x_1^2 x_2 + x_2^2 - 4x_1^2 - 8x_1 - 8x_2$.

$\nabla f(x_1, x_2) = (4x_1^3 + 4x_1 x_2 + 8x_1 - 8, 2x_1^2 + 2x_2^2 - 8)$

$ = (4(x_1^3 + x_1 x_2) + 8(x_1 - 1), 2(x_1^2 + x_2^2) - 8)$


$4(x_1^3 + x_1 x_2) + 8(x_1 - 1) = 0
\implies (x_1^3 + x_1 x_2) + 2(x_1 - 1) = 0
\implies x_1^3 + x_1 x_2 + 2x_1 = 1
\implies x_1(x_1^2 + x_2 + 2) = 1
\implies x_1(\frac{4}{x_2^2} + x_2 + 2) = 1$

$2(x_1^2 + x_2^2) - 8 = 0
\implies 2(x_1^2 + x_2^2) = 8
\implies x_1^2 + x_2^2 = 4
\implies x_1^2 = \frac{4}{x_2^2}$

Don't know what to do next for $iv)$.



v) $f(x_1, x_2) = (x_1 - 2x_2)^4 + 64x_1 x_2$.

$\nabla f(x_1, x_2) = (4(x_1 - 2x_2)^3 + 64x_2, -8(x_1 - 2x_2)^3 + 64x_1)$

Let $u = (x_1 - 2x_2)^3$.

$(4(x_1 - 2x_2)^3 + 64x_2, -8(x_1 - 2x_2)^3 + 64x_1)
= (4u + 64x_2, -8u + 64x_1)$

$4u + 64x_2 = 0
\implies 4u = -64x_2
\implies -8u = 128x_2$

$-8u + 64x_1 = 0
\implies 8u = 64x_1
\implies 4u = 32x_1$

$\implies (4u + 64x_2, -8u + 64x_1) = (32x_1 + 64x_2, 128x_2 + 64x_1)$

$32x_1 + 64x_2 = 0
\implies x_1 + 2x_2 = 0$

$128x_2 + 64x_1 = 0
\implies 2x_2 + x_1 = 0$


If $x_1 = -2x_2$, then it is a stationary point.


\end{document}
