\documentclass{article}

\usepackage{amsfonts}
\usepackage{graphicx}
\usepackage{amssymb}
\usepackage{amsmath}
\usepackage{listings}


\DeclareMathOperator{\sech}{sech}
\newcommand{\NN}{\mathbb{N}}
\newcommand{\RR}{\mathbb{R}}
\newcommand{\QQ}{\mathbb{Q}}
\newcommand{\ZZ}{\mathbb{Z}}
\newcommand{\dV}{\;\mathrm{d}V}
\newcommand{\dA}{\;\mathrm{d}A}
\newcommand{\dx}{\;\mathrm{d}x}
\newcommand{\dy}{\;\mathrm{d}y}
\newcommand{\dz}{\;\mathrm{d}z}
\newcommand{\cA}{\mathcal{A}}
\newcommand{\Bb}{\mathcal{B}}
\newcommand{\Ww}{\mathcal{W}}
\newcommand{\Dd}{\mathcal{D}}
\newcommand{\Ss}{\mathcal{S}}
\newcommand{\Ee}{\mathcal{E}}
\DeclareMathOperator{\im}{im}


\setlength\parindent{18pt}

\begin{document}

1. Beck Exercise 10.6. Consider the maximization problem
\[\max x_1^2 + 2 x_1 x_2 + 2 x_2^2 - 3 x_1 + x_2\]
s.t. $x_1 + x_2 = 1$,
$x_1, x_2 \geq 0$.

i) Is the problem convex?

A twice-differentiable function is convex if and only if its Hessian is positive semidefinite.

The hessian is
\[H = \begin{bmatrix}
    2 & 2 \\
    2 & 4
\end{bmatrix}\]

which is positive definite. Therefore, it is convex.

ii) Find all the KKT points of the problem.

The only KKT point is:
\[\begin{bmatrix}
    x_1 \\
    x_2 \\
    \lambda
\end{bmatrix} = \begin{bmatrix}
    3 \\
    -2 \\
    -1
\end{bmatrix}\]

iii) Find the optimal solution of the problem.

Since we only have one KKT point (3, -2), it is the optimal/minimum solution.
It has a value of -6 at that point.


2. Beck Exercise 10.8. Consider the problem
\[\min x_1^2 + 2 x_2^2 + x_1\]
s.t. $x_1 + x_2 \leq a$,
where $a \in \RR$ is a parameter.

i) Prove that for any $a \in \RR$, the problem has a unique optimal solution (without actually solving it).

We need to check the convexity of both the objective function and the constraint set.

First, for the objective function:
$f(x_1, x_2) = x_1^2 + 2x_2^2 + x_1$ is a quadratic function where the quadratic terms are positive.
This implies f is strictly convex for both $x_1$ and $x_2$.

Second, for the constraint:
It's linear, so it's convex.

Thus, the optimization problem is convex.

ii) Solve the problem (the solution will be in terms of the parameter a).

If we solve this problem using unconstrained minimzation,
we get an unconstrained minimizer of:
\[\begin{bmatrix}
    x_1 \\
    x_2
\end{bmatrix} = \begin{bmatrix}
    -\frac{1}{2} \\
    0
\end{bmatrix}\]

Now we need to check that it is feasible.
The above point is feasible for $a \geq -\frac{1}{2}$.

Now, for the boundary solutions.

If we substitute $x_2 = a - x_1$ into the objective function, then we get:
\[f(x_1, a-x_1) = x_1^2 + 2 (a-x_1)^2 + x_1\]

Then minimize this function in terms of $x_1$ and parameter a to find the optimal $x_1$ (and $x_2$).

iii) Let f(a) be the optimal value of the problem with parameter a. Write an explicit expression for f and prove that it is a convex function.



3. Beck Exercise 11.2. Consider the optimization problem
\[\min \{a^T x : x^T Q x + 2 b^T x + c \leq 0\},\]
where $Q \in \RR^{n \times n}$ is positive definite, $a (\neq 0)$, $b \in \RR^n$, and $c \in \RR$.
i) For which values of Q, b, c is the constraint set non-empty?

Since Q is positive definite, $x^T Q x$ is always non-negative. The constraint
set is non-empty if the quadratic form $x^T Q x + 2b^T x + c$ can have non-positive values.
This is generally possible unless the quadratic form is strictly positive $\forall$ x,
which happens if the quadratic function has a positive minimum. The minimum
value occurs at $x = -Q^{-1} b$. Substituting this back into the quadratic form
gives the conditions for non-emptiness.

ii) For which values of Q, b, c are the KKT conditions necessary?

The KKT conditions are necessary for optimality in problems where
the objective function and the constraints are continuously differentiable,
and the constraint qualification holds. In this case, since Q is positive definite,
the quadratic form in the constraint is continuously differentiable. The KKT conditions
are necessary when the Slater's condition is satisifed, which requires the existence
of a point where the inequality constraint is strictly satisifed (i.e. $x^T Q x + 2b^T x + c < 0$).

iii) For which values of Q, b, c are the KKT conditions sufficient?

The KKT conditions are sufficient if the problem is convex. A minimzation
problem is convex if the objective function is convex and the inequality constraints
are concave. The objective function $a^T x$ is linear (which means it's convex),
and the constraint is concave if $-x^T Q x - 2b^T x - c$ is concave.
Since Q is positive definite, the constraint is concave.
Thus, the KKT conditions are sufficient for any positive definite Q, and any b, c.

iv) Under the conditions of part ii), find the optimal solution of (P) using the KKT conditions.



4. Beck Exercise 11.4. Consider the optimization problem
\[\min x_1^2 - x_2^2 - x_3^2\]
s.t. $x_1^4 + x_2^4 + x_3^4 \leq 1$.

i) Is the problem convex?

The objective function is not convex because its Hessian is not positive definite
($x_2^2$ and $x_3^2$ have negative coefficients).

Therefore the problem is not convex.

ii) Find all the KKT points of the problem.

The real valued KKT points are:
\[(x_1, x_2, x_3, \lambda) = (-1, 0, 0, -\frac{1}{2})\]
\[(x_1, x_2, x_3, \lambda) = (0, -1, 0, \frac{1}{2})\]
\[(x_1, x_2, x_3, \lambda) = (0, 0, -1, \frac{1}{2})\]
\[(x_1, x_2, x_3, \lambda) = (0, 0, 1, \frac{1}{2})\]
\[(x_1, x_2, x_3, \lambda) = (0, 1, 0, \frac{1}{2})\]
\[(x_1, x_2, x_3, \lambda) = (1, 0, 0, -\frac{1}{2})\]

And some other points involving non-integer powers of 2 and 3.

iii) Find the optimal solution of the problem.

The KKT points that attain the minimum value are:
\[(x_1, x_2, x_3, \lambda) = (0, -2^{\frac{3}{4}}/2, -2^{\frac{3}{4}}/2, \sqrt{2}/2)\]
\[(x_1, x_2, x_3, \lambda) = (0, -2^{\frac{3}{4}}/2, 2^{\frac{3}{4}}/2, \sqrt{2}/2)\]
\[(x_1, x_2, x_3, \lambda) = (0, 2^{\frac{3}{4}}/2, -2^{\frac{3}{4}}/2, \sqrt{2}/2)\]
\[(x_1, x_2, x_3, \lambda) = (0, 2^{\frac{3}{4}}/2, 2^{\frac{3}{4}}/2, \sqrt{2}/2)\]

The value at these points is $-\sqrt{2}$.

\end{document}
