\documentclass{article}

\usepackage{amsfonts}
\usepackage{graphicx}
\usepackage{amssymb}
\usepackage{amsmath}
\usepackage{listings}


\DeclareMathOperator{\sech}{sech}
\newcommand{\NN}{\mathbb{N}}
\newcommand{\RR}{\mathbb{R}}
\newcommand{\QQ}{\mathbb{Q}}
\newcommand{\ZZ}{\mathbb{Z}}
\newcommand{\dV}{\;\mathrm{d}V}
\newcommand{\dA}{\;\mathrm{d}A}
\newcommand{\dx}{\;\mathrm{d}x}
\newcommand{\dy}{\;\mathrm{d}y}
\newcommand{\dz}{\;\mathrm{d}z}
\newcommand{\cA}{\mathcal{A}}
\newcommand{\Bb}{\mathcal{B}}
\newcommand{\Ww}{\mathcal{W}}
\newcommand{\Dd}{\mathcal{D}}
\newcommand{\Ss}{\mathcal{S}}
\newcommand{\Ee}{\mathcal{E}}
\DeclareMathOperator{\im}{im}


\setlength\parindent{18pt}

\begin{document}

Problem 1. (3210 Review Problem) Make an educated guess what is the limit of the sequence
$a_n = \frac{5n^2}{n^3 + 5}$. Then use the definition of the limit to prove
that your guess is correct.

My guess is that the limit of the sequence is 0. This is because the denominator
has a higher power than the numerator.

To prove this, we have to show that for $\epsilon > 0$, $\exists N$ such that
$\forall n \geq N$, $|a_n - 0| < \epsilon$.

First note that $\frac{5n^2}{n^3 + 5} < \frac{5n^2}{n^3} = \frac{5}{n}$.

$\frac{5}{n} < \epsilon$, so we choose $N > \frac{5}{\epsilon}$. And since
$\frac{5n^2}{n^3 + 5} < \frac{5}{n}$,
this means that the sequence converges to 0.


Problem 2. (3210 Review Problem) Define the sequence $\{a_n\}$ inductively by
\[a_n = 1; a_{n+1} = \frac{n^2 + n + 1}{n^2 + n + 2} a_n\]
for $n \geq 1$. Prove that the sequence converges. State all theorems, if any,
which you used in the proof.

We need to show that the sequence is both bounded and monotonically
increasing. Then we will know it converges by the monotonic convergence theorem.

First, let's show that it is bounded. Note that $\frac{n^2 + n + 1}{n^2 + n + 2} < 1$.
$\forall n \geq 1$. That means that $a_{n+1}$ is multiplying $a_n$ by a fraction that is
less than 1. This implies that the sequence is decreasing. And since all terms
are strictly positive, this means that the sequence is bounded below by 0.

For monotonicity, we just take what we showed before: that the sequence is decreasing.
This means each $a_{n+1}$ is smaller than the preceding $a_n$ but still positive.
So the sequence is monotonically decreasing.

Applying the monotonic convergence theorem tells that the sequence converges
since it is monotonically decreasing and bounded below by 0.


Problem 3. For the vectors $x = (1, 0, 2)$ and $y = (-1, 3, 1)$ find

a) $2x + y$;

$2x+y = (2, 0, 4) + (-1, 3, 1) = (1, 3, 5)$.

b) $x \cdot y$;

$x \cdot y = (1*-1 + 0*3 + 2*1) = 1$.

c) $||x||$ and $||y||$;

$||x|| = \sqrt{1^2 + 0^2 + 2^2} = \sqrt{1 + 4} = \sqrt{5}$.

$||y|| = \sqrt{(-1)^2 + 3^2 + 1^2} = \sqrt{1 + 9 + 1} = \sqrt{11}$.

d) The cosine of the angle between x and y;

$x \cdot y = ||x|| ||y|| \cos(\theta)$

$\implies \cos(\theta) = \frac{x \cdot y}{||x|| ||y||}$

$= \cos(\theta) = \frac{1}{\sqrt{5} \cdot \sqrt{11}} = \frac{\sqrt{55}}{55}$.

e) the distance from x to y.

$d(x, y) = ||y - x|| = ||(-2, 3, -1)|| = \sqrt{(-2)^2 + 3^2 + (-1)^2} = \sqrt{4 + 9 + 1} = \sqrt{14}$.


Problem 4. Prove that the equality holds in Cauchy-Schwartz Inequality if and only if
one of the vectors u and v is a multiple of the other.

The Cauchy Schwartz Inequality is:
$(u \cdot v)^2 \leq (u \cdot u) \cdot (v \cdot v)$.

First direction: If u is a scalar multiple of v, the equality holds.

Suppose that $u = \lambda v$ for some scalar $\lambda$.
Then, $u \cdot v = (\lambda v) \cdot v = \lambda (v \cdot v)$.

So, $(u \cdot v)^2 = \lambda^2 (v \cdot v)^2$. Also, $u \cdot u = \lambda^2 (v \cdot v)$.

So, $(u \cdot v)^2 = \lambda^2 (v \cdot v)^2 = (u \cdot u) \cdot (v \cdot v)$.

Thus, the equality holds.

Now for the other direction: If the equality holds, u is a scalar multiple of v.

Assume the inequality holds.

Now let $w = u - \frac{u \cdot v}{||v||^2} v$.

Using the dot product, we compute that $w \cdot v = (u \cdot v) - \frac{u \cdot v}{||v||^2} (v \cdot v)
= (u \cdot v) - (u \cdot v) = 0$.

That means they are orthogonal.

Now, we use the Pythagorean theorem.

$||w||^2 = ||u||^2 - \frac{(u \cdot v)^2}{||v||^2}$.

But since we have the equality already assumed, this means that $||w||^2 = 0$
(meaning $w$ is the zero-vector), which means that u and v are scalar multiples of each other
($u = \frac{u \cdot v}{||v||^2} v$).


Problem 5. For $x, y \in \RR$, define

$d_1(x, y) = \sqrt{|x - y|}$

$d_2(x, y) = |x^2 - y^2|$

$d_3(x, y) = |x - 2y|$

Determine, for each of these, whether it is a metric or not. Justify your answers.

a) For $d_1$:

Non-negativity: It is non-negative since it's a square root of an absolute value.

Symmetry: $d_1(x, y) = \sqrt{|x-y|} = \sqrt{|y-x|} = d_1(y, x)$.

Triangle inequality: $\sqrt{|x-z|} \leq \sqrt{|x-y|} + \sqrt{|y-z|}$.
Using the properties of square roots, it is equivalent to $|x-z| \leq |x-y| + |y-z|$.

Definiteness: This is obvious from the definition.

$d_1$ is a valid metric.

b) For $d_2$:

This is not a metric because it fails definiteness.
Let $x = -2$ and $y = 2$.
You will get that $d_2(x, y) = 0$ even though $x \neq y$.

$d_2$ is not a valid metric.

c) For $d_3$:

This is not a metric because it fails symmetry.
$d_2(x, y) = |x - 2y|$ is not equal to $d_2(y, x) = |y - 2x|$ in general.

$d_3$ is not a valid metric.


Problem 6. Using only the definition of the limit of a sequence in the Euclidean space $\RR^2$,
prove that
\[\lim_{n \to \infty} \left(\frac{2n}{n+3}, \frac{1-n}{n}\right) = (2, -1)\]

We need to show that
\[\sqrt{(a_n - L)^2 + (b_n - M)^2} < \epsilon\]

where $a_n = \frac{2n}{n+3}$, $b_n = \frac{1-n}{n}$, $L = 2$, $M = -1$:
\[\sqrt{\left(\frac{2n}{n+3} - 2\right)^2 + \left(\frac{1-n}{n} + 1\right)^2} < \epsilon\]

Our goal is to show that for every $\epsilon > 0$, $\exists N \in \NN$ such that
$\forall n \geq N$, the above inequality holds.

The expression simplifies to $\sqrt{\frac{36}{(n+3)^2} + \frac{1}{n^2}} < \epsilon$
$\forall n \geq N$.

Since both terms inside the square root are positive, we can turn it into two separate
inequalities.

We get

\[\frac{36}{(n+3)^2} < \frac{\epsilon^2}{2}\]

\[\frac{1}{n^2} < \frac{\epsilon^2}{2}\]

Solving these yields $n > \sqrt{\frac{72}{\epsilon^2}} - 3$ and $n > \frac{1}{\sqrt{\epsilon/2}}$.

If we choose $N$ to be the larger of these two inequalities, then

$\sqrt{\left(\frac{2n}{n+3} - 2\right)^2 + \left(\frac{1-n}{n} + 1\right)^2} < \epsilon$

$\forall n \geq N$.

\end{document}
