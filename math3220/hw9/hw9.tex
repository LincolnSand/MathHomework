\documentclass[12]{amsart}
\usepackage{amssymb}
\usepackage{eucal}
\usepackage{amscd}
\usepackage{graphicx}
\usepackage{mathtools}
\usepackage{xcolor}
\usepackage{bm}



\newtheorem{thm}{Theorem}
\newtheorem*{extracredit}{Extra Credit Problem}
\newtheorem{lem}[thm]{Lemma}
\newtheorem{cor}[thm]{Corollary}
\newtheorem{pro}[thm]{Proposition}
\theoremstyle{definition}
\newtheorem{rem}[thm]{Remark}

\newtheorem{exm}[thm]{Example}
\newtheorem{xca}{Problem}
\newcommand{\probl}{Problem}
\newtheorem*{exer}{\probl}
\newenvironment{exercise}[1]{\renewcommand{\probl}{#1}\begin{exer}}
{\end{exer}}
\newcommand{\be}{\begin{equation*}}
\newcommand{\ee}{\end{equation*}}
\newcommand{\R}{\mathbb{R}}
\newcommand{\Z}{\mathbb{Z}}
\newcommand{\N}{\mathbb{N}}
\newcommand{\Q}{\mathbb{Q}}
\newcommand{\C}{\mathbb{C}}
\newcommand{\cl}{\overline}
\newcommand{\sskip}{\newpage}

\newcommand{\norm}[1]{\lVert#1\rVert}
\newcommand{\m}[1]{$ #1 $}
\newcommand{\lskip}{\bigskip}
\newcommand{\der}[2]{\frac{\partial#1}{\partial #2}}


\DeclareMathOperator{\im}{im} \DeclareMathOperator{\rank}{rank}
\DeclareMathOperator{\Mor}{Mor} \DeclareMathOperator{\coker}{coker}
\DeclareMathOperator{\supp}{supp} \DeclareMathOperator{\For}{For}
\DeclareMathOperator{\Hom}{Hom} \DeclareMathOperator{\End}{End}
\DeclareMathOperator{\Int}{Int} \DeclareMathOperator{\order}{order}
\DeclareMathOperator{\ad}{ad} \DeclareMathOperator{\Ad}{Ad}
\DeclareMathOperator{\GL}{GL} \DeclareMathOperator{\SL}{SL}
\DeclareMathOperator{\SO}{SO} \DeclareMathOperator{\Sp}{Sp}
\DeclareMathOperator{\SU}{SU} \DeclareMathOperator{\tr}{tr}
\DeclareMathOperator{\Aut}{Aut} \DeclareMathOperator{\re}{Re}
\DeclareMathOperator{\imag}{Im} \DeclareMathOperator{\Card}{Card}
\newcommand{\Sd}{\ensuremath{\surd}}


\begin{document}

\centerline{ \bf Math 3220-1: Homework 9, due 04/10/2024}
\bigskip
\centerline{ \bf Show all work. Homework has to be uploaded to GradeScope.}
\bigskip
\noindent Name (PRINT): Lincoln Sand\hskip 2.5in ID: u1358804
\smallskip

\hrule

\bigskip


%%%%%%%%%%%%%%%%%%%%%%%%%%%%%%%%%%%%%%%%%%%%%%%%%%%%%%%%%%%%%%%%%%%%%%%%%%%%%%%%%%%

\begin{xca} %1
Let $f:\R^2\to\R^2$ be defined by the equation
$$
f(x,y)=(e^x\cos{y}, e^x\sin{y}).
$$
\begin{itemize}
\item [(a)] Show that $f$ is one-to-one on the set $A=\{(x,y)\in\R^2: 0<y<2\pi\}.$
\smallskip
\item[(b)]What is the set $B=f(A)$?
\smallskip
\item[(c)] If $g$ is the inverse function, find $Dg(0,1)$.
\end{itemize}
\end{xca}


a) Suppose $f(x_1, y_1) = f(x_2, y_2)$. Then:

\[(e^{x_1} \cos(y_1), e^{x_1} \sin(y_1)) = (e^{x_2} \cos(y_2), e^{x_2} \sin(y_2))\]

This gives us two equations:
\[e^{x_1} \cos(y_1) = e^{x_2} \cos(y_2)\]
\[e^{x_1} \sin(y_1) = e^{x_2} \sin(y_2)\]

Dividing the second equation by the first (assuming none of the $\cos$ terms
are $0$, which is guarenteed by $0 < y < 2 \pi$):
\[\frac{\sin(y_1)}{\cos(y_1)} = \frac{\sin(y_2)}{\cos(y_2)} \implies \tan(y_1) = \tan(y_2)\]

Since $y_1$ and $y_2$ are in $(0, 2 \pi)$, $\tan$ is injective. This means that
$y_1$ = $y_2$.

This means we can reduce the equation to:
\[e^{x_1} = e^{x_2} \implies x_1 = x_2\]

Thus, $(x_1, y_1) = (x_2, y_2)$ and $f$ is one-to-one.

b) Since $x$ can be any real number, $x^{x}$ covers all positive real numbers.

Since $y \neq 0$ and $y \neq 2 \pi$, the second term cannot be $0$.

So, the set $B$ is $\{(u, v) \in \R^2 : u > 0\} \cup \{(u, v) \in \R^2 : u = 0, v > 0\}$.

c) To find $Dg(0, 1)$, we have to find a point $(x_0, y_0)$ in $A$ such that
$f(x_0, y_0) = (0, 1)$, compute $Df(x_0, y_0)$, and then invert it.

\[Df(x, y) = \begin{bmatrix}
    e^{x} \cos(y) & -e^{x} \sin(y) \\
    e^{x} \sin(y) & e^{x} \cos(y)
\end{bmatrix}\]

\[Df\left(0, \frac{\pi}{2}\right) = \begin{bmatrix}
    0 & -1 \\
    1 & 0
\end{bmatrix}\]

The inverse is $\begin{bmatrix}
    0 & 1 \\
    -1 & 0
\end{bmatrix}$.

Thus, $Dg(0, 1) = \begin{bmatrix}
    0 & 1 \\
    -1 & 0
\end{bmatrix}$.


\sskip
\begin{xca} %2
Use the Inverse Function Theorem to determine the points of $\R$ near which the $\sin$ function has a smooth local inverse function.
What is the derivative of the inverse function when it exists?
\end{xca}


$\sin : \R \to \R$ is continuously differentiable everywhere.

The derivative of $\sin(x)$ is $\cos(x)$.

To find where the $\sin$ function has a smooth local inverse,
we need to determine where $\cos(x) \neq 0$.

$\cos(x) = 0$ when $x = \frac{\pi}{2} + k \pi$ for any integer $k$.
So, the $\sin$ function will have a smooth local inverse at points
where $x \neq \frac{\pi}{2} + k \pi$.

Now, we have to find the derivative of the inverse function.

If $y = \sin^{-1}(x)$, then:
\[\frac{dx}{dy} = \frac{1}{\frac{dy}{dx}}\]

Given $y = \sin(x)$, and $\frac{dy}{dx} = \cos(x)$, we have:
\[\frac{d}{dx}\left(\sin^{-1}(x)\right) = \frac{1}{\cos(y)}\]

Since $y = \sin^{-1}(x)$ and using the identity that $\sin^{2}(x) + \cos^{2}(x) = 1$,
we can write:
\[\cos(y) = \sqrt{1 - \sin^{2}(y)} = \sqrt{1 - x^2}\]

So, the derivative of the inverse $\sin$ function is:
\[\frac{d}{dx}\left(\sin^{-1}(x)\right) = \frac{1}{\sqrt{1 - x^2}}\]

The derivative exists and is valid for $-1 < x < 1$, correponding to where
the $\sin$ function is defined and smooth, excluding the points where $\cos(x) = 0$.


\sskip
\begin{xca} %3
Are there any points on the graph of the equation $x^3+3xy^2+2y^3=1$ where it may not be possible to solve for $y$ as a smooth function of $x$ in some neighborhood of the point?
\end{xca}


We'll use the Implicit Function Theorem for this.

We need to find where the partial derivative of the function with respect to $y$,
$\frac{\partial}{\partial y} = 6xy + 6y^2$, equals $0$;
because at these points the Implicit Function Theorem
does not guarentee existence of a smooth function.

$6xy + 6y^2 = 0$ when $y = 0$ or $y = -x$.

When $y = 0$, substituting into the original equation $x^3 + 3xy^2 + 2y^3 = 1$
gives $x^3 = 1$, so $x = 1$. Thus, the point of interest is $(1, 0)$.

When $y = -x$, substituting into the original equation gives $-4x^3 = 1$,
resulting in $x^3 = -\frac{1}{4}$.

Thus, it might not be possible to solve for a smooth function of $x$
in some neightborhood of $(1, 0)$. At this point, the Implicit Function
Theorem does not guarentee existence of a smooth function because the partial
derivative with respect to $y$ is $0$.

The other place is the intersection of $y = -x$ and $x^3 = -\frac{1}{4}$.
This intersection occurs at $x = \left(-\frac{1}{4}\right)^{1/3}$.
Thus, a smooth function of $x$ is also not guarenteed in the neighborhood
of $(\left(-\frac{1}{4}\right)^{1/3}, -\left(-\frac{1}{4}\right)^{1/3})$.


\sskip
\begin{xca} %4
Consider the set $S$ described by the equation
$$
xz+yz+\sin{(x+y+z)}=0.
$$
Can $S$ be represented as a smooth parameterized 2-surface near the point $(0,0,0)$? (justify your answer). If so, find an equation of a tangent space to $S$ at this point.
\end{xca}


Let's first find the gradient of the function:
\[\nabla F = \left(z + \cos(x+y+z), z + \cos(x+y+z), x + y + \cos(x+y+z)\right)\]

Evaluated at the point $(0, 0, 0)$ gives $(1, 1, 1)$.

Since the gradient is non-zero at $(0, 0, 0)$, the set $S$ can be represented
as a smooth parameterized 2-surface near this point.

Let's now find it. The tangent space is given by:
\[\nabla F(a, b, c) \cdot (x-a, y-b, z-c) = 0\]

Substituting in $\nabla F(0, 0, 0) = (1, 1, 1)$ and $(a, b, c) = (0, 0, 0)$ gives:
\[1 \cdot (x-0) + 1 \cdot (y-0) + 1 \cdot (z-0) = 0\]

which simplifies to:
\[x + y + z = 0\]

This is the equation for the tangent space to $S$ at $(0, 0, 0)$.


\sskip
\begin{xca} %5

For the system of equations
\begin{align}
\notag x^2+y^2-z^2 & =0,\\
\notag x+y+z &=0,
\end{align}
at which points of the solution set $S$ is there a neighborhood in which $S$ is a smooth curve? At each such point  find an equation of the tangent line.
\end{xca}


First let's compute the gradients:
\[(2x, 2y, -2z)\]
\[(1, 1, 1)\]

For these to be parallel, then $x = y = -z$ must be true. From the second
equation we are given, $x + y + z = 0$, we have that $-z = x + y$. This means that
$x = y = x+y$, which is only true if $x = y = -z = 0$. This is the zero case from
before, which means that the two vectors are always linearly independent
except at $(0, 0, 0)$.

So, at $(0, 0, 0)$, the condition for the existence of a smooth curve is not
satisfied, but it is satisfied everywhere else.

Substituting $z = -x -y$ into the first equation gives $-2xy = 0 \implies xy = 0$.
This means that $y = 0$ when $x \neq 0$.
Using the second equation gives $z = -x$.

So, the points of the solution set $S$ are of the form $(x, 0, -x)$
for $x \neq 0$.

This means that $\nabla F = (2x, 0, 2x)$.

Now, to compute the normal of the tangent space, we need to compute
$\nabla F(x, 0, -x) \times \nabla G(x, 0, -x) = (2x, 0, 2x) \times (1, 1, 1)$.

This works out to $(-2x, 0, 2x)$.

Using the point-direction form of a line, we eventually get:
\[\frac{z+x}{2x} = t\]

This means that $x = x$, $y = 0$, and $z = 2xt - x$.

For each value of $t$, this set of equations describes the coordinates of a point
on the tangent line at $(x, 0, -x)$ on the solution set $S$.


\end{document}
