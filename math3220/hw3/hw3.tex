\documentclass{amsart}
\usepackage{amssymb}
\usepackage{eucal}
\usepackage{amscd}
\usepackage{graphicx}
\usepackage{mathtools}
\usepackage{xcolor}
\usepackage{bm}


\newtheorem{thm}{Theorem}
\newtheorem*{extracredit}{Extra Credit Problem}
\newtheorem{lem}[thm]{Lemma}
\newtheorem{cor}[thm]{Corollary}
\newtheorem{pro}[thm]{Proposition}
\theoremstyle{definition}
\newtheorem{rem}[thm]{Remark}

\newtheorem{exm}[thm]{Example}
\newtheorem{xca}{Problem}
\newcommand{\probl}{Problem}
\newtheorem*{exer}{\probl}
\newenvironment{exercise}[1]{\renewcommand{\probl}{#1}\begin{exer}}
{\end{exer}}
\newcommand{\be}{\begin{equation*}}
\newcommand{\ee}{\end{equation*}}
\newcommand{\R}{\mathbb{R}}
\newcommand{\Z}{\mathbb{Z}}
\newcommand{\N}{\mathbb{N}}
\newcommand{\Q}{\mathbb{Q}}
%\newcommand{\e}{\text{e}}
\newcommand{\C}{\mathbb{C}}

\newcommand{\cl}{\overline}
\newcommand{\sskip}{\bigskip}
\newcommand{\norm}[1]{\lVert#1\rVert}
\newcommand{\m}[1]{$ #1 $}
\newcommand{\lskip}{\newpage}


\DeclareMathOperator{\im}{im} \DeclareMathOperator{\rank}{rank}
\DeclareMathOperator{\Mor}{Mor} \DeclareMathOperator{\coker}{coker}
\DeclareMathOperator{\supp}{supp} \DeclareMathOperator{\For}{For}
\DeclareMathOperator{\Hom}{Hom} \DeclareMathOperator{\End}{End}
\DeclareMathOperator{\Int}{Int} \DeclareMathOperator{\order}{order}
\DeclareMathOperator{\ad}{ad} \DeclareMathOperator{\Ad}{Ad}
\DeclareMathOperator{\GL}{GL} \DeclareMathOperator{\SL}{SL}
\DeclareMathOperator{\SO}{SO} \DeclareMathOperator{\Sp}{Sp}
\DeclareMathOperator{\SU}{SU} \DeclareMathOperator{\tr}{tr}
\DeclareMathOperator{\Aut}{Aut} \DeclareMathOperator{\re}{Re}
\DeclareMathOperator{\imag}{Im} \DeclareMathOperator{\Card}{Card}
\newcommand{\Sd}{\ensuremath{\surd}}



\begin{document}
\centerline{ \bf Math 3220-1: Homework 3, due 01/31/2024 }
\bigskip
\centerline{ \bf Show all work. Homework has to be uploaded to GradeScope.}
\bigskip
\noindent Name (PRINT):\hskip 2.5in ID:
\smallskip

\hrule

\bigskip
Unless stated otherwise, $\R^d$ is the Euclidean space with the "usual" inner product, norm and metric.

\bigskip

\begin{xca}
Prove that three norms we  defined on $\R^d$ are related as follows:
$$
d^{-1}\norm{x}_1\leq \norm{x}_\infty\leq \norm{x}\leq \norm{x}_1.
$$
\end{xca}

Proof of $d^{-1}\norm x_1 \leq \norm x_{\infty}$:

The $\norm{x}_\infty$-norm is the max value of its components.

The $\norm{x}_1$-norm is the sum of the absolute values of its components.

Therefore, $\norm{x}_\infty$-norm is less than or equal to each term
in the sum $\norm{x}_1$-norm.

Since there are d terms in $\norm{x}_1$, $d \cdot \norm{x}_infty$ is less than or equal
to $\norm{x}_1$, or equivalently, $d^{-1} \norm[x]_1 \leq \norm{x}_\infty$.


Proof of $\norm x_{\infty} \leq \norm x$:

Since $|x_i| \leq \norm{x}_infty$ for all i, and given the definition of $\norm{x}$,
each component's square contributes to $\norm{x}^2$.

It follows that $\norm{x}_infty^2$ is less than or equal to the sum of the squares of
each component.

Taking the square root gives us the desired inequality.


Proof of $\norm x \leq \norm x_1$:

By the Cauchy-Schwarz inequality, $\left(\sum_{i=1}^d |x_i|\right)^2 \leq d \sum_{i=1}^d x_i^2$.

Taking the square roots of each side, we get that:

$\norm{x}_1 \leq \sqrt{d} \cdot \norm{x}$.

Since $\sqrt{d} \geq 1$, we get $\norm{x} \leq \norm{x}_1$.


Combining these three inequalities together gives us the desired proposition.


\lskip




\begin{xca}

Let $$D=\{(x_1,x_2)\in \R^2:  x_2\in \mathbb{Q}\}$$
Find
\begin{itemize}
\item [(a)] The interior $D^\circ$ of $D$
\smallskip
\item [(b)] The closure $\cl{D}$ of $D$
\smallskip
\item [(c)] The boundary $\partial(D)$ od $D$
\end{itemize}
\smallskip
Justify your answers.
\end{xca}


a) Interior: Because any open ball is sure to contain
irrational numbers within it in $\R^2$, then the interior must be $\emptyset$.

$D^\circ = \emptyset$

b) Closure: Because no matter the value, for any given real number, you can find a
rational number that is arbitrarily close to it, the closure is all of $\R^2$.

$\cl{D} = \R^2$

c) Boundary:

$\partial(D) = \R^2 \setminus \emptyset = \R^2$



\lskip






\begin{xca}
Let $$D=\{(x_1,x_2)\in \R^2:  \norm{(x_1, x_2)}<1\}\cup \{(x_1,x_2):x_2=0, -2<x_1<2\}$$
Find
\begin{itemize}
\item [(a)] The interior $D^\circ$ of $D$
\smallskip
\item [(b)] The closure $\cl{D}$ of $D$
\smallskip
\item [(c)] The boundary $\partial(D)$ od $D$
\end{itemize}
\smallskip
Justify your answers.
\end{xca}


a) The first part (in the union) is already an interior since it is the open disk
in $\R^2$.

The second part is a line segment in $\R^2$ and since you cannot fit a ball around
any point and be contained on the line segment in $\R^2$, its interior is $\emptyset$.

Therefore, $D^\circ = \{(x_1,x_2)\in \R^2:  \norm{(x_1, x_2)}<1\} \cup \emptyset = \{(x_1,x_2)\in \R^2:  \norm{(x_1, x_2)}<1\}$

b) The closure is the same as $D$, but unioned with the endpoints of the line segment
as well as the boundary line of the unit disk.

So, $\cl{D} = \{(x_1,x_2)\in \R^2:  \norm{(x_1, x_2)} \leq 1\}\cup \{(x_1,x_2):x_2 = 0, -2 \leq x_1 \leq 2\}$

c) The boundary is the boundary line of the unit disk unioned with the line segment
plus the line segment's endpoints, but excluding the parts of the line segment that fall
within the unit disk interior.

In other words, it is given by:

$\partial{D} = \{(x_1,x_2)\in \R^2:  \norm{(x_1, x_2)} = 1\} \cup \{(x_1,x_2):x_2=0, -2 \leq x_1 \leq -1\} \cup \{(x_1,x_2):x_2=0, 1 \leq x_1 \leq 2\}$



\lskip





\begin{xca}
Let $D$ be a subset of $X$ and $c$ an element of $\cl{D}-D$. Prove that every neighborhood of $c$ contains infinitely many points of $D$.
\end{xca}

Let's prove this bu contradiction.

Assume that $\exists$ a neighborhood of c that contains only a finite number of
points of D. Let this neighborhood be N. Since N is a neighborhood of c, it contains an
open set O such that $c \in O$.

Because c is a limit point of D, every open set containing c must intersect D
in some point other than c. Given our assumption that N contains only a finite number
of points of D, we can remove these finite points from O to form a new set O' that
still contains c.

However, O' does not contain any points of D, which contradicts the definition of c
being a limit point of D. A limit point is defined as a point where every neighborhood
intersects with the set in at least one point other than itself.

Since assuming the negative leads to a contradiction, the proposition must be true.

\lskip


\end{document}
