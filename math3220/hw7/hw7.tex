\documentclass[12]{amsart}
\usepackage{amssymb}
\usepackage{eucal}
\usepackage{amscd}
\usepackage{graphicx}
\usepackage{mathtools}
\usepackage{xcolor}
\usepackage{bm}


\newtheorem{thm}{Theorem}
\newtheorem*{extracredit}{Extra Credit Problem}
\newtheorem{lem}[thm]{Lemma}
\newtheorem{cor}[thm]{Corollary}
\newtheorem{pro}[thm]{Proposition}
\theoremstyle{definition}
\newtheorem{rem}[thm]{Remark}

\newtheorem{exm}[thm]{Example}
\newtheorem{xca}{Problem}
\newcommand{\probl}{Problem}
\newtheorem*{exer}{\probl}
\newenvironment{exercise}[1]{\renewcommand{\probl}{#1}\begin{exer}}
{\end{exer}}
\newcommand{\be}{\begin{equation*}}
\newcommand{\ee}{\end{equation*}}
\newcommand{\R}{\mathbb{R}}
\newcommand{\Z}{\mathbb{Z}}
\newcommand{\N}{\mathbb{N}}
\newcommand{\Q}{\mathbb{Q}}
\newcommand{\C}{\mathbb{C}}
\newcommand{\cl}{\overline}
\newcommand{\sskip}{\newpage}
\newcommand{\norm}[1]{\lVert#1\rVert}
\newcommand{\m}[1]{$ #1 $}
\newcommand{\lskip}{\bigskip}


\DeclareMathOperator{\im}{im} \DeclareMathOperator{\rank}{rank}
\DeclareMathOperator{\Mor}{Mor} \DeclareMathOperator{\coker}{coker}
\DeclareMathOperator{\supp}{supp} \DeclareMathOperator{\For}{For}
\DeclareMathOperator{\Hom}{Hom} \DeclareMathOperator{\End}{End}
\DeclareMathOperator{\Int}{Int} \DeclareMathOperator{\order}{order}
\DeclareMathOperator{\ad}{ad} \DeclareMathOperator{\Ad}{Ad}
\DeclareMathOperator{\GL}{GL} \DeclareMathOperator{\SL}{SL}
\DeclareMathOperator{\SO}{SO} \DeclareMathOperator{\Sp}{Sp}
\DeclareMathOperator{\SU}{SU} \DeclareMathOperator{\tr}{tr}
\DeclareMathOperator{\Aut}{Aut} \DeclareMathOperator{\re}{Re}
\DeclareMathOperator{\imag}{Im} \DeclareMathOperator{\Card}{Card}
\newcommand{\Sd}{\ensuremath{\surd}}


\begin{document}

\centerline{ \bf Math 3220-1: Homework 7, due 03/13/2024}
\bigskip
\centerline{ \bf Show all work. Homework has to be uploaded to GradeScope.}
\bigskip
\noindent Name (PRINT): Lincoln Sand\hskip 2.5in ID: u1358804
\smallskip

\hrule

\bigskip
\begin{xca} %4
Let
$$
C=
\begin{bmatrix}
1 & -1\\
4& -6\\
-1&2
\end{bmatrix}
\text{ and } D=
\begin{bmatrix}
2&0 &1 \\
-1 & 1 &3
\end{bmatrix}
$$
Find $CD$ and $DC$.
\end{xca}


$CD:$

$2 \cdot \begin{bmatrix}
    1 \\
    4 \\
    -1
\end{bmatrix} -1 \cdot \begin{bmatrix}
    -1 \\
    -6 \\
    2
\end{bmatrix} = \begin{bmatrix}
    3 \\
    14 \\
    -4
\end{bmatrix}$

$0 \cdot \begin{bmatrix}
    1 \\
    4 \\
    -1
\end{bmatrix} + 1 \cdot \begin{bmatrix}
    -1 \\
    -6 \\
    2
\end{bmatrix} = \begin{bmatrix}
    -1 \\
    -6 \\
    2
\end{bmatrix}$

$1 \cdot \begin{bmatrix}
    1 \\
    4 \\
    -1
\end{bmatrix} + 3 \cdot \begin{bmatrix}
    -1 \\
    -6 \\
    2
\end{bmatrix} = \begin{bmatrix}
    -2 \\
    -14 \\
    5
\end{bmatrix}$

So, $CD = \begin{bmatrix}
    3 & -1 & -2 \\
    14 & -6 & -14 \\
    -4 & 2 & 5
\end{bmatrix}$.

$DC:$

$1 \cdot \begin{bmatrix}
    2 \\
    -1
\end{bmatrix} + 4 \cdot \begin{bmatrix}
    0 \\
    1
\end{bmatrix} -1 \cdot \begin{bmatrix}
    1 \\
    3
\end{bmatrix} = \begin{bmatrix}
    1 \\
    0
\end{bmatrix}$

$-1 \cdot \begin{bmatrix}
    2 \\
    -1
\end{bmatrix} -6 \cdot \begin{bmatrix}
    0 \\
    1
\end{bmatrix} + 2 \cdot \begin{bmatrix}
    1 \\
    3
\end{bmatrix} = \begin{bmatrix}
    0 \\
    1
\end{bmatrix}$

So, $CD = \begin{bmatrix}
    1 & 0 \\
    0 & 1
\end{bmatrix}$.


\sskip
\begin{xca}
Let
$$
A=
\begin{bmatrix}
3 & -1\\
2 & 1
\end{bmatrix}
\text{ and } B=
\begin{bmatrix}
2 & 5 \\
-2 & 2
\end{bmatrix}
$$
Find  $\det(A)$, $\det(B)$, $A^{-1}$, $B^{-1}$.
\end{xca}


$\det(A) = 3 \cdot 1 - (-1 \cdot 2) = 3 + 2 = 5$.

$\det(B) = 2 \cdot 2 - (5 \cdot -2) = 4 + 10 = 14$.

$A^{-1} = \frac{1}{5} \begin{bmatrix}
    1 & 1 \\
    -2 & 3
\end{bmatrix}$.

$B^{-1} = \frac{1}{14} \begin{bmatrix}
    2 & -5 \\
    2 & 2
\end{bmatrix}$.

\sskip

\begin{xca}
Find the matrix of the linear transformation of $\R^2$ which reflects each point through the diagonal line $y=x$ (this transformation
interchanges $x$ and $y$ coordinates of each point).
\end{xca}


Let's look at what happens to the bases:

$e_1 = \begin{bmatrix}
    1 \\
    0
\end{bmatrix} \to \begin{bmatrix}
    0 \\
    1
\end{bmatrix}$

$e_2 = \begin{bmatrix}
    0 \\
    1
\end{bmatrix} \to \begin{bmatrix}
    1 \\
    0
\end{bmatrix}$


Thus, the transformation matrix is: $\begin{bmatrix}
    0 & 1 \\
    1 & 0
\end{bmatrix}$.



\sskip



\begin{xca}
Prove that if $K:\R^p \to \R^q$ and $L:\R^q\to \R^r$ are linear transformations, then
$$
\norm{L\circ K}\leq \norm{L}\,\norm{K}
$$
\end{xca}


Recall:
\[\norm{T} = \sup_{\norm{x} = 1} \norm{T(x)}\]


Consider an arbitrary vector $x \in \R^p$ with $\norm{x} = 1$.

First, observe that:
\[\norm{K(x)} \leq \norm{K} \norm{x} = \norm{K}\]

And:
\[\norm{L\circ K} = \norm{L(K(x))}\]

Applying the definition of the norm of $L$, we get that:
\[\norm{L(K(x))} \leq \norm{L} \norm{K(x)}\]

But, we've established that $\norm{K(x)} \leq \norm{K}$, so:
\[\norm{L(K(x))} \leq \norm{L} \norm{K}\]

Since this inequality holds for any $x$ with $\norm{x} = 1$, it also
holds for the supremum of $\norm{(L\circ K)(x)}$ over all such $x$,
which is exactly $\norm{L\circ K}$:
\[\norm{L\circ K} \leq \norm{L} \norm{K}\]

\[\qed\]


\end{document}
