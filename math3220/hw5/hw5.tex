\documentclass[12]{amsart}
\usepackage{amssymb}
\usepackage{eucal}
\usepackage{amscd}
\usepackage{graphicx}
\usepackage{mathtools}
\usepackage{xcolor}
\usepackage{bm}



\newtheorem{thm}{Theorem}
\newtheorem*{extracredit}{Extra Credit Problem}
\newtheorem{lem}[thm]{Lemma}
\newtheorem{cor}[thm]{Corollary}
\newtheorem{pro}[thm]{Proposition}
\theoremstyle{definition}
\newtheorem{rem}[thm]{Remark}

\newtheorem{exm}[thm]{Example}
\newtheorem{xca}{Problem}
\newcommand{\probl}{Problem}
\newtheorem*{exer}{\probl}
\newenvironment{exercise}[1]{\renewcommand{\probl}{#1}\begin{exer}}
{\end{exer}}
\newcommand{\be}{\begin{equation*}}
\newcommand{\ee}{\end{equation*}}
\newcommand{\R}{\mathbb{R}}
\newcommand{\Z}{\mathbb{Z}}
\newcommand{\N}{\mathbb{N}}
\newcommand{\Q}{\mathbb{Q}}
\newcommand{\C}{\mathbb{C}}
\newcommand{\cl}{\overline}
\newcommand{\sskip}{\newpage}
\newcommand{\norm}[1]{\lVert#1\rVert}
\newcommand{\m}[1]{$ #1 $}
\newcommand{\lskip}{\bigskip}


\DeclareMathOperator{\im}{im} \DeclareMathOperator{\rank}{rank}
\DeclareMathOperator{\Mor}{Mor} \DeclareMathOperator{\coker}{coker}
\DeclareMathOperator{\supp}{supp} \DeclareMathOperator{\For}{For}
\DeclareMathOperator{\Hom}{Hom} \DeclareMathOperator{\End}{End}
\DeclareMathOperator{\Int}{Int} \DeclareMathOperator{\order}{order}
\DeclareMathOperator{\ad}{ad} \DeclareMathOperator{\Ad}{Ad}
\DeclareMathOperator{\GL}{GL} \DeclareMathOperator{\SL}{SL}
\DeclareMathOperator{\SO}{SO} \DeclareMathOperator{\Sp}{Sp}
\DeclareMathOperator{\SU}{SU} \DeclareMathOperator{\tr}{tr}
\DeclareMathOperator{\Aut}{Aut} \DeclareMathOperator{\re}{Re}
\DeclareMathOperator{\imag}{Im} \DeclareMathOperator{\Card}{Card}
\newcommand{\Sd}{\ensuremath{\surd}}


\begin{document}

\centerline{ \bf Math 3220-1: Homework 5, due 02/21/2024}
\bigskip
\centerline{ \bf Show all work}
\bigskip
\noindent Name (PRINT): Lincoln Sand \hskip 2.5in ID: u1358804
\smallskip

\hrule

\bigskip
\
\begin{xca} %1
Let  $f$ be the function $f:\R-\{0\}\to \R$ defined by
$$
f(x)=\sin{(1/x)}
$$
Show that $\lim_{x\to 0}f(x)$ does not exist.

\bigskip
Does $f$ have a continuous extension to $\R$?  Justify your answer.

\bigskip
Comment: This is a review problem from Math 3210.

\end{xca}


To demonstrate that $f(x) = \sin(1/x)$ has no limit as x approaches 0,
we must show that the function values do not approach a single finite value
as x gets arbitrarily close to 0.

The function is defined on $\R \setminus \{0\}$, so x can be any value except 0 itself.

Let us first note the fact that $\sin(1/x)$ oscillates infinitely as x approaches 0.
This is because as x gets closer to 0, 1/x grows without any bound, which causes
the sine function to oscillate between -1 and 1 infinitely many times in any
neighborhood of 0. This means there is no value approached by f within
any neighborhood of 0.

Let us consider two sequences that approach 0: $x_n = \frac{1}{2 n \pi}$ and
$y_n = \frac{1}{(2 n + 1) \pi}$, where n is a positive integer.
Notice that $f(x_n) = \sin(2 n \pi) = 0$ and $f(y_n) = \sin((2 n + 1) \pi) = 0$.
However, if we choose other sequences that approach 0, we can get f to oscillate
between -1 and 1, showing that the limit is not stable.

From above, for any proposed limit L as x approaches 0, we can always find
values of x arbitrarily close to 0 for which f(x) is not arbitrarily close
to L (due to the oscillatory nature of sine).


For a function to have a continuous extension to $\R$, it must be possible
to define the function at the points where it is currently undefined in a way
such that the extended function is continuous at those points. For the function
$f(x) = \sin(1/x)$, this would mean defining $f(0)$.

However, since $\lim_{x \to 0} f(x)$ does not exist, there is no single real number
that we could assign to $f(0)$ to make f continuous at 0. This means that we can not
define a continuous extension to $\R$ for the function $f(x) = \sin(1/x)$.


\sskip

\begin{xca}%1
Consider the function $f:\mathbb R^2\to \mathbb R$ defined by

$$
f(x,y)=
\begin{cases}
\dfrac{xy^2}{x^2+y^2} &\text{if $(x,y)\neq(0,0)$}\\
0 &\text{if $(x,y)=(0,0)$}.
\end{cases}
$$
Is this function continuous at $(0,0)$? Justify your answer.

\end{xca}

To determine if $f : \R^2 \to R$ is continuous at $(0, 0)$,
we need to check if $\lim_{(x,y) \to (0, 0)} = f(0, 0)$.

A function is continuous at a point if the limit of the function as it approaches
that point is equal to the function's value at that point.

Here, $f(0, 0) = 0$, so we need to verify if:
\[\lim_{(x, y) \to (0, 0)} \frac{x y^2}{x^2 + y^2} = 0\]

We can solve this with direct substitution techniques.
Let $x = r \cos(\theta)$ and $y = r \sin(\theta)$. In polar coordinates,
the limit $(x, y) \to (0, 0)$ is equivalent to $r \to 0$ regardless of $\theta$.

\[\lim_{(x, y) \to (0, 0)} \frac{x y^2}{x^2 + y^2} = \lim_{r \to 0} \frac{r \cos(\theta) r^2 \sin^2(\theta)}{r^2 (\cos^2(\theta) + \sin^2(\theta))}\]
\[= \lim_{r \to 0} \frac{r^3 \cos(\theta) \sin^2(\theta)}{r^2} = \lim_{r \to 0} r \cos(\theta) \sin^2(\theta)\]

Note that $|\cos(\theta) \sin^2(\theta)| \leq 1$ $\forall \theta$, so the above limit
approaches 0 as r approaches 0. This suggests that the function $f(x, y)$
approaches 0 as $(x, y)$ approaches $(0, 0)$, which matches $f(0, 0)$.
Thus, $f$ is continuous at $(0, 0)$.


\sskip

\begin{xca} %2
Does the function $f:\mathbb R^2-\{0,0\}\to \mathbb R$ defined by

$$
f(x,y)=\frac{x}{\sqrt{x^2+y^2}}
$$
have a limit as $(x,y)$ approaches $(0,0)$? Justify your answer.

\end{xca}

We can use a similar technique as the previous problem.
With $r = \cos(\theta)$ and $r = \sin(\theta)$, and the resulting equality:
$r = \sqrt{x^2 + y^2}$, we have:

\[\frac{x}{\sqrt{x^2 + y^2}} = \frac{r \cos(\theta)}{r} = \cos(\theta)\]

This expression depends solely on $\theta$ and not r. This means the behavior of
f as $(x, y)$ approaches $(0, 0)$ depends on the angle of the path approaching it.

We can show this further by using path testing. Let's consider approaching
$(0, 0)$ along different paths:

Along the x-axis: If $y = 0$, then $\theta = 0$ or $\pi$ and $\cos(\theta) = 1$ or $-1$.
So, $f(x, 0) = 1$ as x approaches 0 from the positive side and $f(x, 0) = -1$
as x approaches 0 from the negative side.

Along the y-axis: If $x = 0$, then $\theta = \pi/2$ or $-\pi/2$, and $\cos(\theta) = 0$.
So, $f(0, y) = 0$.

Since the limit along different paths differ as $(x, y)$ approach $(0, 0)$
ranges from $-1$ to 1 (depending on the path), this means the limit of
$f(x, y)$ as $(x, y)$ approaches $(0, 0)$ does not exist.

\sskip


\begin{xca} %2
Let $X$ be a metric space, and let $c\in X$. Show that the function $f:X\to\R$ defined by
$$
f(x)=d(c,x)
$$
is continuous. (hint: use the inequality $|d(x,z)-d(y,z)|\leq d(x,y)$ for any $x, y, z\in X$.

\end{xca}

Let $x, y \in X$ and $\epsilon > 0$ be given. We want to show that
$\exists \delta > 0$ such that if $d(x, y) < \delta$, then $|f(x) - f(y)| < \epsilon$.

The function $f$ is defined as $f(x) = d(c, x)$ $\forall x \in X$. Therefore,
the difference $|f(x) - f(y)|$ can be written as:
\[|f(x) - f(y)| = |d(c, x) - d(c, y)|\]
Using the given inequality, we have:
\[|d(c, x) - d(c, y)| \leq d(x, y)\]

For continuity, we want $|d(c, x) - d(c, y)| < \epsilon$. From the above inequality,
this will be satisfied if $d(x, y) < \epsilon$. Thus, we can choose $\delta = \epsilon$.

Thus, for any $\epsilon > 0$, if we set $\delta = \epsilon$, then whenever
$d(x, y) < \delta$, it follows that $|f(x) - f(y)| < \epsilon$.
This shows that $f(x) = d(c, x)$ is continuous $\forall x \in X$ since the choice
of $\delta$ does not depend on the specific points x and y, but only on the distance
between them and the fixed point c.


\end{document}
