%&latex
%\documentclass[draft]{amsart}
%\input seteps
\documentclass[12]{amsart}
\usepackage{amssymb}
\usepackage{eucal}
\usepackage{amscd}
\usepackage{graphicx}
\usepackage{mathtools}
\usepackage{xcolor}
\usepackage{bm}


\newtheorem{thm}{Theorem}
\newtheorem*{extracredit}{Extra Credit Problem}
\newtheorem{lem}[thm]{Lemma}
\newtheorem{cor}[thm]{Corollary}
\newtheorem{pro}[thm]{Proposition}
\theoremstyle{definition}
\newtheorem{rem}[thm]{Remark}

\newtheorem{exm}[thm]{Example}
\newtheorem{xca}{Problem}
\newcommand{\probl}{Problem}
\newtheorem*{exer}{\probl}
\newenvironment{exercise}[1]{\renewcommand{\probl}{#1}\begin{exer}}
 {\end{exer}}
 \newcommand{\be}{\begin{equation*}}
 \newcommand{\ee}{\end{equation*}}
 \newcommand{\R}{\mathbb{R}}
\newcommand{\Z}{\mathbb{Z}}
\newcommand{\N}{\mathbb{N}}
\newcommand{\Q}{\mathbb{Q}}
%\newcommand{\e}{\text{e}}
\newcommand{\C}{\mathbb{C}}
%%%
\newcommand{\cl}{\overline}
\newcommand{\sskip}{\bigskip}
\newcommand{\norm}[1]{\lVert#1\rVert}
\newcommand{\m}[1]{$ #1 $}
\newcommand{\lskip}{\newpage}


\DeclareMathOperator{\im}{im} \DeclareMathOperator{\rank}{rank}
\DeclareMathOperator{\Mor}{Mor} \DeclareMathOperator{\coker}{coker}
\DeclareMathOperator{\supp}{supp} \DeclareMathOperator{\For}{For}
\DeclareMathOperator{\Hom}{Hom} \DeclareMathOperator{\End}{End}
\DeclareMathOperator{\Int}{Int} \DeclareMathOperator{\order}{order}
\DeclareMathOperator{\ad}{ad} \DeclareMathOperator{\Ad}{Ad}
\DeclareMathOperator{\GL}{GL} \DeclareMathOperator{\SL}{SL}
\DeclareMathOperator{\SO}{SO} \DeclareMathOperator{\Sp}{Sp}
\DeclareMathOperator{\SU}{SU} \DeclareMathOperator{\tr}{tr}
\DeclareMathOperator{\Aut}{Aut} \DeclareMathOperator{\re}{Re}
\DeclareMathOperator{\imag}{Im} \DeclareMathOperator{\Card}{Card}
\newcommand{\Sd}{\ensuremath{\surd}}



\begin{document}
\centerline{ \bf Math 3220-1: Homework 2, due 01/24/2024 }
\bigskip
\centerline{ \bf Show all work}
\bigskip
\noindent Name (PRINT):\hskip 2.5in ID:
\smallskip

\hrule

\bigskip
Unless stated otherwise, $\R^d$ is the Euclidean space with the "usual" inner product, norm and metric.

\bigskip

\begin{xca}
Let $\{\bm{x}_n\}$ be a bounded sequence in $\R^d$ and let $\{a_n\}$ be a  sequence of scalars converging to $0$.
Prove that the sequence $\{a_n\bm{x}_n\}$ converges to $\bm{0}$ in $\R^d$.

\end{xca}

Using properties of norms, we have:

\[||a_n x_n|| = |a_n| \cdot ||x_n|| \leq |a_n| \cdot M\]

Since $\{a_n\}$ converges to 0, for any given $\epsilon > 0$, we can choose
$\epsilon' = \frac{\epsilon}{m}$. Then $\exists$ $N \in \N$ such that
$\forall n \geq N$, $|a_n| < \frac{\epsilon}{m}$.

Thus, for any $n \geq N$:

\[||a_n x_n|| \leq |a_n| \cdot M < \frac{\epsilon}{M} \cdot M = \epsilon\]

Thus $\{a_n x_n\}$ converges to 0.

\lskip
\begin{xca}
Let $\mathcal{C}([0,1])$ be the space of all real valued continuous functions regarded as a metric space with the distance

$$d(f, g)=\norm{f-g}_\infty$$
Find $d(f,g)$ where

\begin{itemize}

\item [(a)] $f(x)=x$ and $g(x)=x^2$
\item [(b)] $f(x)=x$ and $g(x)=2x$
\end{itemize}

\end{xca}

\[||h||_{\infty} = \sup_{x \in [0, 1]} |h(x)|\]

a) $|x - x^2|$ attains its maximum at $x = \frac{1}{2}$ on the interval $[0, 1]$.
Plugging it in gives a value of $|\frac{1}{2} - \frac{1}{4} = \frac{1}{4}$.
Thus, $d(x, x^2) = \frac{1}{4}$.

b) $|x - 2x| = |x|$, which obviously attains its maximum at $x = 1$ on the interval
$[0, 1]$ with a value of $1$.
Thus, $d(x, 2x) = 1$.

\lskip

\begin{xca}
Let $D=\{(x_1,x_2)\in \R^2: 0<x_1 < 4, \hskip .05in 0 < x_2 < 4\}$.  Find the largest radius $r$ such that the open ball $B_r(\bm{c})$ is contained in $D$, where
\begin{itemize}
  \item [(1)] $\bm{c}=(1,2)$
  \item [(2)] $\bm{c}=(2,2)$
  \item [(3)] $\bm{c}=(2,7/2)$
  \item [(4)] $\bm{c}=(c_1, c_2)$ be an arbitrary point in $D$.
 \end{itemize}
Use (4) to prove that $D$ is an open set in $\R^2$.

\end{xca}

1) For the $x_1$-axis, it's $\min(1, 3) = 1$.
For the $x_2$-axis, its $\min(2, 2) = 2$.
So the answer is $r = \min(x_1, x_2) = 1$.

2) $r = \min(2, 2) = 2$.

3) Similar to in part a, we get that

$x_1 = \min(2, 2) = 2$

$x_2 = \min(\frac{7}{2}, \frac{1}{2}) = \frac{1}{2}$

$r = \min(x_1, x_2) = \frac{1}{2}$

4) $x_1 = \min(c_1, 4 - c_1)$.

$x_2 = \min(c_2, 4 - c_2)$

$r = \min(x_1, x_2) = \min(\min(c_1, 4 - c_1), \min(c_2, 4 - c_2))$.


To show that D is an open set, we need to show that for every point $c \in D$,
$\exists$ a radius $r > 0$ such that $B_r(c) \subset D$. Since $c$ is an arbitrary
point in D and the above calculation for part 4) shows we can always find a radius r,
then D is indeed an open set in $\R^2$.

\lskip

\begin{xca}
Let $X$ be a metric space.
Prove that every finite subset of $X$ is closed (in $X$).
\end{xca}

Let F be a finite subset of X where $F = \{x_1, x_2, \dots, x_n\}$
where $n$ is a finite number and $x_1 \in X$ for $i = 1, 2, \dots, n$.

We need to show that $X \setminus F$ is open.

For any point in $y \in X \setminus F$, since y is not in F, the distance $d(y, x_i)$
is positive/non-zero for $i = 1, 2, \dots, n$.

Now, let $r = \min(d(y, x_1), d(y, x_2), \dots, d(y, x_n))$.
Since all these distances are positive, then r is also positive.

Consider $B_r(y)$. From our choice of r, none of the points in F are in
this open ball. Thus, $B_r(y) \subset X \setminus F$.

Since y is an arbitrary point in $X \setminus F$, then $X \setminus F$ is open.

Therefore, X is closed.


\lskip

\
\begin{xca}
Let $$D=\{(x_1,x_2)\in \R^2:  x_2\in \mathbb{Q}\}$$
Find
\begin{itemize}
\item [(a)] The interior $D^\circ$ of $D$
\smallskip
\item [(b)] The closure $\cl{D}$ of $D$
\smallskip
\item [(c)] The boundary $\partial(D)$ od $D$
\end{itemize}
\smallskip
Justify your answers.
\end{xca}

a) Because of the density of the rational numbers in the reals, any open set
in $\R^2$ will contain irrational numbers for $x_2$.

Thus, there are no open balls in D that are entirely contained in D.

Therefore, $D^\circ = \emptyset$.

b) In $\R^2$, for any point $(a, b)$ where b is irrational, we can find a sequence
of points in D with $x_2$ as rational numbers that converges to $(a, b)$
because the rationals are dense in the reals.

Since for any point in $\R^2$ we can find a sequence in D that converges to it,
the closure of D is $\R^2$.

$\cl{D} = \R^2$.

c) $\partial(D) = \cl(D) \setminus D^\circ = \R^2 \setminus \emptyset = \R^2$.


\lskip


\
\begin{xca}

Let $D$ be a subset of $X$ and $c$ an element of $\cl{D}-D$. Prove that every neighborhood of $c$ contains infinitely many points of $D$.
\end{xca}

Let's consider a neighborhood N of c in X. Since c is a limit point of D,
N intersects D. Goal: Show that this intersection contains infinitely many points of D.

Let's use proof by contradiction. Suppose that N only contains finitely many
points of D. Let those points be $\{d_1, d_2, \dots, d_n\}$. Since c is not in D,
c is different from each $d_i$. Therefore, $d(c, d_i) > 0$.

Let $\epsilon = \min(d(c, d_1), d(c, d_2), \dots, d(c, d_n))$. Now, consider the
neighborhood $B_{\frac{\epsilon}{2}}(c)$. This neighborhood is a subset of N.
However, by our choice of $\epsilon$, $B_{\frac{\epsilon}{2}}$ doesn't contain any
points of D. This contradicts the fact that c is a limit point of c
(defined as every open ball of a limit point contains at least on point of D).

Thus, our assumption that N contains only finitely many points of D cannot be true.
Therefore, we have shown that every neighborhood of c contains infinitely
many points of D.


\lskip


\end{document}
