\documentclass[12]{amsart}
\usepackage{amssymb}
\usepackage{eucal}
\usepackage{amscd}
\usepackage{graphicx}
\usepackage{mathtools}
\usepackage{xcolor}
\usepackage{bm}



\newtheorem{thm}{Theorem}
\newtheorem*{extracredit}{Extra Credit Problem}
\newtheorem{lem}[thm]{Lemma}
\newtheorem{cor}[thm]{Corollary}
\newtheorem{pro}[thm]{Proposition}
\theoremstyle{definition}
\newtheorem{rem}[thm]{Remark}

\newtheorem{exm}[thm]{Example}
\newtheorem{xca}{Problem}
\newcommand{\probl}{Problem}
\newtheorem*{exer}{\probl}
\newenvironment{exercise}[1]{\renewcommand{\probl}{#1}\begin{exer}}
{\end{exer}}
\newcommand{\be}{\begin{equation*}}
\newcommand{\ee}{\end{equation*}}
\newcommand{\R}{\mathbb{R}}
\newcommand{\Z}{\mathbb{Z}}
\newcommand{\N}{\mathbb{N}}
\newcommand{\Q}{\mathbb{Q}}
\newcommand{\C}{\mathbb{C}}
\newcommand{\cl}{\overline}
\newcommand{\sskip}{\newpage}
\newcommand{\norm}[1]{\lVert#1\rVert}
\newcommand{\m}[1]{$ #1 $}
\newcommand{\lskip}{\bigskip}


\DeclareMathOperator{\im}{im} \DeclareMathOperator{\rank}{rank}
\DeclareMathOperator{\Mor}{Mor} \DeclareMathOperator{\coker}{coker}
\DeclareMathOperator{\supp}{supp} \DeclareMathOperator{\For}{For}
\DeclareMathOperator{\Hom}{Hom} \DeclareMathOperator{\End}{End}
\DeclareMathOperator{\Int}{Int} \DeclareMathOperator{\order}{order}
\DeclareMathOperator{\ad}{ad} \DeclareMathOperator{\Ad}{Ad}
\DeclareMathOperator{\GL}{GL} \DeclareMathOperator{\SL}{SL}
\DeclareMathOperator{\SO}{SO} \DeclareMathOperator{\Sp}{Sp}
\DeclareMathOperator{\SU}{SU} \DeclareMathOperator{\tr}{tr}
\DeclareMathOperator{\Aut}{Aut} \DeclareMathOperator{\re}{Re}
\DeclareMathOperator{\imag}{Im} \DeclareMathOperator{\Card}{Card}
\newcommand{\Sd}{\ensuremath{\surd}}


\begin{document}

\centerline{ \bf Math 3220-1: Homework 6, due 02/28/2024}
\bigskip
\centerline{ \bf Show all work. Homework has to be uploaded to GradeScope.}
\bigskip
\noindent Name (PRINT): Lincoln Sand\hskip 2.5in ID: u1358804
\smallskip

\hrule

\bigskip
\begin{xca}
Let $B_1(\textbf{0})$ be the open unit ball in $\R^2$. Is it true that every continuous function $f:B_1(\textbf{0})\to\R$ maps Cauchy sequence into Cauchy sequences? Justify your answer.
\end{xca}

Yes, it is true that every continuous function $f : B_1(0) \to \R$ maps
Cauchy sequences into Cauchy sequences.

Let $\{x_n\}$ be a Cauchy sequence in $B_1(0)$. Since $f$ is continuous,
for any $\epsilon > 0$, $\exists$ $\delta > 0$ such that if $||x - x_0|| < \delta$,
then $|f(x) - f(x_0)| < \epsilon$.

Given that $\{x_n\}$ is a Cauchy sequence, for any $\epsilon > 0$,
$\exists$ an $N$ such that if $\forall m, n \geq N$, $||x_m - x_n|| < \delta$.
Because $f$ is continuous, this implies that $\forall m, n \geq N$,
$|f(x_m) - f(x_n)| < \epsilon$. Therefore, the sequence $\{f(x_n)\}$
is also a Cauchy sequence.

\sskip


\begin{xca} %4
Let $X$ and $Y$ be metric spaces and $f:X\to Y$. Assume that $X$ is a union of two closed sets $C$ and $D$. Assume also that
$$
f|_{C}:C\to Y \text{ and } f|_{D}:D\to Y
$$
are continuous functions. Prove that $f:X\to Y$ is continuous. Here $f|_{C}$ and $f|_{D}$ denote the restrictions of $f$ to $C$ and $D$, respectively.

\noindent (Hint: Prove that for every closed set $E\subset Y$ the set $f^{-1}(E)$ is closed in $X$.)
\end{xca}

For the restrictions $f|_{C}$ and $f|_{D}$, continuity means that for every closed set
$E \subset Y$, $(f|_{C})^{-1}(E) = f^{-1}(E) \cap C$ and
$(f|_{D})^{-1}(E) = f^{-1}(E) \cap D$ are closed in $C$ and $D$, respectively.
Since $C$ and $D$ are closed in $X$, their intersection with any closed set in $X$
is also closed in $X$.

Consider any closed set $E \subset Y$. We need to show that $f^{-1}(E)$
is closed in $X$. By the continuity of $f|_{C}$ and $f|_{D}$,
we know that $(f|_{C})^{-1}(E)$ and $(f|_{D})^{-1}(E)$ are closed in $X$.
Since $f^{-1}(E) = (f^{-1}(E) \cap C) \cup (f^{-1}(E) \cap D)$,
and both $(f^{-1}(E) \cap C)$ and $(f^{-1}(E) \cap D)$ are closed in $X$,
their union, $f^{-1}(E)$, is also closed in $X$.

Therefore, since the pre-image of every closed set $E$ in $Y$ under $f$ is closed in $X$,
$f$ is continuous.

\sskip


\begin{xca}
Consider the function $f:\R^2\to\R$ define by
$$
f(x,y)=
\begin{cases}
xy &\text{if $xy>0$}\\
0 &\text{if $xy\leq 0$}.
\end{cases}
$$
At which points of $\R^2$ is this function continuous?
\end{xca}

$xy > 0$ in two regions: the first quadrant $(x > 0, y > 0)$
and the third quadrant $(x < 0, y < 0)$.
In these regions, $f(x, y) = xy$, which is a polynomial in $x$ and $y$.
Polynomials are continuous everywhere, so $f$ is continuous in these regions.

$xy \leq 0$ in the second quadrant $(x < 0, y > 0)$,
the fourth quadrant $(x > 0, y < 0)$, and along the axes $(x = 0 \text{ or } y = 0)$.
In these regions, $f(x, y) = 0$, which is continuous. So, the function is continuous
within these regions.

Now we need to exame the boundaries of where the function switches definitions.

First, let's examine the axes excluding the origin
($x = 0$ or $y = 0$, but not $x, y = 0$).

The limit as either $x$ approaches $0$ or $y$ approaches $0$ of $xy$
is $0$, which matches the constant function definition, so the function is continuous
at these boundaries.

Now let's handle the origin. For the points approaching from the first and
third quadrants, the value $f(x, y)$ depends on the product $xy$, which approaches
$0$ as $(x, y)$ approaches $(0, 0)$. And for the other two quadrants, the value
is the constant function $0$. Therefore, since the limit $f(x, y)$
as $(x, y)$ approaches $(0, 0)$ is $0$ from every direction, the function
is continuous at the origin.

Thus, the function $f$ is continuous at all points in $\R^2$.

\sskip


\begin{xca}
Is the function $f:\R^2-\{(2,0)\}\to \R$ defined by

$$
f(x,y)=\frac{1}{(x-2)^2+y^2}
$$
uniformly continuous on $B_1(0,0)$? Is it uniformly continuous on $B_2(0,0)$? Justify your answers.
\end{xca}

Uniform continuity on $B_1(0, 0)$:

$B_1(0. 0)$ is the open ball of radius 1 centered at $(0, 0)$.
Within this ball, $f(x, y)$ is defined and continuous since the point
$(2, 0)$, where $f$ is not defined, is outside of $B_1(0, 0)$.

Given the form of $f(x, y)$, it's clear that as $(x, y)$ moves within $B_1(0, 0)$,
the denominator $(x-2)^2 + y^2$ stays away from $0$, ensuring that $f$
does not approach infinity and remains bounded.

Since $f$ is continuous on a compact set, and because the function does not approach
its undefined point within $B_1(0, 0)$, it is uniformly continuous
on $B_1(0, 0)$ by the Heine-Cantor theorem, which states that a continuous
function on a compact set is uniformly continuous.

Uniform continuity on $B_2(0, 0)$:

$B_2(0, 0)$ is the open ball of radius 2 centered at $(0, 0)$. The point $(2, 0)$
is on the boundary of this ball, and as $(x, y)$ approaches $(2, 0)$
from within $B_2(0, 0)$, the denominator $(x-2)^2 + y^2$ approaches $0$,
causing $f(x, y)$ to approach infinity.

The closer $(x, y)$ gets to $(2, 0)$ within $B_2(0, 0)$,
the larger $f(x, y)$ becomes, which suggests that the changes in $f(x, y)$
can be arbitrarily large for small changes in $(x, y)$ near $(2, 0)$.
This behavior violates the criteria for uniform continuity since no single
$\delta$ can work $\forall$ $\epsilon > 0$ near the point $(2, 0)$.

Conclusion:

$f(x, y)$ is uniformly continuous on $B_1(0, 0)$, but not on $B_2(0, 0)$.

\sskip


\begin{xca}
Show that the function $f:\R\to\R$ defined by $f(x)=x^3$ is not uniformly continuous.
\end{xca}

To prove that $f(x) = x^3$ is not uniformly continuous, we can choose
specific values of $x$ and $y$ that depend on $\delta$ to show that the condition for
uniform continuity is violated.

Let's choose $x = a$ and $y = a + \delta/2$ for some $a \in \R$ and $\delta > 0$.
Clearly, $|x - y| = |a - (a + \delta/2)| = \delta/2 < \delta$, so the condition
$|x - y| < \delta$ is satisfied.

Now, we compute $|f(x) - f(y)|$:
\[|f(x) - f(y)| = |x^3 - y^3| = |a^3 - (a + \delta/2)^3|\]
Expanding $(a + \delta/2)^3$, we get:
\[\left|a^3 - \left(a^3 + \frac{3}{2} a^2 \delta + \frac{3}{4} a \delta^2 + \frac{1}{8} \delta^3\right)\right|\]
\[= \left|\frac{3}{2} a^2 \delta + \frac{3}{4} a \delta^2 + \frac{1}{8} \delta^3\right|\]

For any given $\epsilon > 0$, we can choose $a$ large enough such that the absolute
difference $\left|\frac{3}{2} a^2 \delta + \frac{3}{4} a \delta^2 + \frac{1}{8} \delta^3\right|$
exceeds $\epsilon$, regardless of how small $\delta$ is chosen. This is because
as $a$ becomes very large, the term $\frac{3}{2} a^2 \delta$ dominates
and its magnitude can be made to exceed any fixed $\epsilon$.

Therefore, no matter how small $\delta$ is chosen, we can always find
points $x$ and $y$ such that $|x - y| < \delta$, but $|f(x) - f(y)|$
can be made larger than any given $\epsilon$ by choosing $a$ sufficiently large.
This shows that $f(x) = x^3$ is not uniformly continuous on $\R$.

\end{document}
