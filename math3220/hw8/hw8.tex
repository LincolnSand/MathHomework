
%\documentclass[draft]{amsart}
%\input seteps
\documentclass[12]{amsart}
\usepackage{amssymb}
\usepackage{eucal}
\usepackage{amscd}
\usepackage{graphicx}
\usepackage{mathtools}
\usepackage{xcolor}
\usepackage{bm}



\newtheorem{thm}{Theorem}
\newtheorem*{extracredit}{Extra Credit Problem}
\newtheorem{lem}[thm]{Lemma}
\newtheorem{cor}[thm]{Corollary}
\newtheorem{pro}[thm]{Proposition}
\theoremstyle{definition}
\newtheorem{rem}[thm]{Remark}

\newtheorem{exm}[thm]{Example}
\newtheorem{xca}{Problem}
\newcommand{\probl}{Problem}
\newtheorem*{exer}{\probl}
\newenvironment{exercise}[1]{\renewcommand{\probl}{#1}\begin{exer}}
 {\end{exer}}
 \newcommand{\be}{\begin{equation*}}
 \newcommand{\ee}{\end{equation*}}
 \newcommand{\R}{\mathbb{R}}
\newcommand{\Z}{\mathbb{Z}}
\newcommand{\N}{\mathbb{N}}
\newcommand{\Q}{\mathbb{Q}}
%\newcommand{\e}{\text{e}}
\newcommand{\C}{\mathbb{C}}
%%%
\newcommand{\cl}{\overline}
\newcommand{\sskip}{\newpage%%
}
\newcommand{\norm}[1]{\lVert#1\rVert}
\newcommand{\m}[1]{$ #1 $}
\newcommand{\lskip}{\bigskip}
\newcommand{\der}[2]{\frac{\partial#1}{\partial #2}}


\DeclareMathOperator{\im}{im} \DeclareMathOperator{\rank}{rank}
\DeclareMathOperator{\Mor}{Mor} \DeclareMathOperator{\coker}{coker}
\DeclareMathOperator{\supp}{supp} \DeclareMathOperator{\For}{For}
\DeclareMathOperator{\Hom}{Hom} \DeclareMathOperator{\End}{End}
\DeclareMathOperator{\Int}{Int} \DeclareMathOperator{\order}{order}
\DeclareMathOperator{\ad}{ad} \DeclareMathOperator{\Ad}{Ad}
\DeclareMathOperator{\GL}{GL} \DeclareMathOperator{\SL}{SL}
\DeclareMathOperator{\SO}{SO} \DeclareMathOperator{\Sp}{Sp}
\DeclareMathOperator{\SU}{SU} \DeclareMathOperator{\tr}{tr}
\DeclareMathOperator{\Aut}{Aut} \DeclareMathOperator{\re}{Re}
\DeclareMathOperator{\imag}{Im} \DeclareMathOperator{\Card}{Card}
\newcommand{\Sd}{\ensuremath{\surd}}


\begin{document}

\centerline{ \bf Math 3220-1: Homework 8, due 03/29/2024}
\bigskip
\centerline{ \bf Show all work. Homework has to be uploaded to GradeScope.}
\bigskip
\noindent Name (PRINT):\hskip 2.5in ID:
\smallskip

\hrule

\bigskip


\begin{xca}
Let
$$
f(x,y)=
\begin{cases}
0 &\text{ if $xy\neq 0 $} \\
1 & \text{ if $xy=0.$}
\end{cases}
$$
Show that the partial derivatives $\der{f}{x}$ and $\der{f}{y}$ exist at the point $(0,0)$ but $f$ is not continuous at $(0,0)$. Show also that if $\bf{u}$ is a unit vector different from $\bf{i}$ or $\bf{j}$ then the directional derivative $D_{\bf{u}}f(0,0)$ does not exist.

\end{xca}


Partial derivative with respect to x:

\[\frac{\partial f}{\partial x}(0, 0) = \lim_{h \to 0} \frac{f(0 + h, 0) - f(0, 0)}{h}\]

Since $f(0, 0) = 1$ and $f(h, 0) = 1$ for any $h \neq 0$, the difference
in the numerator is always $0$ for $h \neq 0$, making this limit $0$.


Partial derivative with respect to y:

\[\frac{\partial f}{\partial y}(0, 0) = \lim_{h \to 0} \frac{f(0, 0 + h) - f(0, 0)}{h}\]

which is $0$ by the same reasoning as for the first partial derivative.

So, $\frac{\partial f}{\partial x}(0, 0) = \frac{\partial f}{\partial y}(0, 0) = 0$.


Continuity at $(0, 0)$:

Consider the path $y = x$. Along this path:
\[\lim_{x \to 0} f(x, x) = \lim_{x \to 0} 0 = 0\]
which is different from $f(0, 0) = 1$. Hence, $f$ is not continuous at $(0, 0)$.


The directional derivative of $f$ at $(0, 0)$ in the direction of the unit vector
$u = (u_1, u_2)$ is defined as:
\[D_{u} f(0, 0) = \lim_{t \to 0} \frac{f(t u_1, t u_2) - f(0, 0)}{t}\]

If $u$ is neither $i$ nor $j$, then both $u_1$ and $u_2$ are non-zero. In this
case, for $t \neq 0$, $f(t u_1, t u_2) = 0$. However, as $t$ approaches $0$, this
directional derivative does not approach a single value because it oscillates between
$0$ and $1$ depending on the path taken. Therefore, the directional derivative
does not exist for any unit vector $u$ that is not along the x or y-axis.


\sskip

\begin{xca}
Show that if $f:\R\to\R$ is a smooth function and $z=f(x^2y)$, then $x\der{z}{x}=2y\der{z}{y}.$
  \end{xca}


Compute $\frac{\partial z}{\partial x}$:

Applying the chain rule:
\[\frac{\partial z}{\partial x} = \frac{d f(u)}{du} \cdot \frac{\partial (x^2 y)}{\partial x}\]

Here, $u = x^2 y$, so:
\[\frac{\partial (x^2 y)}{\partial x} = 2xy\]

Therefore,
\[\frac{\partial z}{\partial x} = f'(x^2 y) \cdot 2xy\]


Compute $\frac{\partial z}{\partial y}$:

Using the chain rule gives us the final value of:
\[\frac{\partial z}{\partial y} = f'(x^2 y) \cdot x^2\]


Show $x \frac{\partial z}{\partial x} = 2y \frac{\partial z}{\partial y}$:
\[x \cdot \frac{\partial z}{\partial x} = 2x^2 y f'(x^2 y)\]
\[2y \cdot \frac{\partial z}{\partial y} = 2x^2 y f'(x^2 y)\]


\sskip
\begin{xca} %2

\bigskip

If $F(x,y)=(f_1(x,y), f_2(x,y))$ is a differentiable function from $\R^2$ to $\R^2$ and if we define
$G:\R^2\to\R^2$ by $G(s,t)=F(st,s+t)$, find an expression for the differential matrix of $G$ in terms of the partial derivatives
of $f_1$ and $f_2$.


\end{xca}


Partial derivatives of $G_1$:
\[\frac{\partial G_1}{\partial s} = \frac{\partial f_1}{\partial x} \cdot t + frac{\partial f_1}{\partial y}\]
\[\frac{\partial G_1}{\partial t} = \frac{\partial f_1}{\partial x} \cdot s\]

Partial derivatives of $G_2$:
\[\frac{\partial G_1}{\partial s} = \frac{\partial f_2}{\partial x} \cdot t + frac{\partial f_2}{\partial y}\]
\[\frac{\partial G_1}{\partial t} = \frac{\partial f_2}{\partial x} \cdot s\]

So,
\[J_G(s, t) = \begin{bmatrix}
    \frac{\partial f_1}{\partial x} \cdot t + \frac{\partial f_1}{\partial y} & \frac{\partial f_1}{\partial x} \cdot s \\
    \frac{\partial f_2}{\partial x} \cdot t + \frac{\partial f_2}{\partial y} & \frac{\partial f_2}{\partial x} \cdot s
\end{bmatrix}\]


\sskip


\begin{xca} %3

\bigskip

Suppose $U$ and $V$ are open subsets of $\R^p$ and $F:U\to V$ has an inverse function $G:V\to U$. This means $F\circ G(y)=y$ for all $y\in V$
and $G\circ F(x)=x$ for all $x\in U$. Show that if $F$ is differentiable on $U$ and $G$ is differentiable on $V$, then the differential matrix $DF(x)$ is non-singular at each $x\in U$, and for each $x\in U$,
$$
DF(x)^{-1}=DG(y)
$$
where $y=F(x)$.

\end{xca}


Given that $F$ and $G$ are inverses, we have:
\[F(G(y)) = y \forall y \in V\]
\[G(F(x)) = x \forall x \in U\]

Differentiating both sides of $G(F(x)) = x$ with respect to $x$ gives:
\[D(G \circ F)(x) = D G(F(x)) \cdot D F(x) = I\]

Similarly, differentiating $F(G(y)) = y$ with respect to $y$ gives:
\[D(F \circ G)(y) = D F(G(y)) \circ D G(y) = I\]


Now, for the implications:

$D G(F(x)) \cdot D F(x) = I$ implies that $D F(x)$ has a right inverse,
which is $D G(F(x))$.

$D F(G(y)) \circ D G(y) = I$ implies that $D F(x)$ has a left inverse,
which is $D G(y)$, where $y = F(x)$.


Thus, we have:
\[D F(x) \cdot D G(F(x)) = D G(F(X)) \cdot D F(x) = I\]


Therefore, $D F(x)$ is non-singular for each $x \in U$, and its inverse
is given by $D F(x)^{-1} =  D G(y)$ where $y = F(x)$.


\sskip



\begin{xca} %5

\bigskip

For the function $f(x,y)=x^2+y^3+xy$, find the gradient at the point $(1,1)$, the direction of greatest ascent of $f$ at this point, and a direction in which the rate of increase of this function is $0$ (the answer to the last two questions should be unit vectors).


\end{xca}


\[\frac{\partial f}{\partial x} = 2x + y\]
\[\frac{\partial f}{\partial y} = 3y^2 + x\]

Evaluating at $(1,1)$:
\[\frac{\partial f}{\partial x}(1, 1) = 2(1) + 1 = 3\]
\[\frac{\partial f}{\partial y}(1, 1) = 3(1)^2 + 1 = 4\]

So, the gradient of $f$ at $(1, 1)$ is $\nabla f(1, 1) = (3, 4)$.


Direction:

To get the direction, we just take the gradient at the point and normalize the vector.

The magnitude of $(3, 4)$ is $\sqrt{3^2 + 4^2} = \sqrt{9 + 16} = \sqrt{25} = 5$.

So, the unit vector for the greatest ascent is $\left(\frac{3}{5}, \frac{4}{5}\right)$.


Direction with zero rate of increase:

The direction with zero rate of increase is perpendicular to the gradient.
The perpendicular vector is $(-4, 3)$. Now, we normalize it to get
$\left(-\frac{4}{5}, \frac{3}{5}\right)$.


\sskip

\begin{xca} %6

\bigskip

Find the tangent space at $(2,4,1)$ for the parameterized surface in $\R^3$ parameterized by the function $G:U\to\R^3$, where
$$
U=\{(u,v)\in\R^2: u>0, v>0\}\text{ and } G(u,v)=(uv,u^2,v^2).
$$
\end{xca}


To find the tangent space on a parameterized surface, we need to calculate
the partial derivatives of the parameterization function at that point.
These partial derivatives will give us the basis vectors for the tangent space.

We first need to solve the following system:
\[uv = 2\]
\[u^2 = 4\]
\[v^2 = 1\]

This gives us $u = 2$ ($u > 0$) and $v = 1$ ($v > 0$).

Partial derivatives:
\[G_u(u, v) = (v, 2u, 0)\]
\[G_v(u, v) = (u, 0, 2v)\]

Let's now compute the values at $(2, 1)$:
\[G_u(2, 1) = (1, 4, 0)\]
\[G_v(2, 1) = (2, 0, 2)\]

These two vectors, $(1, 4, 0)$ and $(2, 0, 2)$ form a basis for the
tangent space at the point $(2, 4, 1)$ at the surface.


\end{document}
