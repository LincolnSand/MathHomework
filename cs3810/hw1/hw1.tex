\documentclass{article}

\usepackage{amsfonts}
\usepackage{graphicx}
\usepackage{amssymb}
\usepackage{amsmath}
\usepackage{listings}


\DeclareMathOperator{\sech}{sech}
\newcommand{\NN}{\mathbb{N}}
\newcommand{\RR}{\mathbb{R}}
\newcommand{\QQ}{\mathbb{Q}}
\newcommand{\ZZ}{\mathbb{Z}}
\newcommand{\dV}{\;\mathrm{d}V}
\newcommand{\dA}{\;\mathrm{d}A}
\newcommand{\dx}{\;\mathrm{d}x}
\newcommand{\dy}{\;\mathrm{d}y}
\newcommand{\dz}{\;\mathrm{d}z}
\newcommand{\cA}{\mathcal{A}}
\newcommand{\Bb}{\mathcal{B}}
\newcommand{\Ww}{\mathcal{W}}
\newcommand{\Dd}{\mathcal{D}}
\newcommand{\Ss}{\mathcal{S}}
\newcommand{\Ee}{\mathcal{E}}
\DeclareMathOperator{\im}{im}


\setlength\parindent{18pt}

\begin{document}

1. System A takes 500 seconds to run a program. The same program
takes 450 seconds to run on a new system B. What is the
speedup provided by B over A? What is the performance improvement of B over A?

Speedup: The formula for speedup is
\[\frac{\text{Execution Time of A}}{\text{Execution Time of B}}.\]

Performance Improvement: The formula for performance improvement is
\[\left(\frac{\text{Execution Time of A $-$ Execution Time of B}}{\text{Execution Time of A}}\right) \times 100\%.\]

We are given that Execution Time of A = 500 seconds and
Execution Time of B = 450 seconds.

Plugging this in gives $1. \overline{11}$ times faster for speedup and gives $10\%$
for performance improvement.


2. System A has two processors. Program X takes 230 seconds to execute on one
of the processors. Program Y takes 230 seconds to execute in parallel on the
other processor. System B has a single processor that can execute only one program at a time.
Program X takes 100 seconds to execute on this processor. Program Y takes 100 seconds to execute
on this processor. Which system would you pick if you cared about overall system throughput
for a workload that is a mix of programs X and Y?

The total time taken by System A to complete both Program X and Program Y
is 230 seconds since they can be done in parallel.

The total time taken by System B to complete both Program X and Program Y
is 200 seconds since it has to complete each program sequentially (100 seconds each).

Therefore, System B would be the better option for throughput for a workload mix of
Program X and Program Y because it completes them 30 seconds faster than System A.


3. A program executes 600 billion instructions. It executes on an AMD processor that has
an average CPI of 0.8 and a clock frequency of 3.2 GHz. How many seconds does the program
take to execute? What is the cycle time of this AMD processor? Assume that an IBM processor takes
100 seconds to execute the program. What is the speedup provided by the IBM processor, relative to the AMD processor?

Execution Time for the ADM processor:

The formula is given by: \[\frac{\text{Number of Instructions $\times$ CPI}}{\text{Clock Frequency}}\]

Plugging in the given values of Number of Instructions = 600 billion, CPI = 0.8,
Clock frequency = 3.2 GHz
gives us 150 seconds to execute using the ADM processor.

Cycle time for the ADM processor:

The formula is given by: \[\frac{1}{\text{Clock Frequency}}\]

Plugging in the Clock Frequency = 3.2 GHz gives us a cycle time of 3.125 nanoseconds.

Speedup by the IBM processor:

The formula is given by: \[\frac{\text{Execution Time on AMD}}{\text{Execution time on IBM}}\]

Plugging in Execution Time on AMD = 150 seconds and Execution time on IBM = 100 seconds
gives us a speedup of 1.5 times faster.


4. Colin designs a 4 GHz processor where two important programs, A and B, take one second each to execute.
Program A has a CPI of 1 and program B has a CPI of 1.25. Nisha is tasked with designing the company's next-generation
processor. She comes up with an idea that improves the CPI of A to 0.6 and the CPI of B to 1.0. But the idea is so
complex that the processor can only be implemented with a cycle time of 0.33 ns. Does Nisha's new processor out-perform
Colin's processor on program A? Does Nisha's new processor out-perform Colin's processor on program B?

The formula for Execution Time is:
\[\frac{\text{Number of Instructions $\times$ CPI}}{\text{Clock Frequency}}\]

Colin's processor: $4 \text{GHz} = 4 \times 10^9 \text{Hz}$.
Nisha's processor: $\frac{1}{\text{Cycle Time}} = \frac{1}{0.33 \text{ns}} = \frac{1}{0.33 \times 10^{-9}} \text{Hz}$

Plugging in the values given, we get that:

Program A takes approximately $0.792$ seconds to execute on Nisha's processor.

Program B takes approximately $1.056$ seconds to execute on Nisha's processor.

Comparing these results with Colin's processor:

For Program A, Nisha's processor performs better than Colins's (0.792 seconds vs 1 second).

For Program B, Nisha's processor performs worse than Colin's (1.056 seconds vs 1 second).


\end{document}
