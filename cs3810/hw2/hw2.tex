\documentclass{article}

\usepackage{amsfonts}
\usepackage{graphicx}
\usepackage{amssymb}
\usepackage{amsmath}
\usepackage{listings}

\usepackage{xcolor}
\usepackage{listings}

\usepackage{xparse}

\NewDocumentCommand{\codeword}{v}{%
\texttt{\textcolor{blue}{#1}}%
}

\lstset{language=C,keywordstyle={\bfseries \color{blue}}}


\DeclareMathOperator{\sech}{sech}
\newcommand{\NN}{\mathbb{N}}
\newcommand{\RR}{\mathbb{R}}
\newcommand{\QQ}{\mathbb{Q}}
\newcommand{\ZZ}{\mathbb{Z}}
\newcommand{\dV}{\;\mathrm{d}V}
\newcommand{\dA}{\;\mathrm{d}A}
\newcommand{\dx}{\;\mathrm{d}x}
\newcommand{\dy}{\;\mathrm{d}y}
\newcommand{\dz}{\;\mathrm{d}z}
\newcommand{\cA}{\mathcal{A}}
\newcommand{\Bb}{\mathcal{B}}
\newcommand{\Ww}{\mathcal{W}}
\newcommand{\Dd}{\mathcal{D}}
\newcommand{\Ss}{\mathcal{S}}
\newcommand{\Ee}{\mathcal{E}}
\DeclareMathOperator{\im}{im}

\newcommand{\lskip}{\newpage}


\setlength\parindent{18pt}

\begin{document}

\textbf{1.}

\textbf{a.}

Source registers: \codeword{$t2}, \codeword{$zero}

Destination register: \codeword{$t3}

Note: \codeword{$zero} is a special register that is always zero.

\textbf{b.}

Source registers: \codeword{$t1}, \codeword{$t4}

Destination register: \codeword{$t3}

\textbf{c.}

Source register: \codeword{$t2}

Destination register: \codeword{$t1}

Note: \codeword{100} is an immediate value and not a register.

\textbf{d.}

Source register: \codeword{$gp}

Destination register: \codeword{$s1}

Note: The memory address from where the load happens is \codeword{$gp+4}.

\textbf{e.}

Source registers: \codeword{$s1}, \codeword{$gp}

Destination register: None

Note: the operation is written to memory (the address is \codeword{$gp+12}).

\textbf{f.}

Source registers: \codeword{$t1}, \codeword{$s2}

Destination register: None

Note: This is a branch instruction.

\lskip

\textbf{2.}

\textbf{a.}

\codeword{x} is at the base address, so it is at memory address 2000.

\codeword{y} is the next integer, so 2000 + 4 = 2004.

\codeword{z} is the next integer, so 2004 + 4 = 2008.

\codeword{w[0]} is the next integer, so 2008 + 4 = 2012.

\codeword{w[1]} is the next integer, so 2012 + 4 = 2016.

\textbf{b.}

\begin{lstlisting}

lw $s1, 0($gp)     # $s1 = x (value at address 2000, which is 10)
lw $s2, 4($gp)     # $s2 = y (value at address 2004, which is 10)
add $s3, $s2, $s2  # $s3 = y + y = 10 + 10 = 20
sub $s1, $s1, $s2  # $s1 = x - y = 10 - 10 = 0
add $s2, $s1, $s3  # $s2 = $s1 + $s3 = 0 + 20 = 20
sw $s1, 8($gp)     # z = $s1 = 0 (value 0 stored at address 2008)
sw $s2, 12($gp)    # w[0] = $s2 = 20 (value 20 stored at address 2012)
subi $s2, $s3, 120 # $s2 = $s3 - 120 = 20 - 120 = -100
sw $s2, 16($gp)    # w[1] = $s2 = -100 (value -100 stored at address 2016)

\end{lstlisting}


At the end of the program:

\codeword{x}: 10

\codeword{y}: 10

\codeword{z}: 0

\codeword{w[0]}: 20

\codeword{w[1]}: -100

\lskip

\textbf{3.}

Binary:

$\frac{194}{2} = 97$, remainder = 0.

$\frac{97}{2} = 48$, remainder = 1.

$\frac{48}{2} = 24$, remainder = 0.

$\frac{24}{2} = 12$, remainder = 0.

$\frac{12}{2} = 6$, remainder = 0.

$\frac{6}{2} = 3$, remainder = 0.

$\frac{3}{2} = 1$, remainder = 1.

$\frac{1}{2} = 0$, remainder = 1.

So, 194 = 11000010.

Hexadecimal:

$\frac{194}{16} = 12$, remainder = 2.

$\frac{12}{16} = 0$, remainder = 12.

So, 194 = C2.

\lskip

\textbf{4.}

Decimal:

$0 \cdot 2^0 + 0 \cdot 2^1 + 1 \cdot 2^2 + 1 \cdot 2^3 + 1 \cdot 2^4 + 1 \cdot 2^5 + 0 \cdot 2^6 + 1 \cdot 2^7$

= 0 + 0 + 4 + 8 + 16 + 32 + 0 + 128 = 188.

So, 10111100 = 188.

Hexadecimal:

Group it into two groups: 1011, 1100

Convert each group: 11, 12 = B, C.

So, 10111100 = BC.

\lskip

\textbf{5.}

Decimal:

First note that $D = 13$.

So, $13 \cdot 16^0 + 0 \cdot 16^1$ = 13 + 144 = 157.

So, 9D = 157.

Binary:

9 in hex is 1001 in binary and D in hex is 1101 in binary.

Thus, 9D = 10011101.

\lskip

\textbf{6.}

\begin{lstlisting}

xor $t0, $t0, $t0

xor $t1, $t1, $t1
add $t1, $gp, $8

loop:
    bge $t0, 10, end

    sll $t2, $t0, 5

    sll $t3, $t0, 2
    add $t3, $t3, $t1

    sw $t2, 0($t3)

    addi $t0, $t0, 1

    j loop

end:
    sw $t0, 4($gp)

\end{lstlisting}

\end{document}
