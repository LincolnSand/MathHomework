\documentclass{article}

\usepackage{amsfonts}
\usepackage{graphicx}
\usepackage{amssymb}
\usepackage{amsmath}
\usepackage{listings}
\usepackage{xcolor}
\usepackage{xparse}
\usepackage{booktabs}

\NewDocumentCommand{\codeword}{v}{%
\texttt{\textcolor{blue}{#1}}%
}

\lstset{language=C,keywordstyle={\bfseries \color{blue}}}


\DeclareMathOperator{\sech}{sech}
\newcommand{\NN}{\mathbb{N}}
\newcommand{\RR}{\mathbb{R}}
\newcommand{\QQ}{\mathbb{Q}}
\newcommand{\ZZ}{\mathbb{Z}}
\newcommand{\dV}{\;\mathrm{d}V}
\newcommand{\dA}{\;\mathrm{d}A}
\newcommand{\dx}{\;\mathrm{d}x}
\newcommand{\dy}{\;\mathrm{d}y}
\newcommand{\dz}{\;\mathrm{d}z}
\newcommand{\cA}{\mathcal{A}}
\newcommand{\Bb}{\mathcal{B}}
\newcommand{\Ww}{\mathcal{W}}
\newcommand{\Dd}{\mathcal{D}}
\newcommand{\Ss}{\mathcal{S}}
\newcommand{\Ee}{\mathcal{E}}
\DeclareMathOperator{\im}{im}

\newcommand{\lskip}{\newpage}


\setlength\parindent{18pt}

\begin{document}

\Large{Lincoln Sand}


\textbf{1.} Consider a sequential circuit that implements the following function in a self-driving car.
In every second, the car's position within the lane is determined with cameras and fed as input to the
sequential circuit. Accordingly, the car's steering wheel is moved to one of 2 positions: Left, Right.
The car's position can be "Middle of a lane", "Drifting to the right edge of the lane", and "Drifting
to the left edge of the lane". If the car is in the "Middle", the steering position is left unchanged.
If the car is drifting to one of the edges of the lane, the steering is moved away from the edge, i.e.,
steering should be moved right if the car is near the left edge of the lane. (40 points)

    1. What are the states for the circuit controlling the steering?
    2. What are the inputs to the circuit and what values can those inputs have?
    3. Draw the finite state table and finite state diagram for this circuit.

1.

State $S_0$: Steering in the middle position (no change needed).

State $S_1$: Steering is moved to the left (to correct from drifting to the right).

State $S_2$: Steering is moved to the right (to correct from drifting to the left).

2.

$X = 0$: The car is in the middle of the lane.

$X = 1$: the car is drifting to the right edge of the lane.

$X = 2$: The car is drifting to the left edge of the lane.

3. 

\begin{table}[htbp]
\centering
\begin{tabular}{@{}llll@{}}
\toprule
Current State & Input (X) & Next State & Output (Y) \\ \midrule
$S_0$ (Middle)   & 0          & $S_0$         & 1 (No change) \\
$S_0$ (Middle)   & 1          & $S_1$         & 0 (Steer left) \\
$S_0$ (Middle)   & 2          & $S_2$         & 2 (Steer right) \\
$S_1$ (Left)     & 0          & $S_0$         & 1 (No change) \\
$S_1$ (Left)     & 1          & $S_1$         & 0 (Steer left) \\
$S_1$ (Left)     & 2          & $S_2$         & 2 (Steer right) \\
$S_2$ (Right)    & 0          & $S_0$         & 1 (No change) \\
$S_2$ (Right)    & 1          & $S_1$         & 0 (Steer left) \\
$S_2$ (Right)    & 2          & $S_2$         & 2 (Steer right) \\ \bottomrule
\end{tabular}
\end{table}

This table produces this FSM behavior:

State $S_0$ (Middle):

Stays in $S_0$ on input $X = 0$.

Moves to $S_1$ on input $X = 1$.

Moves to $S_2$ on input $X = 2$.

State $S_1$ (Left):

Moves to $S_0$ on input $X = 0$.

Stays in $S_1$ on input $X = 1$.

Moves to $S_2$ on input $X = 2$.

State $S_2$ (Right):

Moves to $S_0$ on input $X = 0$.

Moves to $S_1$ on input $X = 1$.

Stays in $S_2$ on input $X = 2$.


\newpage


\textbf{2.} Consider the following circuit to control a moving walkway. Every 2 seconds,
the circuit examines two weight sensors and decides if the walkway will travel Northward
or Southward or be turned off. There is a weight sensor in the North half of the walkway
and a weight sensor in the South half of the walkway. If both sensors detect zero weight,
the walkway is turned off. If the walkway is off and both sensors detect weight, the walkway
remains off. If the walkway is off and one of the sensors detects a weight, the walkway
starts moving in the opposite direction, i.e., if the North weight sensor detects a weight,
the walkway starts moving Southward. Once the walkway is in motion, it retains that motion
until both weight sensors detect zero weight. (60 points)

    1. What are the states for the circuit controlling the walkway?
    2. What are the inputs to the circuit and what values can those inputs have?
    3. Draw the finite state table and finite state diagram for this circuit.

1.

State $S_0$: The walkway is turned off.

State $S_1$: The walkway is moving Northward.

State $S_2$: The walkway is moving Southward.

2.

$X = 00$: No weight on either sensor ($\text{North sensor} = 0$, $\text{South sensor} = 0$).

$X = 01$: Weight detected only on South sensor ($\text{North sensor} = 0$, $\text{South sensor} = 1$).

$X = 10$: Weight detected only on North sensor ($\text{North sensor} = 1$, $\text{South sensor} = 0$).

$X = 11$: Weight detected on both sensors ($\text{North sensor} = 1$, $\text{South sensor} = 1$).

3.

\begin{table}[htbp]
\centering
\begin{tabular}{@{}llll@{}}
\toprule
Current State & Input (X) & Next State & Output (Y) \\ \midrule
$S_0$ (Off)      & 00          & $S_0$         & Off \\
$S_0$ (Off)      & 01          & $S_1$         & Move Northward \\
$S_0$ (off)      & 10          & $S_2$         & Move Southward \\
$S_0$ (Off)      & 11          & $S_0$         & Off \\
$S_1$ (North)    & 00          & $S_0$         & Off \\
$S_1$ (North)    & 01          & $S_1$         & Move Northward \\
$S_1$ (North)    & 10          & $S_1$         & Move Northward \\
$S_1$ (North)    & 11          & $S_1$         & Move Northward \\
$S_2$ (South)    & 00          & $S_0$         & Off \\
$S_2$ (South)    & 01          & $S_2$         & Move Southward \\
$S_2$ (South)    & 10          & $S_2$         & Move Southward \\
$S_2$ (South)    & 11          & $S_2$         & Move Southward \\ \bottomrule
\end{tabular}
\end{table}

This table produces this FSM behavior:

State $S_0$ (Off):

Stays in $S_0$ on input $00$ and $11$.

Moves to $S_1$ on input $01$.

Moves to $S_2$ on input $10$.

State $S_1$ (North):

Moves to $S_0$ on input $00$.

Stays in $S_1$ on input $01$, $10$, and $11$.

State $S_2$ (South):

Moves to $S_0$ on input $00$.

Stays in $S_2$ on input $01$, $10$, and $11$.

\end{document}
