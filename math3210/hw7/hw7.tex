\documentclass[answers]{exam}
\title{Foundations of Analysis, HW 3}
\author{James Cameron}
\date{}
\usepackage{amsmath}
\usepackage{amssymb}
\usepackage{graphicx}
\usepackage{amsthm}

\newtheorem{thm}{Theorem}
\newtheorem{hthm}[thm]{*Theorem}
\newtheorem{lemma}[thm]{Lemma}
\newtheorem{cor}[thm]{Corollary}
\newtheorem{obs}[thm]{Observation}
\newtheorem{prop}[thm]{Proposition}
\newtheorem{con}[thm]{Conjecture}
\newtheorem{exer}[thm]{Exercise}

\newtheorem{scho}[thm]{Scholium}
\newtheorem*{Thm}{Theorem}
\newtheorem*{Con}{Conjecture}
\newtheorem*{Axiom}{Axiom}

\theoremstyle{remark}
\newtheorem{remark}[thm]{Remark}
\newtheorem{notation}[thm]{Notation}


\theoremstyle{definition}
\newtheorem{Def}[thm]{Definition}
\newtheorem{example}[thm]{Example}
\newtheorem{ques}[thm]{Question}
\everymath{\displaystyle}

\pagestyle{head}
\header{\bfseries\large Foundations of Analysis}{\bfseries\large Homework 7 \ifprintanswers (solutions) \fi}{\bfseries\large Fall, 2023}
\headrule

\DeclareMathOperator{\sech}{sech}
\newcommand{\NN}{\mathbb{N}}
\newcommand{\RR}{\mathbb{R}}
\newcommand{\QQ}{\mathbb{Q}}
\newcommand{\ZZ}{\mathbb{Z}}
\newcommand{\dV}{\;\mathrm{d}V}
\newcommand{\dA}{\;\mathrm{d}A}
\newcommand{\dx}{\;\mathrm{d}x}
\newcommand{\dy}{\;\mathrm{d}y}
\newcommand{\dz}{\;\mathrm{d}z}
\pointname{}
\newcommand{\cA}{\mathcal{A}}
\newcommand{\Bb}{\mathcal{B}}
\newcommand{\Ww}{\mathcal{W}}
\newcommand{\Dd}{\mathcal{D}}
\newcommand{\Ss}{\mathcal{S}}
\newcommand{\Ee}{\mathcal{E}}
\DeclareMathOperator{\im}{im}
\begin{document}
% \maketitle


% \noindent
 This is due Saturday 10/28 by 11:59 pm on Gradescope. Please either neatly write up your solutions or type them up. You can find a .tex template on Canvas. Your proofs should be written in complete sentences and paragraphs, using a combination of words and symbols. They should be \textbf{correct, clear, and concise}. You will be graded on all three, especially the first two!

\noindent


\begin{questions}

\question Suppose that $f:[a,b] \to \RR$ is continuous, and suppose that $0\not= f(x)$ for any $x \in [a,b]$. Show that there is some $m >0$ such that one of the following is true: either $f(x)>m$ for all $x \in [a,b]$, or $f(x) < -m$ for all $x \in [a,b]$.
\begin{solution}

Because $f : [a, b] \to \RR$ is continuous on the entire domain and doesn't equal 0,
by the intermediate value theorem, it must always be positive,
or it must always be negative.

If f(x) is always positive, then the minimum value
of the function f(x) on the domain [a, b] is positive.
So we can just set m to be half the minimum.

On the other hand, if f(x) is always negative, then the
maximum value of the function f(x) on the domain [a, b] is negative.
So we can just set m to be half the absolute value of the maximum.

Therefore, we have found an m>0 such that
f(x) is either always greater than m or always
less than -m.

\end{solution}

\question Show that if $f: \RR \to \RR$ is a polynomial of odd degree, i.e. $f(x)= \sum_{i=0}^n a_i x^i$, with $n$ odd and $a_n \not=0$, then there is some $x \in \RR$ with $f(x)=0$.
\begin{solution}

We will first analyze the behavior of f as we approach $\infty$
and $-\infty$.

As $x \to \infty$, the $a_n x^n$ term dominates the sum
and since n is odd, the sign is the same as that of
$a_n$. In other words, as $x \to \infty$,
$f(x) \to \infty$ if $a_n > 0$ and $f(x) \to -\infty$
if $a_n < 0$.

Similarly, as $x \to -\infty$, the $a_n x^n$ term dominates the sum
and since n is odd, the sign is the same as that of
$a_n$. In other words, as $x \to -\infty$,
$f(x) \to -\infty$ if $a_n > 0$ and $f(x) \to \infty$
if $a_n < 0$.

Now, since they are on different signs and we know all polynomials
are continuous, then by the intermediate value theorem,
f(x) = 0 somewhere between $-\infty$ and $\infty$.

\end{solution}

\question Show that $f:(0,1) \to \RR$ defined by $f(x)=\sin(1/x)$ is not uniformly continuous.
\begin{solution}

Our goal is to show that $\exists \epsilon > 0$
such that $\forall \delta > 0$,
$\exists x, y \in (0, 1)$
with $|x-y| < \delta$ and $|f(x) - f(y)| \geq \epsilon$.

Let $\epsilon = 1$. We aim to show that for any
$\delta > 0$, $\exists x, y \in (0, 1)$
such that $|x-y| < \delta$, but $|sin(1/x) - sin(1/y)| \geq 1$.

Let $\delta > 0$ be given. Choose an $n \in \NN$ such that
$\frac{1}{2 \pi n} < \delta$. Now consider the points:

\[x = \frac{1}{2 \pi n}\]
\[y = \frac{1}{2 \pi n + \pi}\]

Then we have $|x - y| = \frac{\pi}{2 \pi n (2 \pi n + \pi)} < \frac{1}{2 \pi n} < \delta$.

$|sin(1/x) - sin(1/y)| = |0 - 1| = 1 \geq \epsilon$

Therefore, the function is not uniformly continuous on (0, 1).

\end{solution}

\question Show that $f:(0,1) \to \RR$ defined by $f(x)=x \sin(1/x)$ is uniformly continuous.
\begin{solution}

Let $\epsilon > 0$ be given. We want to find $\delta > 0$
such that $\forall x, y \in (0, 1)$, if $|x-y| < \delta$,
then $|x sin(1/x) - y sin(1/y)| < \epsilon$.

First, Let's observe that for any $x, y \in (0, 1)$, we have the following:
\[|f(x) - f(y)| = |x sin(1/x) - y sin(1/y)| \leq |x sin(1/x) - x sin(1/y)| + |x sin(1/y) - y sin(1/y)|\]

For the first term, use the fact that the sin function is bounded.
$|sin(x)| \leq 1$, so:
\[|x sin(1/x) - x sin(1/y)| \leq |x| |sin(1/x) - sin(1/y)| \leq |x-y|\]

For the second term, let's use the mean value theorem. Applying this
theorem to $g(x) = x sin(1/x)$, we get:
\[|x sin(1/y) - y sin(1/y)| = |g(x) - g(y)| \leq |g'(c)||x-y| = |(c cos(1/c) + sin(1/c))||x-y| \leq 2|x-y|\]

Thus, we now have:
\[|f(x) f(y)| \leq 3|x-y|\]

To make $|f(x) - f(y)| < \epsilon$, we can just do $\delta = \epsilon/3$.
Then, $\forall x, y \in (0, 1)$ such that $|x-y| < \delta$,
we have:
\[|f(x) - f(y)| \leq 3|x-y| < 3 \delta = \epsilon\]

Therefore, $f(x) = x sin(1/x)$ is uniformly continuous on (0, 1).

\end{solution}

\question 
Show that if $f:[0, \infty)$ is uniformly continuous on $[0,a]$ and on $[a,\infty)$ then it is uniformly continuous on all of $[0, \infty)$. Note that this implies that $f(x)= \sqrt{x}$ is uniformly continuous on all of $[0, \infty)$.
\begin{solution}

Let $\epsilon > 0$. Since f is uniformly continuous on [0, a], $\exists \delta_1 > 0$
such that $\forall x, y \in [0, a]$, if $|x-y| < \delta_1$,
then $|f(x) - f(y)| < \epsilon/2$.

Similarly, Since f is uniformly continuous on $[0, \infty)$, $\exists \delta_2 > 0$
such that $\forall x, y \in [0, \infty)$, if $|x-y| < \delta_2$,
then $|f(x) - f(y)| < \epsilon/2$.

Let $\delta = min\{\delta_1, \delta_2\}$. Now, consider any
$x, y \in [0, \infty)$ such that $|x-y| < \delta$.

1. If $x, y \in [0, a]$, then $|f(x) - f(y)| < \epsilon/2 < \epsilon$.
2. If $x, y \in [a, \infty)$, then $|f(x) - f(y)| < \epsilon/2 < \epsilon$.
3. If $x \in [0, a]$ and $y \in [\infty)$ (or vice versa),
assume $x \leq a \leq y$. Since $|x-y| < \delta \leq \delta_1$, we
get $|f(x)-f(a)| < \epsilon/2 < \epsilon$. Similarly, since
$|a-y| \leq |x-y| < \delta \leq \delta_2$, we have
$|f(a) - f(y)| < \epsilon/2 < \epsilon$. By the triangle inequality,
\[|f(x)-f(y)| \leq |f(x)-f(a)| + |f(a)-f(y)| < \epsilon/2 + \epsilon/2 = \epsilon\]

Thus, in all cases, $|f(x)-f(y)| < \epsilon$.
Therefore, f is uniformly continuous on all of $[0, \infty)$.

To follow up on the note mentioned above, since the function
$f(x) = \sqrt{x}$ is uniformly continuous on any
closed interval $[0, a]$ as well as on $[a, \infty)$
$\forall a > 0$, then $\sqrt{x}$ is uniformly continuous on
all of $[0, \infty)$.

\end{solution}

\end{questions}
\end{document}