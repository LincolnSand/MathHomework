\documentclass[answers]{exam}
\title{Foundations of Analysis, HW 8}
\author{James Cameron}
\date{}
\usepackage{amsmath}
\usepackage{amssymb}
\usepackage{graphicx}
\usepackage{amsthm}

\newtheorem{thm}{Theorem}
\newtheorem{hthm}[thm]{*Theorem}
\newtheorem{lemma}[thm]{Lemma}
\newtheorem{cor}[thm]{Corollary}
\newtheorem{obs}[thm]{Observation}
\newtheorem{prop}[thm]{Proposition}
\newtheorem{con}[thm]{Conjecture}
\newtheorem{exer}[thm]{Exercise}

\newtheorem{scho}[thm]{Scholium}
\newtheorem*{Thm}{Theorem}
\newtheorem*{Con}{Conjecture}
\newtheorem*{Axiom}{Axiom}

\theoremstyle{remark}
\newtheorem{remark}[thm]{Remark}
\newtheorem{notation}[thm]{Notation}


\theoremstyle{definition}
\newtheorem{Def}[thm]{Definition}
\newtheorem{example}[thm]{Example}
\newtheorem{ques}[thm]{Question}
\everymath{\displaystyle}

\pagestyle{head}
\header{\bfseries\large Foundations of Analysis}{\bfseries\large Homework 8 \ifprintanswers (solutions) \fi}{\bfseries\large Fall, 2023}
\headrule

\DeclareMathOperator{\sech}{sech}
\newcommand{\NN}{\mathbb{N}}
\newcommand{\RR}{\mathbb{R}}
\newcommand{\QQ}{\mathbb{Q}}
\newcommand{\ZZ}{\mathbb{Z}}
\newcommand{\dV}{\;\mathrm{d}V}
\newcommand{\dA}{\;\mathrm{d}A}
\newcommand{\dx}{\;\mathrm{d}x}
\newcommand{\dy}{\;\mathrm{d}y}
\newcommand{\dz}{\;\mathrm{d}z}
\pointname{}
\newcommand{\cA}{\mathcal{A}}
\newcommand{\Bb}{\mathcal{B}}
\newcommand{\Ww}{\mathcal{W}}
\newcommand{\Dd}{\mathcal{D}}
\newcommand{\Ss}{\mathcal{S}}
\newcommand{\Ee}{\mathcal{E}}
\DeclareMathOperator{\im}{im}
\begin{document}
% \maketitle

% \noindent
% This week on the homework you will get practice with induction. \\


% \noindent
 This is due Saturday 11/4 by 11:59 pm on Gradescope. Please either neatly write up your solutions or type them up. You can find a .tex template on Canvas. Your proofs should be written in complete sentences and paragraphs, using a combination of words and symbols. They should be \textbf{correct, clear, and concise}. You will be graded on all three, especially the first two!

\noindent


\begin{questions}

\question[4] Suppose that $\{b_k\}$ is a sequence of bounded numbers. Show that the sequences of functions $f_n:(-1,1) \to \RR$ defined by $f_n(x)= \sum_{k=0}^n b_kx^k$ converge (not necessarily uniformly) to a continuous function $f$ on $(-1,1)$. \textbf{Hint:} Show that for any $r \in (0,1)$ that the sequence $\{f_n\}$ is uniformly Cauchy on $[-r,r]$. Be sure to explain why showing this suffices to solve the problem. Feel free to use the fact, which follows from the geometric series, that $\sum_{k=0}^n |x|^k \le \frac{1}{1-|x|}$ for $x \in (-1,1)$. \textbf{Further Hint:} In class we did this when all the $b_k$ where $1$, so you might look at your notes for this.
\begin{solution}

To show that $\{f_n\}$ converges, we have to prove that it is uniformly Cauchy
on any closed subinterval [-r, r] for $0 < r < 1$. A sequence of functions
is uniformly Cauchy if for every $\epsilon > 0$, $\exists N \in \NN$ such that
$\forall m, n \geq N$ and all $x \in [-r, r]$, we have:
\[|f_n(x) - f_m(x)| < \epsilon\]

Let's take $m > n$, then:
\[|f_m(x) - f_n(x)| = |\Sigma_{k = n+1}^m b_k x^k|\]

From the triangle inequality, this is:
\[\leq \Sigma_{k=n+1}^m |b_k||x|^k\]

Since $\{b_k\}$ is bounded, $\exists$ a bound B such that
$|b_k| \leq B$ $\forall k$. Then we have:
\[\leq B \Sigma_{k=n+1}^m |x|^k\]

For $x \in [-r, r]$, $|x|^k \leq r^k$, thus:
\[\leq B \Sigma_{k=n+1}^m r^k\]

We can bound the series further by using the geometric
series sum formula for $r < 1$:
\[\Sigma_{k=n+1}^m r^k \leq \Sigma_{k=n+1}^{\infty} r^k = \frac{r^{n+1}}{1-r}\]

Hence:
\[|f_n(x) - f_m(x)| \leq B \frac{r^{n+1}}{1-r}\]

As $n \to \infty$, the term $r^{n_1} \to 0$ for $r < 1$. This means the right
side is arbitrarily small. Thus, $\{f_n\}$ is uniformly Cauchy.

Since every point in $(-1, 1)$ is contained in some interval $[-r, r]$,
where $0 < r < -1$, we can extend the continuity of f to all of $(-1, 1)$
from the continuity of f on every subinterval. This extension is well-defined
and continuous because for any two such intervals that overlap, the limit function
f will be continuous on their union, which is also a closed interval.

Therefore, by showing that $\{f_n\}$ is uniformly Cauchy on every closed subinterval
of $(-1, 1)$, we have established that the sequence of functions converges uniformly
on these intervals to a continuous function f on $(-1, 1)$. This uniform convergence
on every compact subinterval implies pointwise convergence on $(-1, 1)$ to
the same continuous function f.

\end{solution}
\question[4] In the previous question, show that if the sequece $\{b_k\}$ is constantly one then the sequence $\{f_n\}$ (defined as in the previous problem) converges pointwise to $f(x)=\dfrac{1}{1-x}$, but the convergence is \emph{not} uniform.
\begin{solution}

If $\{b_k\}$ is constantly one, then each $f_n(x)$ is the partial sum
of the geometric series. In this case, it's given by:
\[f_n(x) = \Sigma_{k=0}^n x^k\]

Because of the properties of geometric series, the sequence
$\{f_n\}$ converges pointwise to the function:
\[f(x) = \frac{1}{1-x}\]

Now for uniform convergence. For the sequence $\{f_n\}$ to converge
uniformly to $f(x)$, we must have that for every $\epsilon > 0$,
$\exists$ an N such that $\forall n \geq N$ and $\forall x \in (-1, 1)$, we have:
\[|f_n(x) - \frac{1}{1-x}| < \epsilon\]

To see why the convergence is not uniform, consider what happens as x gets
very close to 1. The difference between $f_n(x)$ and $f(x)$ is:
\[|\Sigma_{k=0}^n x^k - \frac{1}{1-x}| = |\frac{1-x^{n+1}}{1-x} - \frac{1}{1-x}| = \frac{-x^{n+1}}{1-x}\]

This simplifies to:
\[|x^{n+1}|\]

As x approaches 1, for any fixed n, $x^{n+1}$ approaches 1. Therefore, regardless
of how big n is, if x is close enough to 1, the difference $|x^{n+1}|$ will
be larger than any fixed $\epsilon > 0$. Hence, it cannot be uniformly
convergent.

\end{solution}

\question[4] Show that if $\{f_n\}$ is a sequence of uniformly continuous functions $f_n: D \to \RR$ that converge uniformly to $f:D \to \RR$, then $f$ is also uniformly continuous. \textbf{Note:} We proved this in class for \emph{continous functions}, and said that the same proof worked for \emph{uniformly continuous} functions. So, you should appropriately adapt the proof for continuous functions that we gave in class.
\begin{solution}

Let $\epsilon > 0$ be given. Since $\{f_n\}$ converges uniformly to f,
$\exists$ an $N \in \NN$ such that $\forall n \geq N$ and $\forall x \in D$,
we have:
\[|f_(x)-f(x)| < \frac{\epsilon}{3}\]

Now, because $f_N$ is uniformly continuous, $\exists$ a $\delta > 0$
such that $\forall x, y \in D$ with $|x-y| < \delta$, we get:
\[f_N(x) - f_N(y)| < \frac{\epsilon}{3}\]

Now let's consider any two points $x, y \in D$ such that $|x-y| < \delta$. From
the triangle inequality and the above properties, we have:
\[|f(x) - f(y)| \leq |f(x) - f_N(x)| + |f_N(x) - f_N(y)| + |f_N(y) - f(y)| < \frac{\epsilon}{3} + \frac{\epsilon}{3} + \frac{\epsilon}{3} = \epsilon\]

Therefore f is uniformly continuous on D.

\end{solution}

\question[4] Suppose that $f: (a,b) \to \RR$ and that  either $u=a^+, u=b^-$, or $u \in (a,b) $. Show that if $\lim_{x \to u} f(x)>0$, then there is some $\delta >0$ such that for all $ x \in (a,b)$ with $|x-u|< \delta$ we have that $f(x)>0$.
\begin{solution}

Recall the definition of the limit:
For $\lim_{x \to u} f(x) > 0$, it means that for any $\epsilon > 0$,
$\exists$ a $\delta > 0$ such that $\forall x$ within the interval $(a,b)$
and $0 < |x-u| < \delta$, it holds that $f(x)$ is within $\epsilon$ distance
from some positive limit L, where $L > 0$.

Let $\epsilon = \frac{L}{2} > 0$. From the definition of the limit, $\exists$
a corresponding $\delta > 0$ such that if $0 < |x-u| < \delta$ and $x \in (a, b)$,
then $|f(x) - L| < \epsilon$.

Since $L > 0$, this means the values of $f(x)$ are contained in the interval
$(L - \epsilon, L + \epsilon) = (\frac{L}{2}, \frac{3 L}{2})$. All the
values of f(x) are positive because $\frac{L}{2}$ is positive (since $L > 0$).

Hence, we have that $\forall x \in (a, b)$ such that $0 < |x-u| < \delta$,
$f(x) > \frac{L}{2} > 0$. This means that $f(x)$ is positive in the neighborhood
of $u$, excluding $u$ itself (if $u \notin (a, b)$).

That means there are 3 cases for u:
1. $u = a^+$, then the $\delta$-neighborhood is $(a, a + \delta)$.
2. $u = b^-$, then the $\delta$-neighborhood is $(b - \delta, b)$.
3. $u \in (a, b)$, then the $\delta$-neighborhood is $(u - \delta, u + \delta)$ intersected with $(a, b)$.

In each case, within the appropriate $\delta$-neighborhood, $f(x) > 0$.

\end{solution}

\question[4] Proceeding directly from the definition of a limit, show that $\lim_{ x \to \infty} \frac{2x^2+1}{3x^2-1}= \frac{2}{3}$.
\begin{solution}

To show that $\lim_{x \to \infty} \frac{2x^2+1}{3x^2-1} = \frac{2}{3}$, we need
to show that for every $\epsilon > 0$, $\exists$ a number $M > 0$
such that $\forall x > M$, the absolute difference between $\frac{2x^2+1}{3x^2-1}$
and $\frac{2}{3}$ is less than $\epsilon$.

Let's compute the difference and try to bound it by $\epsilon$:
\[|\frac{2x^2+1}{3x^2-1} - \frac{2}{3}| = \frac{5}{|9x^2 - 3|}\]

As $x \to \infty$, the deonominator will approach $\infty$,
so the whole expression approaches 0. But we still need to show this happens
in a controlled way with respect to $\epsilon$.

We want:
\[\frac{5}{|9x^2 - 3|} < \epsilon\]

To find the appropriate M, we solve for x in terms of $\epsilon$:
\[5 < \epsilon|9x^2 - 3|\]
\[\frac{5}{\epsilon} < |9x^2 - 3|\]

For $x^2 > \frac{1}{3}$, the value will be positive,
so we can ignore the absolute value sign.
\[\frac{5}{9x^2 - 3} < \epsilon \implies 9x^2 - 3 > \frac{5}{\epsilon}\]
\[9x^2 > \frac{5}{\epsilon} + 3\]
\[x^2 > \frac{5}{9 \epsilon} + \frac{1}{3}\]

Therefore,
\[x > \sqrt{\frac{5}{9 \epsilon} + \frac{1}{3}}\]

If we choose M to be:
\[M > \max\{\frac{1}{\sqrt{3}}, \sqrt{\frac{5}{9 \epsilon} + \frac{1}{3}}\}\]

we ensure that $9x^2 - 3$ is both positive and the fraction is less than $\epsilon$
$\forall x > M$. Thus, with the above value of M, we get:
\[|\frac{2x^2 + 1}{3x^2 - 1} - \frac{2}{3}| < \epsilon, \forall x > M.\]

Which proves that $\lim_{x \to \infty} \frac{2x^2 + 1}{3x^2 - 1} = \frac{2}{3}$.

\end{solution}

\end{questions}
\end{document}