\documentclass[answers]{exam}
\title{Foundations of Analysis, HW 3}
\author{James Cameron}
\date{}
\usepackage{amsmath}
\usepackage{amssymb}
\usepackage{graphicx}
\usepackage{amsthm}

\newtheorem{thm}{Theorem}
\newtheorem{hthm}[thm]{*Theorem}
\newtheorem{lemma}[thm]{Lemma}
\newtheorem{cor}[thm]{Corollary}
\newtheorem{obs}[thm]{Observation}
\newtheorem{prop}[thm]{Proposition}
\newtheorem{con}[thm]{Conjecture}
\newtheorem{exer}[thm]{Exercise}

\newtheorem{scho}[thm]{Scholium}
\newtheorem*{Thm}{Theorem}
\newtheorem*{Con}{Conjecture}
\newtheorem*{Axiom}{Axiom}

\theoremstyle{remark}
\newtheorem{remark}[thm]{Remark}
\newtheorem{notation}[thm]{Notation}


\theoremstyle{definition}
\newtheorem{Def}[thm]{Definition}
\newtheorem{example}[thm]{Example}
\newtheorem{ques}[thm]{Question}
\everymath{\displaystyle}

\pagestyle{head}
\header{\bfseries\large Foundations of Analysis}{\bfseries\large Homework 3 \ifprintanswers (solutions) \fi}{\bfseries\large Fall, 2023}
\headrule

\DeclareMathOperator{\sech}{sech}
\newcommand{\NN}{\mathbb{N}}
\newcommand{\RR}{\mathbb{R}}
\newcommand{\QQ}{\mathbb{Q}}
\newcommand{\ZZ}{\mathbb{Z}}
\newcommand{\dV}{\;\mathrm{d}V}
\newcommand{\dA}{\;\mathrm{d}A}
\newcommand{\dx}{\;\mathrm{d}x}
\newcommand{\dy}{\;\mathrm{d}y}
\newcommand{\dz}{\;\mathrm{d}z}
\pointname{}
\newcommand{\cA}{\mathcal{A}}
\newcommand{\Bb}{\mathcal{B}}
\newcommand{\Ww}{\mathcal{W}}
\newcommand{\Dd}{\mathcal{D}}
\newcommand{\Ss}{\mathcal{S}}
\newcommand{\Ee}{\mathcal{E}}
\DeclareMathOperator{\im}{im}
\begin{document}
% \maketitle

% \noindent
% This week on the homework you will get practice with induction. \\


% \noindent
 This is due Saturday 9/23 by 11:59 pm on Gradescope. Please either neatly write up your solutions or type them up. You can find a .tex template on Canvas. Your proofs should be written in complete sentences and paragraphs, using a combination of words and symbols. They should be \textbf{correct, clear, and concise}. You will be graded on all three, especially the first two!

\noindent


\begin{questions}

\question Show that if $D$ and $D'$ are dedekind cuts, then $D+D'$ is also a Dedekind cut. Recall that $D+D'=\{ x\in \QQ: x= d+d', ~\textrm{for some}~ d \in D, d' \in D' \}$. Do these directly from the definition of dedekind cuts. \textbf{Remark:} This is one of the  things that needs to be checked in the course of showing that the set of Dedekind cuts forms an ordered field.

\begin{solution}

A dedekind cut is a set with a least upper bound.
So $D$ has least upper bound $d$ and $D'$ has least upper bound $d'$.
Since we sum up all of the entries of $D$ and $D'$ to get the entries of
$D + D'$, then we get a least upper bound of $d + d'$.
And since we have a least upper bound for $D + D'$,
then $D + D'$ is a dedekind cut.

\end{solution}


\question Show that for $S$ a subset of $\RR$ the following are equivalent:
\begin{enumerate}
\item For all $x,y \in \RR$ with $x<y$, there is an $s \in S$ so that $x<s<y$.
\item For all $x \in \RR$ and for all $\epsilon \in \RR_{>0}$ (this means the nonnegative reals), there is an $s \in S$ so that $|x-s|< \epsilon$.
\end{enumerate}
(Subsets of the reals that satisfies these equivalent properties are called \emph{dense})

\begin{solution}

Claim $1) \iff 2)$.

I should note that for the rest of this proof,
I am assuming $S$ is a set such that either 1) or 2) holds.
In the case that this isn't true, neither would hold
because of the fact that the Archimedes' principle would fail.
I am concerning myself with the non-trivial case:
showing that they must both be true if one of them is.


\[x < y
\implies 0 < y-x\]
By Archimedes' principle
\[\implies \frac{1}{n} < y-x\]
for some $n \in \NN$, where $n > 0$.

\[\implies x + \frac{1}{n} < y\]
and
\[x < x + \frac{1}{n}\]
So,
\[x < x + \frac{1}{n} < y\]

Let $s = x + \frac{1}{n}$ and we get $1)$.

Also,
\[x < x + \frac{1}{n} < y < y + \frac{1}{n}\]
\[\implies x < y + \frac{1}{n}\]
\[x-y < y + \frac{1}{n}\]
and
\[y-x < y + \frac{1}{n}\]
That means:
\[|y-x| < |y| + \frac{1}{n}\]

Note that $|y| + \frac{1}{n}$ is the definition of
all positive real numbers.

If we let $|y| + \frac{1}{n}$ be $\epsilon$ and have $s = x$,
we get $2)$ with minimal rewriting.

 \end{solution}


\question Compute the supremum and infimum of the following sets. Prove that your answers are correct (remember for us that $\NN$ doesn't contain $0$).

    \begin{parts}\vspace{5pt}
        \part $\{m/n : m,n \in \NN \text{ with } m<n\}$\vspace{5pt}
        \part \(\{n/(5n+1) : n \in \NN\}\)\vspace{5pt}
    \end{parts}
\begin{solution}
\begin{parts}
\part The infimum is 0 and the supremum is 1.

\[0 < \frac{m}{n} < 1\]

Let us set $m$ to the $inf(\NN) = 1$.
Then we increase $n$ to be arbitrarily high.
As $n$ approaches $\inf$ ($sup(\NN)$), then $\frac{1}{n} approaches 0$.
This must be the infimum since we are maximizing the
difference between $n$ and $m$ where $n > m$.
It can't be larger or else it would fall within the set.

For the opposite case, we want to decrease the difference between $n$
and $m$ where $n > m$.
Since we are dealing with $\NN$, the smallest difference between two
natural numbers where they aren't equal is $1$.
If we set $m$ to be arbitrarily large and then set $n = m+1$,
then as $m$ gets larger, the closer that $\frac{m}{m+1}$ gets to $1$.
So the supremum must be $1$.
It can't be smaller than $1$ or else it would fall within the set.

\part The infimum is $\frac{1}{6}$ and the supremum is $\frac{1}{5}$.

Similar reasoning to above holds.

If we minimize the difference between
$n$ and $5n + 1$, then $n = 1$.
That leaves us with $\frac{1}{6}$. Since $\frac{1}{6}$
is a lower bound and in the set, it must be the infimum.

If we maximize the difference, then we have $n$ be arbitrarily big,
which means $\frac{n}{5n+1}$ will approach $\frac{1}{5}$.
The supremum of the set can't be lower or else it would be in the set.
Therefore the supremum of the set must be $\frac{1}{5}$.


\end{parts}
\end{solution}

\question Let $A,B$ be non-empty subsets of $\RR$. Prove that if $A \subseteq B$, then \[\inf B \leq \inf A \leq \sup A \leq \sup B.\]

\begin{solution}

Let's consider $A$. Since $A$ is a subset of $B$,
then the infimum of $A$ must be a lower bound of $B$.
But let us consider the case of $B = \{1, 2\}; A = \{2\}$.
In this case, the infimum of $A$ is greater than the infimum of $B$.
Therefore, since the infimum of $B$ is a lower bound of $A$, but the
infimum of $B$ can be greater than the infimum of $A$, then the infimum
of $A$ must be greater than or equal to the infimum of $B$.

Now, for the case of the supremum. Since $B$ is a superset of $A$,
that means that the supremum of $B$ must be an upper bound of $A$.
But let us consider the case of $B = \{1, 2\}; A = \{1\}$.
In this case, the supremum of $A$ is less than the supremum of $B$.
Therefore, since the supremum of $B$ is an upper bound of $A$, but the
supremum of $A$ can be less than the supremum of $B$, then the supremum
of $A$ must be less than or equal to the supremum of $B$.

\end{solution}


\question Let $A$ and $B$ be non-empty subsets of $\RR$. Prove that $\sup (A\cup B) = \max\{\sup A, \sup B\}.$

\begin{solution}

Let $C = A \cup B$

Since $A \subseteq C$, $sup(A) \leq sup(C)$.
Since $B \subseteq C$, $sup(B) \leq sup(C)$.

But the only way that this can be true is if
$sup(C) = \max\{\sup A, \sup B\}$ because otherwise
$sup(C) < sup(A)$ or $sup(C) < sup(B)$.

For instance, if $sup(A) < sup(B)$,
then $sup(C) = sup(B)$.
This satisfies that $sup(B) \leq sup(C)$ since $sup(C) = sup(B)$
and $sup(A) \leq sup(C)$ since $sup(A) < sup(B)$.

Now, if $sup(B) < sup(A)$,
then $sup(C) = sup(A)$.
This satisfies that $sup(A) \leq sup(C)$ since $sup(C) = sup(A)$
and $sup(B) \leq sup(C)$ since $sup(B) < sup(A)$.

And in the csae that $sup(A) = sup(B)$,
trivially both $sup(A) \leq sup(C)$ and $sup(B) \leq sup(C)$
are satisfied because
$sup(A) = sup(B) = sup(C)$.

\end{solution}

\end{questions}
\end{document}
