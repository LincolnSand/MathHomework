\documentclass[answers]{exam}
\title{Foundations of Analysis, HW 1}
\author{James Cameron}
\date{}
\usepackage{amsmath}
\usepackage{amssymb}
\usepackage{graphicx}
\everymath{\displaystyle}

\pagestyle{head}
\header{\bfseries\large Foundations of Analysis}{\bfseries\large Homework 1 \ifprintanswers (solutions) \fi}{\bfseries\large Fall, 2023}
\headrule

\DeclareMathOperator{\sech}{sech}
\newcommand{\NN}{\mathbb{N}}
\newcommand{\RR}{\mathbb{R}}
\newcommand{\QQ}{\mathbb{Q}}
\newcommand{\ZZ}{\mathbb{Z}}
\newcommand{\dV}{\;\mathrm{d}V}
\newcommand{\dA}{\;\mathrm{d}A}
\newcommand{\dx}{\;\mathrm{d}x}
\newcommand{\dy}{\;\mathrm{d}y}
\newcommand{\dz}{\;\mathrm{d}z}
\pointname{}
\newcommand{\cA}{\mathcal{A}}
\newcommand{\Bb}{\mathcal{B}}
\newcommand{\Ww}{\mathcal{W}}
\newcommand{\Dd}{\mathcal{D}}
\newcommand{\Ss}{\mathcal{S}}
\newcommand{\Ee}{\mathcal{E}}
\DeclareMathOperator{\im}{im}
\begin{document}
% \maketitle

% \noindent
% This week on the homework you will get practice with induction. \\


% \noindent
 This is due Saturday 9/2 by 11:59 pm on Gradescope. Please either neatly write up your solutions or type them up. You can find a .tex template on Canvas. Your proofs should be written in complete sentences and paragraphs, using a combination of words and symbols. They should be \textbf{correct, clear, and concise}. You will be graded on all three, especially the first two!

\noindent


\begin{questions}

\question Let $\NN$ be the natural numbers $\{0,1,2,3,4,\dots \}$, $O \subseteq \NN$ be the odd numbers $O=\{1,3,5,7,\dots \}$ and $E \subseteq \NN$ be the even numbers $\{0,2,4,6,\dots \}$.
\begin{parts}
\part[1] What is $E \cup O$?
\part[1] What is $\NN \smallsetminus E$?
\part[1] What is $O \cap E$?
\part[1] What is $\NN \smallsetminus (E \cap O)$?
\end{parts}
\begin{solution}
\begin{parts}
\part $E \cup O = \NN$.
This is true because any $n \in \NN$ can only ever be
an even number (a member of $E$) exclusive or be an odd number (a member of $O$).
And it has to be one of them. So, therefore the union of $E$ and $O$
must be the set of all natural numbers.
\part $\NN \smallsetminus E =  O$. This follows from the reasoning above.
If a natural number isn't even, then it must be odd.
\part $O \cap E = \varnothing$. Since all even numbers are not odd
and all odd numbers are not even,
the only thing that would be in their intersection would be the empty set.
\part $\NN \smallsetminus (E \cap O) = \NN$. This is pretty obvious since
$(E \cap O)$ is $\varnothing$. And $A \smallsetminus \varnothing = A$
for any set $A$.
\end{parts}
\end{solution}
 
\question Recall that for $a,b \in \RR$ the interval $(a,b)$ is the open inverval between $a$ and $b$ (so $a,b \not\in (a,b)$), that $[a,b]$ is the closed interval (so $a,b \in (a,b)$), and that $[a,b)$ and $(b,a]$ are the half open invervals. The rays $(a,\infty)$, $[a,\infty)$, $(-\infty,a)$, $(-\infty,a]$ are also (hopefully!\footnote{Ask me if not!}) familiar from past classes. 

Let $\cA=\{ (x,y) \subseteq \RR: 0<x ~\textrm{and}~ y<1 \}$.

\begin{parts}

\part[2] What is $\cap \cA$? This is some interval or union of intervals, say what it is. Be sure to justify your answer.
\part[2] What is $\cup \cA$? This is again some interval or union of intervals, say what it is. Be sure to justify your answer.
\end{parts}
\begin{solution}
\begin{parts}
\part $\cap \cA = (0, 1)$. This is because all of the intervals must overlap
this region in order to be valid members of the set. If we consider an
interval to contain all the numbers within it, then all of these intervals
in the set must contains the numbers between 0 and 1. It is not inclusive
because the overlap range uses stricly less than instead of less than or equal to.
\part $\cup \cA = (-\infty, \infty)$. This is because,
if we consider an interval to be a set containing all
the numbers within that interval range, then every real number
would appear in either the $a$ and/or the $b$ position, so they would
belong to the set. Therefore the interval of the union would be the
interval containing all the real numbers.
\end{parts}
\end{solution}

\question[4] Show that if $A$ and $B$ are subsets of $X$ then $X \smallsetminus (A \cap B) = (X \smallsetminus A) \cup (X \smallsetminus B)$.
\begin{solution}
This is true because $X \smallsetminus (A \cap B)$
eliminates all of the elements of $X$ that are both in $A$ and in $B$.
Let's compare this to $(X \smallsetminus A) \cup (X \smallsetminus B)$.
This is taking the elements in $X$ that are not in $A$ and adding them
to the elements in $X$ that are not in $B$.
Well, in the case that an element is in $B$, but is not in $A$,
the second set being unioned would add it back and so
it would end up in the final set.
The same holds for the opposite case where an element in $X$ is in $A$ and not in $B$:
The first set adds it back and so it ends up in the final set.
The only time an element in $X$ wouldn't end up in the final set
is if it is in both $A$ and $B$. This is because it then is in neither
the first set or the second set being unioned, and so it never gets "added back"
into the final set. And that is exactly the definition of subtracting the
intersection of $A$ and $B$ from $X$.
\end{solution}

\question[4] Show that if $f: X \to Y$ is injective if and only if for every pair of subsets $A,B \subset X$ we have that $f(A \cap B) = f(A) \cap f(B)$.

\begin{solution}
Suppose that $f(A) = a \in Y, f(B) = b \in Y$.
Well, by the definition of injective functions,
$a = b \implies f(A) = f(B)$.
In other words, all function mappings from the domain to the codomain are unique.
Therefore, $\forall x \in (a \cap b) \implies \exists X \in (A \cap B)$
where $f(X) = x$.
And since functions can only ever map a value in the domain to a
single value in the codomain, that means that the opposite also holds:
$\forall X \in (A \cap B) \implies \exists x \in (a \cap b)$ where $x = f(X)$.
Therefore, if we substitute in the definition $a$ and $b$, it is equivalent to:
$f(A \cap B)= f(A) \cap f(B)$.
\end{solution}

\question Suppse that $f:X \to Y$ and $g:Y \to Z$ are functions.
\begin{parts}
\part[3] Show that if $g \circ f$ is onto, then $g$ is onto.
\part[1] Give an example to show that if $g$ is onto then $g \circ f$ is not necessarily onto.
\end{parts}

\begin{solution}
\begin{parts}
\part Let us consider the case where $g \circ f$ is onto and $g$ is not.
In that case, for $g \circ f \implies \forall z \in Z, \exists x \in X$.
But in the case of $g$, $g \implies \exists z \in Z, \forall y \in Y, g(y) \neq z$.
This is logically impossible. $g \circ f$ is telling us that all elements in $Z$
must have a mapping to them. But $g$ is telling us that there are elements in $Z$
where no mapping to them exists. In other words, $f$ transports some elements into
$Y$ from $X$, but once they are in $X$, there is no way to construct a mapping
so that each element in $Z$ is in the range, no matter how $f$ did the mapping.
Therefore, if $g \circ f$ is onto, then $g$ must be as well.
If this weren't the case, we'd end up with a logical contradiction.
\part Let $A = \{a, b\}, g: A \longrightarrow A, f: A \longrightarrow A$.
Now, $g(x) = x$ and $f(x) = a$.
In this case, $g$ (the identity function) is trivially onto.
$a = g(a); b = g(b)$.
But $g \circ f$ is not onto.
This is because of the fact that $f$ always maps to $a$, so $g$ has
no option but to map $f$'s output to $a$.
This means there is no way for $g \circ f$ to ever output $b$.
$(g \circ f)(a) = a; (g \circ f)(b) = a;$.
Therefore, $g$ is onto, but $g \circ f$ is not.
\end{parts}
\end{solution}
\end{questions}
\end{document}