\documentclass[answers]{exam}
\title{Foundations of Analysis, HW 3}
\author{James Cameron}
\date{}
\usepackage{amsmath}
\usepackage{amssymb}
\usepackage{graphicx}
\usepackage{amsthm}

\newtheorem{thm}{Theorem}
\newtheorem{hthm}[thm]{*Theorem}
\newtheorem{lemma}[thm]{Lemma}
\newtheorem{cor}[thm]{Corollary}
\newtheorem{obs}[thm]{Observation}
\newtheorem{prop}[thm]{Proposition}
\newtheorem{con}[thm]{Conjecture}
\newtheorem{exer}[thm]{Exercise}

\newtheorem{scho}[thm]{Scholium}
\newtheorem*{Thm}{Theorem}
\newtheorem*{Con}{Conjecture}
\newtheorem*{Axiom}{Axiom}

\theoremstyle{remark}
\newtheorem{remark}[thm]{Remark}
\newtheorem{notation}[thm]{Notation}


\theoremstyle{definition}
\newtheorem{Def}[thm]{Definition}
\newtheorem{example}[thm]{Example}
\newtheorem{ques}[thm]{Question}
\everymath{\displaystyle}

\pagestyle{head}
\header{\bfseries\large Foundations of Analysis}{\bfseries\large Homework 6 \ifprintanswers (solutions) \fi}{\bfseries\large Fall, 2023}
\headrule

\DeclareMathOperator{\sech}{sech}
\newcommand{\NN}{\mathbb{N}}
\newcommand{\RR}{\mathbb{R}}
\newcommand{\QQ}{\mathbb{Q}}
\newcommand{\ZZ}{\mathbb{Z}}
\newcommand{\dV}{\;\mathrm{d}V}
\newcommand{\dA}{\;\mathrm{d}A}
\newcommand{\dx}{\;\mathrm{d}x}
\newcommand{\dy}{\;\mathrm{d}y}
\newcommand{\dz}{\;\mathrm{d}z}
\pointname{}
\newcommand{\cA}{\mathcal{A}}
\newcommand{\Bb}{\mathcal{B}}
\newcommand{\Ww}{\mathcal{W}}
\newcommand{\Dd}{\mathcal{D}}
\newcommand{\Ss}{\mathcal{S}}
\newcommand{\Ee}{\mathcal{E}}
\DeclareMathOperator{\im}{im}
\begin{document}
% \maketitle

% \noindent
% This week on the homework you will get practice with induction. \\


% \noindent
 This is due Wednesday 10/18 by 11:59 pm on Gradescope. Please either neatly write up your solutions or type them up. You can find a .tex template on Canvas. Your proofs should be written in complete sentences and paragraphs, using a combination of words and symbols. They should be \textbf{correct, clear, and concise}. You will be graded on all three, especially the first two!

\noindent


\begin{questions}

\question[4] Directly from the definition of continuity, show that $f: \RR_{>0} \to \RR$ defined by $f(x)=1/x$ is continuous at every $a \in \RR_{>0}$.
\begin{solution}

$f(x) = \frac{1}{x}$ and we want to show that f is continuous at some arbitrary point.
Let's call this point a. Let $\epsilon > 0$. We need to find a $\delta > 0$ such that,
$0 < |x - a| < \delta \implies |\frac{1}{x} - \frac{1}{a}| < \epsilon$.

Recall:
\[\left| \frac{1}{x} - \frac{1}{a} \right| = \left| \frac{a-x}{ax} \right| = \frac{|a-x|}{ax}\]

Since it has to be less than $\epsilon$, that means that $\frac{|a-x|}{ax} < \epsilon$.

For $\delta$, we can do:

\[|x-a| < \delta \implies \frac{|a-x|}{ax} < \epsilon\]

Now we have to rewrite $\delta$ in terms of $\epsilon$.
For $|a-x| < \delta$, we can get:

\[\frac{|a-x|}{ax} \leq \frac{\delta}{a(a-\delta)} \leq \frac{\delta}{a(a-z^2 \epsilon/2)}\]
\[= \frac{2\delta}{a^2 \epsilon} \leq \epsilon\]

We have now shown that for every $\epsilon > 0$, we can find a $\delta > 0$,
such that if $|x - a| < \delta$, $|\frac{1}{x} - \frac{1}{a}| < \epsilon$.
This means that the function $f(x) = \frac{1}{x}$ is continous forall of $\RR_{>0}$.

\end{solution}

\question[4] Show that $f: \RR \to \RR$ defined by $f(x)= \begin{cases}
0 &x=0 \\
\sin(1/x) & x \not=0
\end{cases}$ is \emph{not} continuous at $x=0$. \textbf{Hint:} use the sequential formulation of continuity.
\begin{solution}

The function is continuous at 0 if, forall sequences $(x_n)_{n=1}^{\infty}$ in $\RR$
that converges to 0, the sequence $(f(x_n))_{n=1}^{\infty}$ converges to f(0) = 0.

Consider the sequence defined by $x_n = \frac{1}{2 \pi n}$.
This sequence converges to 0 as n goes to infinity.
But we have,

\[f(x_n) = sin(2 \pi n) = 0\]

So the sequence $f(x_n)$ is constantly 0 and thus trivially converges to 0.

Now consider the sequence $y_n = \frac{1}{2 \pi n + \frac{\pi}{2}}$.
This sequence also converges to 0 as n goes to infinity.
However, we have,

\[f(y_n) = sin(2 \pi n + \frac{\pi}{2}) = 1\]

Since the numbers aren't both 0, the function is not continuous at x = 0.

\end{solution}


\question[4] Show that if $h: \RR \to \RR$ is bounded then $g: \RR \to \RR$ defined by $g(x)=xh(x)$ is continuous at $0$. This let's you conclude that $f: \RR \to \RR$ defined by $f(x)= \begin{cases}
0 &x=0 \\
x\sin(1/x) & x \not=0
\end{cases}$ is continous on all of $\RR$.
\begin{solution}

To show that g(x) = xh(x)g(x) = xh(x) is continuous at 0,
we need to show that for every $\epsilon > 0$,
there exists a $\delta > 0$ such that $|x| < \delta \implies |g(x) - g(0)| < \epsilon$.

Since g(0) = 0, we want to show that $|xh(x)| < \epsilon$ whenever $|x| < \delta$.
Given $\epsilon > 0$, let $\delta = \frac{\epsilon}{M}$ where M is the bound of the function.
We get:

\[xh(x) \leq |x|M \leq \frac{\epsilon}{M} M = \epsilon\]

Therefore g is continous at 0.

For f(x), we know that $h(x) = sin(\frac{1}{x})$ is bounded, so,
by the argument above $g(x) = xsin(\frac{1}{x})$ is also continous at 0.
Since f(x) = 0 when x = 0 and $f(x) = xsin(\frac{1}{x})$ otherwise,
it is clear that f is continous at 0. And because $f(x) = xsin(\frac{1}{x})$
is continous forall other values, it means that f is continuous forall of $\RR$.


\end{solution}

\question[4] Suppose that $f: [a,b] \to \RR$ and $g: [a,b] \to \RR$ are continuous and that $f(a) \ge g(a)$, $f(b) \le g(b)$. Show that there is some $x \in [a,b]$ with $f(x)=g(x)$. \textbf{Hint:} Intermediate value theorem.
\begin{solution}

Recall:
Let $h(x) = f(x) - g(x)$. If f and g are continous, so is h.

Since $f(a) \geq g(a)$ and $f(b) \leq g(b)$, then
$h(a) \geq 0$ and $h(b) \leq 0$.

Now, from the Intermediate Value Theorem, since h is continous on the closed
interval [a, b] and changes sign, there must be an $x \in [a, b]$
such that h(x) = 0. i.e. $\exists x \in [a, b]$ such that
$f(x) - g(x) = 0$ or $f(x) = g(x)$.

\end{solution}

\question
\begin{parts}
\part[2] Give an example of a continuous function $f: D \to \RR$ with $D \subseteq \RR$ and a Cauchy sequence $\{a_n\}$ in $D$ such that $\{f(a_n)\}$ is not Cauchy.
\begin{solution}

Consider $f: (0, 1) \to \RR$ defined by $f(x) = \frac{1}{x}$. The domain D
is a subset of $\RR$ and f is continous on D.

Let $\{a_n\}$ be the sequence on D defined by $a_n = \frac{1}{n}$.
The sequence is a Cauchy sequence in D since forall D, $\exists N$
such that forall $n, m \geq N$, we get:

\[|a_n - a_m| = \left| \frac{1}{n} - \frac{1}{m} \right| = \frac{|n-m}{nm} < \epsilon\]

But $\{f(a_n)\}$ is not Cauchy even though $\{a_n\}$ is and f is continous.
The continuity of f is not sufficient to preserve the Cauchy
property of the sequence since the function is not uniformly continous on D.


\end{solution}
\part[2] Recall that $f: D \to \RR$ is \emph{uniformly continuous} if for all $\epsilon>0$ there is a $\delta>0$ such that if $x,y \in D$ and $|x-y|<\delta$, then $|f(x)-f(y)| < \epsilon$ (You met this concept on the week 7 worksheet).

Show that if $f: D \to \RR$ is \emph{uniformly continuous} then if $\{a_n\}$ is a Cauchy sequence in $D$ then $\{f(a_n)\}$ is also Cauchy.
\begin{solution}

Suppose that $\{a_n\}$ is a Cauchy sequence in D.
This implies that forall $\epsilon' > 0$, $\exists N$
such that forall $n, m \geq N$, we get $|a_n - a_m| < \epsilon'$.

Since f is uniformly continous, it allows us to select $\epsilon' = \delta$.
This means that if $|a_n - a_m| < \delta$, then $|f(a_n) - f(a_m) \epsilon$.

So, for any $\epsilon > 0$, we choose $\delta$ as described above and let N
be a value such that $|a_n - a_m| < \delta$ when $n, m \geq N$.
So, forall $n, m \geq N$, we end up with:

\[|f(a_n) - f(a_m)| < \epsilon\]

We have shown that $\{f(a_n)\}$ is Cauchy.

\end{solution}
\end{parts}

\end{questions}
\end{document}