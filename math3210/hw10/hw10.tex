\documentclass[answers]{exam}
\title{Foundations of Analysis, HW 10}
\author{James Cameron}
\date{}
\usepackage{amsmath}
\usepackage{amssymb}
\usepackage{graphicx}
\usepackage{amsthm}

\newtheorem{thm}{Theorem}
\newtheorem{hthm}[thm]{*Theorem}
\newtheorem{lemma}[thm]{Lemma}
\newtheorem{cor}[thm]{Corollary}
\newtheorem{obs}[thm]{Observation}
\newtheorem{prop}[thm]{Proposition}
\newtheorem{con}[thm]{Conjecture}
\newtheorem{exer}[thm]{Exercise}

\newtheorem{scho}[thm]{Scholium}
\newtheorem*{Thm}{Theorem}
\newtheorem*{Con}{Conjecture}
\newtheorem*{Axiom}{Axiom}

\theoremstyle{remark}
\newtheorem{remark}[thm]{Remark}
\newtheorem{notation}[thm]{Notation}


\theoremstyle{definition}
\newtheorem{Def}[thm]{Definition}
\newtheorem{example}[thm]{Example}
\newtheorem{ques}[thm]{Question}
\everymath{\displaystyle}

\pagestyle{head}
\header{\bfseries\large Foundations of Analysis}{\bfseries\large Homework 10 \ifprintanswers (solutions) \fi}{\bfseries\large Fall, 2023}
\headrule

\DeclareMathOperator{\sech}{sech}
\newcommand{\NN}{\mathbb{N}}
\newcommand{\RR}{\mathbb{R}}
\newcommand{\QQ}{\mathbb{Q}}
\newcommand{\ZZ}{\mathbb{Z}}
\newcommand{\dV}{\;\mathrm{d}V}
\newcommand{\dA}{\;\mathrm{d}A}
\newcommand{\dx}{\;\mathrm{d}x}
\newcommand{\dy}{\;\mathrm{d}y}
\newcommand{\dz}{\;\mathrm{d}z}
\pointname{}
\newcommand{\cA}{\mathcal{A}}
\newcommand{\Bb}{\mathcal{B}}
\newcommand{\Ww}{\mathcal{W}}
\newcommand{\Dd}{\mathcal{D}}
\newcommand{\Ss}{\mathcal{S}}
\newcommand{\Ee}{\mathcal{E}}
\DeclareMathOperator{\im}{im}
\begin{document}
% \maketitle

% \noindent
% This week on the homework you will get practice with induction. \\


% \noindent
 This is due Saturday 11/25 by 11:59 pm on Gradescope. Please either neatly write up your solutions or type them up. You can find a .tex template on Canvas. Your proofs should be written in complete sentences and paragraphs, using a combination of words and symbols. They should be \textbf{correct, clear, and concise}. You will be graded on all three, especially the first two!

\noindent


\begin{questions}

\question[4] Suppose that $f: \RR \to \RR$ is differentiable, and that $\lim_{x \to \infty} f'(x)= \infty$. Show that for all $M \in \RR$ there is some $N \in \RR$ so that for all $x$ and $y$ greater than $N$ we have that $|f(x)-f(y)|> M|x-y|$.

\begin{solution}

Let's first apply the MVT. For any two numbers x and y
with $x < y$, $\exists$ a c between x and y such that
\[f'(c) = \frac{f(y) - f(x)}{y - x}\]
i.e.
\[f(y) - f(x) = f'(c)(y - x)\]

We know that $\lim_{x \to \infty} f(x) = \infty$. This means that
for any given M, $\exists$ some N such that $\forall$ $x > N$,
f'(x) > M.

Choose any $x, y > N$ with $x < y$. From MVT, there is some c
between x and y such that $f(y) - f(x) = f'(c)(y - x)$.
Since $c > x > N$, $f'(c) > M$. Therefore,
$f(y) - f(x) = f'(c)(y - x) > M(y - x)$.

Notice that
\[|f(y) - f(x)| = |f'(c)||y-x| > M|y-x|\]

$f'(c) > M \implies |f'(c)| > M$.

Thus, we have shown that $\forall M \in \RR$, $\exists$ an $N \in \RR$
such that $\forall x, y > N$, $|f(x) - f(y)| > M|x-y|$.

\end{solution}

\question[4] Show that if $f: \RR \to \RR$ is differentiable and $\lim_{x \to \infty} f'(x)= \infty$ then $f$ is \emph{not} uniformly continuous on $\RR$.
\begin{solution}

We will show this by using a proof by contradiction.

Assume f is uniformly continuous on $\RR$. Then, for $\epsilon = 1$,
$\exists \delta > 0$ such that $\forall x, y \in \RR$, if $|x-y| < \delta$,
then $|f(x) - f(y)| < 1$.

Since $\lim_{x \to \infty} f'(x) = \infty$, $\exists$ some X
such that $\forall x > X$, $f'(x) > \frac{1}{\delta}$.

Choose $x, y > X$ such that $y-x < \delta$. This is always possible because
$\delta > 0$. According to our assumption of uniform continuity, $|f(x) - f(y)| < 1$.

This creates a contradiction because by MVT, $\exists$ a c between x and y
such that $f(y) - f(x) = f'(c)(y - x)$.
Since $c > X$, $f'(c) > \frac{1}{\delta}$, so
\[|f(y)-f(x)| = |f'(c)||y-x| > \frac{1}{\delta}|y-x|\].
But $|y-x| < \delta$, so $|f(y) - f(x)| > 1$, which contradicts the assumption
of uniform continuity that $|f(y) - f(x)| < 1$.

\end{solution}

\question[4]
Find $\lim_{x \to 0} \frac{\cos(x)-1}{x^2}$.
\begin{solution}

Since $\frac{\cos(x)-1}{x^2}$ at $x = 0$
is $\frac{0}{0}$, we can use L'Hopital's rule.
So we get $\lim_{x \to 0} \frac{-\sin(x)}{2x}$.
This new limit at $x = 0$ is $\frac{0}{0}$.
Thus, we can apply L'Hopital's rule again.
So we get $\lim_{x \to 0} \frac{-\cos(x)}{2}$.
Evaluating this limit at $x = 0$ gives us
$-\frac{1}{2}$.

Therefore, $\lim_{x \to 0} \frac{\cos(x)-1}{x^2} = -\frac{1}{2}$

\end{solution}

\question \begin{parts}

\part[2] Show that $\sum_{i=1}^n i^2= \frac{n(n+1)(2n+1)}{6}$.
\begin{solution}

We will use proof by induction.

Base case:

For $n = 1$, the left side is $\sum_{i = 1}^{1} i^2 = 1^2 = 1$.
The rhs is $\frac{1(1+1)(2 \cdot 1 + 1)}{6} = \frac{1 \cdot 2 \cdot 3}{6} = 1$.

Induction case:

Assume that $\sum_{i=1}^{k} i^2 = \frac{k(k+1)(2k+1)}{6}$ is true.

Goal:
Show that $\sum_{i=1}^{k+1} i^2 = \frac{(k+1)(k+2)(2k+3)}{6}$ is true.

\[\sum_{i=1}^{k+1} i^2 = (k+1)^2 + \sum_{i=1}^{k} i^2\]
From the inductive hypothesis, we get:
\[\sum_{i=1}^{k+1} i^2 = (k+1)^2 + \frac{k(k+1)(2k+1)}{6}\]

Expanding and simplifying this gives us
$\frac{(k+1)(k+2)(2k+3)}{6}$.

Since both the base case and inductive cases hold, the proof is complete.

\end{solution}

\part[2] Show that $\sum_{i=1}^n i^3= \left(\frac{n(n+1)}{2} \right)^2$.
\begin{solution}

We will use proof by induction.

Base case:

For $n = 1$, the left side is $\sum_{i = 1}^{1} i^3 = 1^3 = 1$.
The rhs is equal to $\left(\frac{1(1+1)}{2}\right)^2 = \left(\frac{2}{2}\right)^2 = 1^2 = 1$.

Induction case:

Assume that $\sum_{i=1}^k i^3 = \left(\frac{k(k+1)}{2} \right)^2$.

Goal:
Show that $\sum_{i=1}^{k+1} i^3 = \left(\frac{(k+1)(k+2)}{2}\right)$ is true.

\[\sum_{i=1}^{k+1} i^3 = (k+1)^3 + \sum_{i=1}^{k} i^3\]
From the inductive hypothesis, we get:
\[\sum_{i=1}^{k+1} i^2 = (k+1)^2 + \left(\frac{k(k+1)}{2} \right)^2\]

Expanding and simplifying this gives us
$\left(\frac{(k+1)(k+2)}{2}\right)$.

Since both the base case and inductive cases hold, the proof is complete.

\end{solution}


\end{parts}
You can show both of these by induction, and that is how I suggest you proceed. You might also try to come up with a counting proof or a proof by picture.

\question[4] Compute, using only things from section 5.1 of your text, $\int_{0}^2 x^2-1 \dx$.

\begin{solution}

First partition the interval. Divide $[0, 2]$ into n subintervals,
each of length $\Delta x = \frac{2}{n}$.
The points of division will be $x_i = 0, \frac{2}{n}, \frac{4}{n}, \dots, 2$.

The infimum is $f(x_{i-1}) = (x_{i-1})^2 - 1$.
The supremum is $f(x_i) = (x_i)^2 - 1$.

\[L = \sum_{i = 1}^{n} ((x_{i-1})^2 - 1) \Delta x\]
\[U = \sum_{i = 1}^{n} ((x_i)^2 - 1) \Delta x\]

Substituting $x_i = \frac{2i}{n}$ and $\Delta x = \frac{2}{n}$, we get:
\[L = \sum_{i = 1}^{n} \left(\left(\frac{2(i-1)}{n}\right)-1\right)\frac{2}{n}\]
\[U = \sum_{i = 1}^{n} \left(\left(\frac{2i}{n}\right)-1\right)\frac{2}{n}\]

Evaluating either U or L as $n \to \infty$ gives us $\frac{2}{3}$.

Thus, $\int_{0}^2 x^2-1 \dx = \frac{2}{3}$.

\end{solution}

\end{questions}
\end{document}