\documentclass[answers]{exam}
\title{Foundations of Analysis, HW 2}
\author{James Cameron}
\date{}
\usepackage{amsmath}
\usepackage{amssymb}
\usepackage{graphicx}
\usepackage{amsthm}

\newtheorem{thm}{Theorem}
\newtheorem{hthm}[thm]{*Theorem}
\newtheorem{lemma}[thm]{Lemma}
\newtheorem{cor}[thm]{Corollary}
\newtheorem{obs}[thm]{Observation}
\newtheorem{prop}[thm]{Proposition}
\newtheorem{con}[thm]{Conjecture}
\newtheorem{exer}[thm]{Exercise}

\newtheorem{scho}[thm]{Scholium}
\newtheorem*{Thm}{Theorem}
\newtheorem*{Con}{Conjecture}
\newtheorem*{Axiom}{Axiom}

\theoremstyle{remark}
\newtheorem{remark}[thm]{Remark}
\newtheorem{notation}[thm]{Notation}


\theoremstyle{definition}
\newtheorem{Def}[thm]{Definition}
\newtheorem{example}[thm]{Example}
\newtheorem{ques}[thm]{Question}
\everymath{\displaystyle}

\pagestyle{head}
\header{\bfseries\large Foundations of Analysis}{\bfseries\large Homework 2 \ifprintanswers (solutions) \fi}{\bfseries\large Fall, 2023}
\headrule

\DeclareMathOperator{\sech}{sech}
\newcommand{\NN}{\mathbb{N}}
\newcommand{\RR}{\mathbb{R}}
\newcommand{\QQ}{\mathbb{Q}}
\newcommand{\ZZ}{\mathbb{Z}}
\newcommand{\dV}{\;\mathrm{d}V}
\newcommand{\dA}{\;\mathrm{d}A}
\newcommand{\dx}{\;\mathrm{d}x}
\newcommand{\dy}{\;\mathrm{d}y}
\newcommand{\dz}{\;\mathrm{d}z}
\pointname{}
\newcommand{\cA}{\mathcal{A}}
\newcommand{\Bb}{\mathcal{B}}
\newcommand{\Ww}{\mathcal{W}}
\newcommand{\Dd}{\mathcal{D}}
\newcommand{\Ss}{\mathcal{S}}
\newcommand{\Ee}{\mathcal{E}}
\DeclareMathOperator{\im}{im}
\begin{document}
% \maketitle

% \noindent
% This week on the homework you will get practice with induction. \\


% \noindent
 This is due Saturday 9/2 by 11:59 pm on Gradescope. Please either neatly write up your solutions or type them up. You can find a .tex template on Canvas. Your proofs should be written in complete sentences and paragraphs, using a combination of words and symbols. They should be \textbf{correct, clear, and concise}. You will be graded on all three, especially the first two!

\noindent


\begin{questions}





\question  Decide for which natural numbers $2^n > n^2$ is true, and prove your claim for those natural numbers.

\begin{solution}

$2^n > n^2$ on two intervals: $[0, 2)$, $(4, \infty)$.

The first interval we can prove by cases since we are using natural numbers.

$2^0 = 1; 0^2 = 0; 1 > 0$
$2^1 = 2; 1^2 = 1; 2 > 1$
They are equal at $n = 2$:
$2^2 = 4; 2^2 = 4; 4 = 4$

Now for the second interval:
They are equal at $n = 4$:
$2^4 = 16; 4^2 = 16; 16 = 16$,
so we start at $5$ as the base case in our proof.
We will prove this by induction:

Base case:
$2^5 = 32; 5^2 = 25; 32 > 25$.

Inductive case:
Assume $2^n > n^2$ is true.
\[2^{n+1} > (n+1)^2
\implies 2 \cdot 2^n > n^2 + 2n + 1\]
\[\implies 2^n + 2^n > n^2 + 2n + 1
\implies 2^n > n^2 - 2^n + 2n + 1\]
Since we know by assumption that $2^n > n^2$,
that means we can knock out the $n^2$ on the rhs and get:
\[2^n > 2n + 1\]
This is obviously true since an exponential function grows drastically faster than a linear function.
\[\therefore 2^n > n^2 \implies 2^{n+1} > (n+1)^2\]

So, $2^n > n^2$. $\qed$

\end{solution}

\question Show that for every $n \in \NN$ there exists $a_i \in \{0,1\}$ and $k \in \NN$ such that $n= \sum_{i=0}^k a_i 2^i$. For example, $3=1\cdot 2^1+ 1 \cdot 2^0$, so $k=1$ and $a_0=a_1=1$ for $n=3$. \textbf{Hint:} Use strong induction

\begin{solution}

So, we have to prove that a base 2 system can span all the natural numbers.
Let's start with the base case:

\[1 = 2^0\]

Now, for the (strong) inductive case:

Assume that $\forall n \in \NN, 1 \geq m \geq n,$ m can be written in terms of powers of 2.

Now, let's start with a few cases.
If $n + 1$ is a power of 2, then it can be written using a new
power of 2 that hasn't appeared yet.

If $n + 1$ is odd, then, we just add $2^0$ onto the end of the sum for $n$.

If $n + 1$ is even, and not a power of 2, then things are more interesting.
In this case, $n$ must be odd. So we can't simply add $2^0$ to the end since
it is already present. Instead, we will remove the $2^0$ and replace it with $2^1$.
If $2^1$ already exists, we remove it and replace it with $2^2$. We repeat this process
until we find a power of 2 not in the list. It follows that
$(2^1 + 2^2 + ... + 2^i) + 1$ equals $2^{i+1}$ from the above case where
$n+1$ is a power of 2. But since $n+1$
is not a power of 2 in this case,
it is the same except that $i < n$ and that there are terms after it.
In other words, we can write the current number as
$2^i + ... + 2^j$ where $2^j$ was the last term for $n$ in base 2 form.

$\therefore$ We can write any natural number in terms of base 2. $\qed$.

\end{solution}

\question The \emph{well ordering principle} states that every nonempty subset of natural numbers has a smallest element. The well ordering principle follows from and is in fact equivalent to induction and strong induction.

Use the well ordering principle to show that every nonzero $x \in \QQ$ can be written in the form $x = \frac{p}{q}$, where $p,q \in \ZZ$ and $p,q$ have no prime factors in common.

\begin{solution}

First of all, for the rest of this proof, $x$ is not $0$.

If $x$ were $0$, then you can write it in an infinite number of reduced forms
(where the numerator is $0$ and the denominator is any number except for $0$).

Let's handle this in two cases. Coprime means that the
greatest common divider of $p$ and $q$ is 1.

In the case where $p$ and $q$ are already coprime, then $x$ is in reduced form.

Now, for the case where $p$ and $q$ are not coprime, we need to show that
$x$ can be written in terms of $p'$ and $q'$ where
$p'$ and $q'$ are coprime.

Well, let's start by taking the $gcd$ of $p$ and $q$.
Since that number is present in both $p$ and $q$,
we can factor it out and write $x$ as:

$x = \frac{pq_gcd}{pq_gcd} \cdot \frac{p_1}{q_1}$.
Since any number divided by itself (excluding $0$) is $1$.
Any number multiplied by $1$ is itself.

That means that we can rewrite the above as;
$x = 1 \cdot \frac{p_1}{q_1} = \frac{p_1}{q_1}$

Then, we keep on repeating this process, but we put $p_1$ and $q_1$ in for $p$ and $q$
respectively and keep going until they have a $gcd$ of 1.
This is guarenteed to terminate because the list of factors for $p$ and $q$ shrinks at
every step and the base case is $1$.

Eventually we will end up with two coprime numbers by repeatedly eliminating
their $gcd$ repeatedly until we reach $1$.

$\therefore$ We can always rewrite $x$ to be a ratio of two coprime integers. $\qed$


\end{solution}

\question Show that an ordered field $F$ is Archimedean if and only if for all $x,y \in F$ with $x >0$ there is an $n \in \NN$ so that $nx>y$.
\begin{solution}

Let $n = \frac{y+1}{x}$.

Then we get
$nx > y \implies y+1 > y$.

$\qed$.

\end{solution}

\question Show that the set of rational numbers $S=\{ x \in \QQ: x^2  \le 2 \}$ is a bounded above set that does not have a least upper bound (among the rational numbers). \textbf{Hint:} Proceed by contradiction. We showed in class that there is no $x \in \QQ$ with $x^2=2$. So, if a least upper bound $l$ did exist we know $l^2 \not=2$. If $l^2<2$, find an $s \in S$ with $l<s$, contradicting that $l$ is an upper bound. If $l^2>2$, find some $b \in \QQ$ that is an upper bound for $S$ and with $b<l$, contradicting that $l$ is the \emph{least} upper bound.

\begin{solution}

In the case of picking a rational number $L$ whose square is less than 2,
the issue is that since it is an open interval (since $\sqrt{2}$ is not in the rationals),
there are an infinite number of rational nunbers between any two rational numbers.
So you could always just pick
any of these rational numbers that fall between $L$ and $\sqrt{2}$.

In the case of picking a rational number $T$ greater than 2,
the same issue persists. You could always just pick any rational number that falls
between $\sqrt{2}$ and $T$ since it forms an open interval and the rationals aren't discrete.

$\therefore$ You can never pick any rational number that both bounds all
of the members of the set and is less than or equal to all other possible bounds of the set.

\end{solution}

\end{questions}
\end{document}