\documentclass{article}

\usepackage{amsfonts}
\usepackage{graphicx}
\usepackage{amssymb}
\usepackage{amsmath}
\usepackage{listings}
\usepackage{hyperref}


\DeclareMathOperator{\sech}{sech}
\newcommand{\NN}{\mathbb{N}}
\newcommand{\RR}{\mathbb{R}}
\newcommand{\QQ}{\mathbb{Q}}
\newcommand{\ZZ}{\mathbb{Z}}
\newcommand{\dV}{\;\mathrm{d}V}
\newcommand{\dA}{\;\mathrm{d}A}
\newcommand{\dx}{\;\mathrm{d}x}
\newcommand{\dy}{\;\mathrm{d}y}
\newcommand{\dz}{\;\mathrm{d}z}
\newcommand{\cA}{\mathcal{A}}
\newcommand{\Bb}{\mathcal{B}}
\newcommand{\Ww}{\mathcal{W}}
\newcommand{\Dd}{\mathcal{D}}
\newcommand{\Ss}{\mathcal{S}}
\newcommand{\Ee}{\mathcal{E}}
\DeclareMathOperator{\im}{im}

\tolerance=1
\emergencystretch=\maxdimen
\hyphenpenalty=10000
\hbadness=10000

\setlength\parindent{18pt}

\begin{document}

\Large{Lincoln Sand}


\textbf{M1.}

\textbf{579.9}: Find the relationship of the fluxions using Newton's
rules for the equation $y^2 - a^2 - x \sqrt{a^2 - x^2} = 0$.
Put $z = x \sqrt{a^2 - x^2}$.

First, let's differentiate the equation implicitly with respect to time:

Given the equation $y^2 - a^2 - x \sqrt{a^2 - x^2} = 0$,
we differentiate implicitly $2 y \dot{y} - \frac{d}{dt}(x \sqrt{a^2 - x^2}) = 0$.

Now we need to differentiate $z = x \sqrt{a^2 - x^2}$:

we apply the product rule $\dot{z} = \dot{x} \sqrt{a^2 - x^2} + x \frac{d}{dt}(\sqrt{a^2 - x^2})$.

Using the chain rule on $\sqrt{a^2 - x^2}$, we get $- \frac{x \dot{x}}{\sqrt{a^2 - x^2}}$.

Substituting back in, we get $\dot{z} = \dot{x} \sqrt{a^2 - x^2} \left(1 - \frac{x^2}{a^2 - x^2}\right)$.

Simplifying this gives $\dot{z} = \dot{x} \sqrt{a^2 - x^2} \frac{a^2 - 2x^2}{a^2 - x^2}$.

Substituting back again gives us $2 y \dot{y} - \dot{x} \sqrt{a^2 - x^2} \frac{a^2 - 2x^2}{a^2 - x^2} = 0$.

Solving for $\dot{y}$ finally yields $\dot{y} = \frac{\dot{x} \sqrt{a^2 - x^2} (a^2 - 2x^2)}{2y(a^2 - x^2)}$.


\textbf{579.24}: Given the curve $y^q = x^p$ ($q > p > 0$),
show using the transmutation theorem that
\[\int_{0}^{x_0} y dx = \frac{q x_0 y_0}{p+q}\]
Note that from $y^q = x^p$, it follows that $q dy/y = p dx/x$ and
therefore that $z = y - x dy/dx = [(q-p)/q]y$.

We know from above that $y = x^{p/q}$ and $q \frac{dy}{y} = p \frac{dx}{x}$.
This implies that $\frac{dy}{dx} = \frac{p}{q} \frac{y}{x}$.

$z$ is defined as $z = y - x \frac{dy}{dx}$. Substituting from above, we get that
$z = \frac{q-p}{q} y$.

We can rewrite this as $z = \frac{q-p}{q} x^{p/q}$.

For the integral, using the transmutation theorem, we get that
$\int_{0}^{x_0} y dx = \frac{q}{q-p} \int_{0}^{x_0} z dx = \frac{q}{q-p} \int_{0}^{x_0} x^{p/q} dx$.

This becomes $\frac{q-p}{p} \left(\frac{x^{p/q + 1}}{(p/q + 1)}\right)_{0}^{x_0} = \frac{q-p}{q} \frac{x_0^{p/q + 1}}{p/q + 1}$.

Substituting back gives us $\int_{0}^{x_0} y dx = \frac{q}{q-p} \left(\frac{q-p}{q} \frac{x_0^{p/q + 1}}{p/q + 1}\right)$.
Simplifying gives us $\frac{q x_0 y_0}{p + q}$ as expected.


\textbf{579.25}: Prove the quotient rule $d(\frac{x}{y}) = \frac{y dx - x dy}{y^2}$
by an argument using differentials.

Given $z = \frac{x}{y}$, we can write this as $z = x \cdot y^{-1}$.

Now, we apply the product rule to get $dz = d(x \cdot y^{-1}) = dx \cdot y^{-1} + x \cdot d(y^{-1})$.

We can calculate $d(y^{-1})$ using the chain rule to get $- y^{-2} \cdot dy$.

Substituting back gives us $dz = dx \cdot y^{-1} + x \cdot - y^{-2} \cdot dy = \frac{dx}{dy} - \frac{x \cdot dy}{y^2}$.

Using basic algebra to turn this into a common denominator, we get
$dz = \frac{y \cdot dx - x \cdot dy}{y^2}$. Which is the quotient rule we
wanted to prove.


\textbf{M2.} Recall Napier's logarithm $Nlog(x) = m$ if
$10^7 (1 - 10^{-7})^m = x$. Show that
\[Nlog(x) + Nlog(y) = Nlog(xy) + Nlog(1)\]

\[10^7(1 - 10^{-7})^m = x\]
\[10^7(1 - 10^{-7})^n = y\]

Multiplying these together:
\[10^7(1 - 10^{-7})^m \cdot 10^7(1 - 10^{-7})^n = 10^7 \cdot 10^7(1 - 10^{-7})^{m+n}\]

We can write $10^7 \cdot 10^7$ as $10^14$, but this means we have to divide by $10^7$
to match Napier's logarithm. So, we have $10^7(1 - 10^{-7})^{m+n}$.

Now, if we take Napier's logarithm of both sides, we get $Nlog(xy) = m + n$.

Now, we have to handle $Nlog(1)$.

For $x=1$, we have $10^7(1-10^{-7})^m = 1$.
The only power of any number that will equal $1$ is $0$.
So, $m = 0$.
Thus, $Nlog(1) = 0$.

Substituting back in gives us:
\[Nlog(x) + Nlog(y) = m + n\]
\[Nlog(xy) + Nlog(1) = m + n + 0\]
Thus, the relation:
\[Nlog(x) + Nlog(y) = Nlog(xy) + Nlog(1)\]
holds for Napier's logarithm.


\textbf{M3.} Show that the binomial series gives
\[\frac{1}{\sqrt{1-t^2}} = 1 + \frac{1}{2} t^2 + \frac{1 \cdot 3}{2 \cdot 4} t^4 + \frac{1 \cdot 3 \cdot 5}{2 \cdot 4 \cdot 6} t^6 + \dots\]
Then use
\[\sin^{-1}(x) = \int_{0}^{x} \frac{dt}{\sqrt{1 - t^2}}\]
to derive Newton's series for $\sin^{-1}(x)$.

Part 1: Binomal series for $\frac{1}{\sqrt{1-t^2}}$

First, let's expand it using the binomal series expansion for a power
of $(1-x)$. The binomal series expansion for $(1-x)^n$ is given by:
\[(1-x)^n = \sum_{i=0}^{\infty} \binom{n}{i} (-x)^i\]
where $\binom{n}{i} = \frac{n(n-1)\dots(n-i+1)}{i!}$.

For $n = -\frac{1}{2}$ and $x = t^2$, we get:
\[\sum_{i=0}^{\infty} \binom{-\frac{1}{2}}{i} (-1)^i t^{2i}\]

The binomial coefficient $\binom{-\frac{1}{2}}{i}$ simplifies to:
\[\frac{(-1)^i (1 \cdot 3 \cdot 5 \dots (2i-1))}{2^i \cdot i!}\]

Notice that $\frac{(-1)^i (1 \cdot 3 \cdot 5 \dots (2i-1))}{2^i \cdot i!}$ simplifies to:
\[\frac{(-1)^i (1 \cdot 3 \cdot 5 \dots (2i-1))}{2 \cdot 4 \cdot 6 \dots (2n)}\]

This means we have:
\[(1-t^2)^{-1/2} = \sum_{i=0}^{\infty} \frac{(-1)^i (1 \cdot 3 \cdot 5 \dots (2i-1))}{2 \cdot 4 \cdot 6 \dots (2n)} t^{2n}\]

We can write this as:
\[\frac{1}{\sqrt{1-t^2}} = \sum_{i=0}^{\infty} \frac{(-1)^i (1 \cdot 3 \cdot 5 \dots (2i-1))}{2 \cdot 4 \cdot 6 \dots (2n)} t^{2n}\]

Part 2: Newton's Series for $\sin^{-1}(x)$

Given $\sin^{-1}(x) = \int_{0}^x \frac{dt}{\sqrt{1 - t^2}}$,
substituting the expansion from part 1 gives:
\[\sin^{-1}(x) = \int_{0}^x (1 + \frac{1}{2}t^2 + \frac{1 \cdot 3}{2 \cdot 4} t^4 + \frac{1 \cdot 3 \cdot 5}{2 \cdot 4 \cdot 6} t^6 + \dots) dt\]

Integrating by terms yields:
\[\sin^{-1}(x) = \left(t + \frac{1}{2} \frac{t^3}{3} + \frac{1 \cdot 3}{2 \cdot 4} \frac{t^5}{5} + \frac{1 \cdot 3 \cdot 5}{2 \cdot 4 \cdot 6} \frac{t^7}{7} + \dots\right)_{0}^x\]

This yields:
\[x + \frac{1}{2} \frac{x^3}{3} + \frac{1 \cdot 3}{2 \cdot 4} \frac{x^5}{5} + \frac{1 \cdot 3 \cdot 5}{2 \cdot 4 \cdot 6} \frac{x^7}{7} + \dots\]

This is Newton's series for the arcsine function.


\textbf{M4.} Use Fermat's method of ad-equation to find the slope
of the curve $f(x) = x^2 - \sqrt{x}$ at $x > 0$.

Let's increment $x$ by a very small value $e$:
\[f(x+e) = (x+e)^2 - \sqrt{x+e}\]

\[(x+e)^2 = x^2 + 2xe + e^2\]
For $\sqrt{x+e}$, using the first terms of the Taylor series, we get:
\[\sqrt{x+e} \approx \sqrt{x} + \frac{1}{2 \sqrt{x}} e\]

Substituting back gives:
\[f(x+e) \approx x^2 + 2xe + e^2 - \sqrt{x} - \frac{1}{2 \sqrt{x}} e\]

Now we subtract $f(x)$ and factor out $e$:
\[f(x+e) - f(x) = 2xe + e^2 - \frac{1}{2 \sqrt{x}} e = e\left(2x + e - \frac{1}{\sqrt{x}}\right)\]

Now we apply adequality. Using adequality, we simplify it to:
\[e(2x - \frac{1}{\sqrt{x}})\]

Then we divide out the $e$ to finally get:
\[f'(x) = 2x - \frac{1}{2 \sqrt{x}}\]


\textbf{M5.} Use Newton's version of Newton's method to approximate the
root of $x^2 - 2 = 0$ to an accuracy of eight decimal places.

Let's first list the steps involved in Newton's version of Newton's method
(according to the class notes):

1. Take the current approximation $x_i$.

2. Consider a small change $p$ such that $x_{i+1} = x_i + p$.

3. Substitute $x_{i+i}$ into the equation $x^2 - 2 = 0$ and ignore the
higher-order terms of $p$.

4. Solve for $p$ and update the approximation for $x_{i+1}$.

5. Repeat until $|p| < 10^{-8}$.

For $f(x) = x^2 - 2$, the linearized equation around $x_i$ is:
\[(x_i - p)^2 - 2 = x_i^2 - 2x_i p + p^2 - 2\]
Since $p$ is small, we ignore $p^2$ and simplify it to:
\[x_i^2 - 2x_i p - 2 = 0\]

Now, let's solve for $p$.

\[2x_i p = 2 x_i^2 \implies p = \frac{2 - x_i^2}{2x_i}\]

Now, we are trying to approximate $\sqrt{2}$. So let's pick $x_0 = 1.4$
and start iterating.

Iteration 1:
\[x_0 = 1.4\]
\[p_1 = \frac{2 - 1.4^2}{2 \cdot 1.4} \approx 0.014285714285714379\]
\[x_1 = x_0 + p_1 \approx 1.414285714285714379\]

Iteration 2:
\[p_2 = \frac{2 - 1.414285714285714379^2}{2 \cdot 1.414285714285714379} \approx -0.00007215007215011227\]
\[x_2 = x_1 + p_2 \approx 1.4142135642135643\]

Iteration 3:
\[p_3 = \frac{2 - 1.4142135642135643^2}{2 \cdot 1.4142135642135643} \approx -1.8404691290714918 \cdot 10^{-9}\]
\[x_3 = x_2 + p_3 \approx 1.4142135623730951\]

Since $p_3 < 10^{-8}$, $x_3$ is out final approximation of the root ($\sqrt{2}$).

Note: I used a calculator and had to manually type the numbers above, so if there
are any typos/mistakes, I apologize.

\newpage

\textbf{M6.} Essay on Modern Mathematics Proposal

Lincoln Sand

Working title: Infinity and Sets: George Cantor's Controversial Set theory

Essay topic description: I want to write about the birth of set theory by George Cantor
and its controversy.

Interesting fact: Cantor came up with the idea of different "sizes"
of infinity with his famous diagonal argument
(to prove there were more reals than rationals).

Style manual I will use:

MLA

Two internet references:

1. Ferreirós, J. (2020, June 18). The early development of set theory. Stanford Encyclopedia of Philosophy. https://plato.stanford.edu/entries/settheory-early/ 

2. Set theory from Cantor to Cohen. (n.d.). https://booksite.elsevier.com/samplechapters/9780444516213/sample.pdf

Two journal/book references:

1. Zenkin, Alexander (2004), "Logic Of Actual Infinity And G. Cantor's Diagonal Proof Of The Uncountability Of The Continuum", The Review of Modern Logic, vol. 9, no. 30, pp. 27-80

2. https://www.math.uwaterloo.ca/~xzliu/cantor-set.pdf

(will figure out how to MLA citate this second source properly later)


\end{document}
