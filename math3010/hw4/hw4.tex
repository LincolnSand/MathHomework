\documentclass{article}

\usepackage{amsfonts}
\usepackage{graphicx}
\usepackage{amssymb}
\usepackage{amsmath}
\usepackage{listings}
\usepackage{hyperref}


\DeclareMathOperator{\sech}{sech}
\newcommand{\NN}{\mathbb{N}}
\newcommand{\RR}{\mathbb{R}}
\newcommand{\QQ}{\mathbb{Q}}
\newcommand{\ZZ}{\mathbb{Z}}
\newcommand{\dV}{\;\mathrm{d}V}
\newcommand{\dA}{\;\mathrm{d}A}
\newcommand{\dx}{\;\mathrm{d}x}
\newcommand{\dy}{\;\mathrm{d}y}
\newcommand{\dz}{\;\mathrm{d}z}
\newcommand{\cA}{\mathcal{A}}
\newcommand{\Bb}{\mathcal{B}}
\newcommand{\Ww}{\mathcal{W}}
\newcommand{\Dd}{\mathcal{D}}
\newcommand{\Ss}{\mathcal{S}}
\newcommand{\Ee}{\mathcal{E}}
\DeclareMathOperator{\im}{im}

\tolerance=1
\emergencystretch=\maxdimen
\hyphenpenalty=10000
\hbadness=10000

\setlength\parindent{18pt}

\begin{document}

\textbf{D1:}

\textbf{48.21}: In Zeno's Achilles paradox, assume the quick runner
Achilles is racing against a tortoise. Assume further that the
tortoise has a 500-yard head start but that Achilles speed is
fifty times that of the tortoise. Finally, assume that the
tortoise moves 1 yard in 5 seconds. Determine the time t
it will take until Achilles overtakes the tortoise and
the distance d he will have traveled. Note that Achilles
must first travel 500 yards to reach the point where the
tortoise started. This will take 50 seconds. But in that
time the tortoise will move 10 yards further. Continue
this analysis by writing down the sequence of distances
that Achilles must travel to reach the point the tortoise
had already been. Show that the sum of this infinite sequence
of distances is equal to the distance d calculated first.

$\text{Speed of the tortoise} = \frac{1}{5}$

$\text{Speed of Achilles} = 50 \times \frac{1}{5} = 10$

$\text{Tortoise equation} = \frac{1}{5} t + 500$

$\text{Achilles equation} = 10 t$

$\frac{1}{5} t + 500 = 10 t \implies t + 2500 = 50 t \implies 2500 = 49 t \implies t = \frac{2500}{49} \approx 51.020$


For the sum:

1. 500 yards

2. 10 yards

3. 0.2 yards

4. 0.004 yards

5. $8 \times 10^{-5}$ yards

6. $1.6 \times 10^{-6}$ yards

7. $3.2 \times 10^{-8}$ yards

8. $6.4 \times 10^{-10}$ yards

9. $1.28 \times 10^{-11}$ yards

10. $2.56 \times 10^{-13}$ yards

11. $5.12 \times 10^{-15}$ yards

The sum of these distances is approximately $510.204$ yards, which is pretty much equal
to the distance d. If we add infinitely many terms, they will converge to the same value.


\textbf{D2:} In Book VI of Elements, Euclid gives the following
argument for the Pythagorean Theorem based on similar triangles.
Show that the three triangles in the figure are similar, and
hence prove the Pythagorean Theorem by equating ratios to
corresponding sides. [Stillwell Mathematics and its History. p. 10]

Let's first demonstrate similarity.

All three triangles share the same right angle.

The small triangle on the left shares angle $A$ with the large triangle, and the
small triangle on the right shares angle $B$ with the large triangle.

Since the angles in a triangle must sum to 180 degrees, the third angle in the smaller
triangles must be equal to each other.

$\frac{a}{c} = \frac{c_1}{a}$

$a^2 = c_1 \cdot c$

$\frac{b}{c} = \frac{c_2}{b}$
$b^2 = c_2 \cdot c$

Since $c = c_1 + c_2$,

$a^2 + b^2 = c_1 \cdot c + c_2 \cdot c$

$\implies a^2 + b^2 = c \cdot (c_1 + c_2)$

$\implies a^2 + b^2 = c^2$


\textbf{D3:} Show that the Golden ratio $\phi$ is irrational.

$\phi = \frac{1 + \sqrt{5}}{2}$

Let's prove this by contradiction.

Assume that $\phi$ is rational and expressed in lowest terms as $\frac{a}{b}$.

$\phi = \frac{1 + \sqrt{5}}{2} = \frac{a}{b}$

$\implies 2a = b + b \sqrt{5}$

$\implies 2a - b = b \sqrt{5}$

$\implies (2a - b)^2 = (b \sqrt{5})^2$

$\implies 4 a^2 - 4 a b + b^2 = 5 b^2$

$\implies 4 a^2 - 4 a b - b^2 = 0$

$\implies a(4a - 4b) = b^2$

Since $a$ and $b$ are integers, then $4a - 4b$ is also an integer.
Let's call it $k$ such that $a = \frac{b^2}{k}$.

Now, $k$ must perfectly divide $b^2$ since $a$ is an integer. And since $k = 4a - 4b$,
it must also divide $4b$ perfectly. But that means that both $a$ and $b$
share $k$ in common and thus we have a contradiction that $\frac{a}{b}$ is
in lowest terms. Thus, $\phi$ is irrational.


\textbf{D4:} Check Euclid's construction of the regular pentagon.
Suppose the collinear points A, B, C have distances
$a = AB, \phi a = AC$. Construct the isosceles
triangle $\triangle (BDC)$ with $a = BD = CD$.
Let M be the bisector between B and C.
Let the angle $\alpha = \angle (ACD)$.

\textbf{a.} Find the length of $DM$ using the triangle $\triangle (BMD)$.

Since $M$ is the midpoint of $BC$, we know that $BM = MC = \frac{a}{2}$.
Now, $\triangle BMD$ is a right triangle.

Using the Pythagorean Theorem:

$DM = \sqrt{BD^2 + BM^2} = \sqrt{a^2 - \left(\frac{a}{2}\right)^2}$

\textbf{b.} Show that $AD = AC$ using the triangle $\triangle (AMD)$.

$\triangle AMD$ is an isosceles triangle with $AM = MC$ because $M$ is the
midpoint of $AC$ and $AC = \phi a$. Hence, $AM = MC = \frac{\phi a}{2}$.
We know $DM$ from part a. Now, using the fact that $\triangle AMD$ is isosceles
with $AM = MD$, it follows that $AD = AC$.

\textbf{c.} Express the angles of triangle $\triangle (ABD)$ in
terms of $\alpha$ and show that $\alpha = 72^{\circ}$. Thus
it is the central angle of a sector of the regular pentagon.
Hint: The sum of the interior angles of any triangle is $180^{\circ}$.

\newpage

\textbf{D5:} Essay on Mathematics of Antiquity proposal

Working title: Large Numbers and The Sand Reckoner

Essay topic description:

I want to write an essay about the invention of the Sand Reckoner
by Archimedes and how he came up with it and why it works.

Interesting fact:

Archimedes ended up discovering and proving the law of exponents in the process.

Style manual I will use:

MLA

Two internet references:

1.

https://web.calstatela.edu/faculty/hmendel/Ancient
\%20Mathematics/Archimedes/SandReckoner/SandReckoner.html

2.

https://web.calstatela.edu/faculty/hmendel/Ancient
\%20Mathematics/Archimedes/SandReckoner/Ch.1/Ch1.html

3. (bonus)

https://www.famousscientists.org/how-archimedes-invented-the-beast-number/

1. Eureka Man: The Life and Legacy of Archimedes

It covers the Sand Reckoner in Chapter 4.

2. The Sand Reckoner translated by Thomas L Heath

\end{document}
