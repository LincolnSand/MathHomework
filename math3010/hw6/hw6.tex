\documentclass{article}

\usepackage{amsfonts}
\usepackage{graphicx}
\usepackage{amssymb}
\usepackage{amsmath}
\usepackage{listings}
\usepackage{hyperref}


\DeclareMathOperator{\sech}{sech}
\newcommand{\NN}{\mathbb{N}}
\newcommand{\RR}{\mathbb{R}}
\newcommand{\QQ}{\mathbb{Q}}
\newcommand{\ZZ}{\mathbb{Z}}
\newcommand{\dV}{\;\mathrm{d}V}
\newcommand{\dA}{\;\mathrm{d}A}
\newcommand{\dx}{\;\mathrm{d}x}
\newcommand{\dy}{\;\mathrm{d}y}
\newcommand{\dz}{\;\mathrm{d}z}
\newcommand{\cA}{\mathcal{A}}
\newcommand{\Bb}{\mathcal{B}}
\newcommand{\Ww}{\mathcal{W}}
\newcommand{\Dd}{\mathcal{D}}
\newcommand{\Ss}{\mathcal{S}}
\newcommand{\Ee}{\mathcal{E}}
\DeclareMathOperator{\im}{im}

\tolerance=1
\emergencystretch=\maxdimen
\hyphenpenalty=10000
\hbadness=10000

\setlength\parindent{18pt}

\begin{document}

\Large{Lincoln Sand}

\normalsize{}

\textbf{F1.}

\textbf{90.7}:

Let us denote the following:

EFGH as the rectangle on the left side.

ABEF as the rectangle on the right side.

ABML as the figure on the bottom right side.


Let us note the following relations of areas:

Area of GDFH = Area of EFGH + Area of ABEF + Area of BEFG

Area of BEFG = Area of GDFH - (Area of EFGH + Area of ABEF)

Area of ABML = Area of GDFH - Area of BEFG


Given the fact that the area of BEFG is calculated by subtracting the areas of
EFGH and ABEF from the areas of GDFH, and the area of ABML is the remaining part
of GDFH after subtracting BEFG, we get that:

Area of BEFG = Area of ABML


\textbf{90.10}:

Let us denote the following:

The whole line as c.

One segment as a.

The other segment as b.

i.e. c = a + b

The proposition gives us that:

$4 \cdot (a + b) \cdot a + b^2 = (a + c)^2$

Using the above substitution of $c = a + b$ yields:

$4 \cdot (a + b) \cdot a + b^2 = (2a + b)^2$

Expanding the square gives:

$4 \cdot (a + b) \cdot a + b^2 = 4a^2 + 4ab + b^2$

Distributing the $4a$ gives:

$4a^2 + 4ab + b^2 = 4a^2 + 4ab + b^2$

I know I should also generate a diagram, but I didn't like how it turned out.

\textbf{90.19}:

963 and 657:

\[963 = 657 \cdot 1 + 306\]
\[657 = 306 \cdot 2 + 45\]
\[306 = 45 \cdot 6 + 36\]
\[45 = 36 \cdot 1 + 9\]
\[36 = 9 \cdot 4 + 0\]

So, the answer is 9.

2689 and 4001:

\[4001 = 2689 \cdot 1 + 1312\]
\[2689 = 1312 \cdot 2 + 65\]
\[1312 = 65 \cdot 20 + 12\]
\[65 = 12 \cdot 5 + 5\]
\[12 = 5 \cdot 2 + 2\]
\[5 = 2 \cdot 2 + 1\]
\[2 = 1 \cdot 2 + 0\]

So, the answer is 1.

\textbf{F2.}

For the Euclidean algorithm for gcd(a, b).

We get:

\[a = b q_1 + r_1\]
\[b = r_1 q_2 + r_2\]
\[r_1 = r_2 q_3 + r_3\]
\[\vdots\]
\[r_{n-2} = r_{n-1} q_n + r_n\]
\[r_{n-1} = r_n q_{n+1} + 0\]

We can work backwards and rewrite it into the form:

\[r_{n-1} = r_n q_{n+1}\]
\[r_n = r_{n-2} - r_{n-1} q_n\]
\[\vdots\]
\[r_2 = b - r_1 q_2\]
\[r_1 = a - b q_1\]

Through substitution, we arrive at being able to express $r_n$ in terms of $m a + n b$
where $m$ and $n$ are some integers.

\textbf{F3.}

We know from above that $gcd(a, b) = m a + n b$.

For $a x + b y = c$ with integer coefficients:

If $gcd(a, b)$ divides c, then $\exists k \in \ZZ$ such that:

$c = k \cdot gcd(a, b)$

Substituting in what we know from above into this equation gives:

$c = k(m a + n b)$

$c = (k m) a + (k n) b$

This shows that $x = k m$ and $y = k n$ are integer solutions to the equation
$a x + b y = c$.

Therefore, $a x + b y = c$ has an integer solution if and only if gcd(a, b) divides c.

\textbf{F4.}

$gcd(12, 15) = 3$

From above, we know that gcd(a, b) must divide c.

But 3 does not divide 1.

Therefore, the equation has no integer solution.

\end{document}
