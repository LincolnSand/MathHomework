\documentclass{article}

\usepackage{amsfonts}
\usepackage{graphicx}
\usepackage{amssymb}
\usepackage{amsmath}
\usepackage{listings}
\usepackage{hyperref}


\DeclareMathOperator{\sech}{sech}
\newcommand{\NN}{\mathbb{N}}
\newcommand{\RR}{\mathbb{R}}
\newcommand{\QQ}{\mathbb{Q}}
\newcommand{\ZZ}{\mathbb{Z}}
\newcommand{\dV}{\;\mathrm{d}V}
\newcommand{\dA}{\;\mathrm{d}A}
\newcommand{\dx}{\;\mathrm{d}x}
\newcommand{\dy}{\;\mathrm{d}y}
\newcommand{\dz}{\;\mathrm{d}z}
\newcommand{\cA}{\mathcal{A}}
\newcommand{\Bb}{\mathcal{B}}
\newcommand{\Ww}{\mathcal{W}}
\newcommand{\Dd}{\mathcal{D}}
\newcommand{\Ss}{\mathcal{S}}
\newcommand{\Ee}{\mathcal{E}}
\DeclareMathOperator{\im}{im}

\tolerance=1
\emergencystretch=\maxdimen
\hyphenpenalty=10000
\hbadness=10000

\setlength\parindent{18pt}

\begin{document}

\Large{Lincoln Sand}


\textbf{K1.}

\textbf{263.23}:

The following are the Brahmagupta to find the next solution:
\[x_2 = \frac{x_1 \cdot 9 + 1 \cdot y_1}{2}\]
\[y_2 = \frac{y_1 \cdot 9 + 83 \cdot x_1 \cdot 1}{2}\]

Plugging in $(x_1, y_1) = (1, 9)$ gives $(9, 82)$.

Check answer:
\[83 \cdot 9^2 + 1 = 6724\]
\[82^2 = 6724\]


\textbf{263.24}:

Let's plug:
\[u_1 = \frac{1}{2} uv(v^2 + 1)(v^2 + 3)\]
\[v_1 = (v^2 + 2)\left(\frac{1}{2}(v^2 + 1)(v^2 + 3) - 1\right)\]

into $Dx^2 + 1 = y^2$.

Now, let's expand $D u_1^2 + 1 = v_1^2$.

This gives us:
\[D\left(\frac{1}{2} uv(v^2+1)(v^2+3)\right)^2 = \left(\left(v^2+2\right)\left(\frac{1}{2}(v^2+1)(v^2+3)-1\right)\right)^2\]

With algebraic manipulation, these probably cancel out, but I don't want to
fill this with pages of nasty algebra.

For the integer part, note that $u_1$ and $v_1$ are constructed using addition
and multiplication on $u$ and $v$. And the $\frac{1}{2}$ cancels because of the
even terms in the products. Thus, $u_1$ and $v_1$ are integers if $u$ and $v$
are, regardless of parity.


\textbf{263.25}:

We get the following formulas:
\[u_1 = \frac{1}{2} uv(v^2 + 1)(v^2 + 3)\]
\[v_1 = (v^2 + 2)\left(\frac{1}{2}(v^2 + 1)(v^2 + 3) - 1\right)\]

Substituting in $(u, v) = (1, 3)$ gives us $(180, 649)$.

Check answer:
\[13 \cdot 180^2 + 1 = 421201\]
\[649^2 = 421201\]


\textbf{318.1}:

$8023 \cdot 8 = 64184$

$8023 \cdot 3 = 24069$, shifted one place to the left: $240690$

$8023 \cdot 6 = 48138$, shifted two places to the left: $4813800$

$8023 \cdot 4 = 32092$, shifted three places to the left: $32092000$

$64184 + 240690 + 4813800 + 32092000 = 37210674$

Checking answer: $8023 \cdot 4638 = 37210674$.


\textbf{318.11}:

Base case:

\[1^3 = 1^2 = 1\]

Inductive case:

Assume:

\[\sum_{i=1}^k i^3 = \left(\sum_{i=1}^k i\right)^2\]

Want to show:

\[\sum_{i=1}^{k+1} i^3 = \left(\sum_{i=1}^{k+1} i\right)^2\]

The left hand side gives:
\[\left(\sum_{i=1}^k i\right)^2 + (k+1)^3\]

The right hand side gives:
\[\left(\frac{(k+1)(k+2)}{2}\right)\]

So, we have:
\[\left(\sum_{i=1}^k i\right)^2 + (k+1)^3 = \left(\frac{(k+1)(k+2)}{2}\right)\]

This reduces to:
\[\left(\sum_{i=1}^k i\right)^2 = \left(\frac{k(k+1)}{2}\right)\]

which is true by the assumption.

Comparison with Al-Karaji's Proof:

Al-Karaji didn't use induction. Instead he relied on geometric properties and
represented numbers using dots to represent cubes and squares.


\textbf{K2.}

Bhaskara II's method is analogue to the quadratic formula.
By plugging in given constants, we get that:
\[x = \pm \frac{\sqrt{83y^2 - 83}}{83}\]

This gives some specific integer solutions of $(-9, 82)$, $(0, 1)$, and $(9, 82)$.


\textbf{K3.}

The formula is:
\[f(x) = f(a) + (x-a)\frac{\nabla f(a)}{\nabla x} + \frac{(x-a)(x-a-\nabla x)}{2!} \frac{\nabla^2 f(a)}{(\nabla x)^2}\]

where $f(x)$ is the function we want to interpolate, $f(a)$ is the function
value at the nearest known point below $x$, $\nabla f(a)$ is the first difference
at $a$, and $\nabla^2 f(a)$ is the second difference at $a$.

Looking at the table, the closest known angles to $100^\circ$ are $90^\circ$,
(which corresponds to $900$ minutes) and $112.5^\circ$
(which corresponds to $1125$ minutes).

Since $100^\circ$ is $1000$ minutes, we will interpolate between
$900$ minutes and $1125$ minutes. The known sine values are $890$ for $900$
minutes and $1105$ for $1125$ minutes. The first difference is the difference
between $1125$ and $900$ minutes. The second difference is the difference
of the first differences.

We calculate the second difference by looking at the pattern in the
"Sine Difference" column.

Since the radius RR is given as $3438$,
the values in the "Sine" column are actual sine values
multiplied by the radius.

The estimated value for $\sin(100^\circ)$ works out to
approximately $986.05$. Dividing by $R$ ($3438$) gives us
$0.2868$.


\end{document}
