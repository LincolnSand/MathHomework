\documentclass{article}

\usepackage{amsfonts}
\usepackage{graphicx}
\usepackage{amssymb}
\usepackage{amsmath}
\usepackage{listings}


\DeclareMathOperator{\sech}{sech}
\newcommand{\NN}{\mathbb{N}}
\newcommand{\RR}{\mathbb{R}}
\newcommand{\QQ}{\mathbb{Q}}
\newcommand{\ZZ}{\mathbb{Z}}
\newcommand{\dV}{\;\mathrm{d}V}
\newcommand{\dA}{\;\mathrm{d}A}
\newcommand{\dx}{\;\mathrm{d}x}
\newcommand{\dy}{\;\mathrm{d}y}
\newcommand{\dz}{\;\mathrm{d}z}
\newcommand{\cA}{\mathcal{A}}
\newcommand{\Bb}{\mathcal{B}}
\newcommand{\Ww}{\mathcal{W}}
\newcommand{\Dd}{\mathcal{D}}
\newcommand{\Ss}{\mathcal{S}}
\newcommand{\Ee}{\mathcal{E}}
\DeclareMathOperator{\im}{im}


\setlength\parindent{18pt}

\begin{document}

\textbf{C1:}

\textbf{28.10}:

\[x + \frac{2}{3}x = \frac{5}{3}x\]
\[\frac{5}{3}x - \frac{1}{3}\left(\frac{4}{3}x\right) = 10\]
\[\implies \frac{10}{9}x = 10\]
\[\implies x = 9\]


\textbf{47.3}:

\[\frac{200}{9} = 22 + \frac{2}{9}\]

$\frac{1}{5}$ is the largest unit fraction that is less than or equal to $\frac{2}{9}$.

\[\frac{2}{9} - \frac{1}{5} = \frac{1}{45}\]

Thus, $\frac{200}{9} = 22 + \frac{1}{5} + \frac{1}{45}$.

The greek representation is something like "22 and 1/5 and 1/45"??


\textbf{47.7}:

\[\frac{\text{Height of Stick}}{\text{Length of Stick's Shadow}} =
\frac{\text{Height of Pyramid}}{\text{Length of Pyramid + Length of Shadow}}\]

Let $h = \text{Height of Pyramid}$.

We then get:
\[\frac{6}{9} = \frac{h}{756 + 342}\]

This works out to $h = 732 \text{ feet}$.

This measurement is most accurate at local noon when the sun is at the highest point
in the sky and directly south.


\textbf{47.8}:

A triangular number is of the form:

\[T_n = 1 + 2 + \dots + n\]

Now, write $T_n$ both forwards and backwards:

\[1 + 2 + 3 + \dots + (n-1) + n\]
\[n + (n-1) + (n-2) + \dots + 2 + 1\]

Now, sum them together.
You will note that each term becomes $n+1$. And since we have $n$ of them,
we get $n(n+1)$. But this is the same as $2 \cdot T_n$.
Thus, $T_n = \frac{n(n+1)}{2}$.

An oblong number is of the form: $n(n+1)$.

\[n(n+1) = 2 \cdot \frac{n(n+1)}{2} = 2 \cdot T_n\]

Thus, an oblong number is twice that of a triangular number.


\textbf{47.10}:

Visual representation proof:

1) Triangular Number to Square

A triangular number $T_n$ can be represented as a triangle of dots.

$8 \cdot T_n$ can be arranged as an octagon.

Adding 1 dot to it at the center turns the octagon into a square.

2) Square to Triangular Number

An odd square can be Visualized as a square grid of dots.

Removing 1 dot from the center of the square and rearranging the dots
forms an octagon made up of 8 regular triangles.

Algebraic proof:

1) Triangular Number to Square

\[T_n = \frac{n(n+1)}{2}\]

\[\implies 8 \cdot T_n = 8 \cdot \frac{n(n+1)}{2} + 1 = 4n(n+1) + 1\]

Notice that $(2n+1)^2 = 4n^2 + 4n + 1 = 4n(n+1) + 1$.

So, $8 \cdot T_n + 1$ forms a square.

2) Square to Triangular Number

Consider the odd square: $(2n + 1)^2$.

\[(2n + 1)^2 - 1 = 4n^2 + 4n = 4n(n + 1) = 8 \cdot \frac{n(n+1)}{2} = 8 \cdot T_n\]


\textbf{47.12}:

1) Using the formula $(n, \frac{n^2 - 1}{2}, \frac{n^2 + 1}{2})$ for an odd $n$:

- (3, 4, 5)

- (5, 12, 13)

- (7, 24, 25)

- (9, 40, 41)

- (11, 60, 61)

2) Using the formula $(m, \left(\frac{m}{2}\right)^2 - 1,
\left(\frac{m}{2}\right)^2 + 1)$ for any even $m$:

- (4, 3, 5)

- (6, 8, 10)

- (8, 15, 17)

- (10, 24, 26)

- (12, 35, 37)


\textbf{C2:}

1. The ancient Egyptians used the formula:

\[\text{Area} = \left(\frac{8}{9} \cdot \text{Diameter}\right)^2\]

For the given problem:

\[\text{Area} = \left(\frac{8}{9} \cdot 12\right)^2\]

2. Modern formula:

\[\text{Area} = \pi \cdot \left(\frac{\text{Diameter}}{2}\right)^2\]

For the given problem:

\[\text{Area} = \pi \cdot 6^2\]

3. The formula for percentage error:

\[\text{Percentage Error} = \left|\frac{\text{Modern Value $-$ Ancient Egyptian Value}}{Modern Value}\right| \cdot 100\%\]

The ancient Egyptian value is approximately $113.78$ square units.

The modern value is approximately $113.10$ square units.

The percentage error is approximately $0.60\%$.


\textbf{C3:}

Rectangle:

\[\frac{1}{2}(a + a) \cdot \frac{1}{2}(b + b) = a \cdot b\]

which is the correct formula.

Non-Rectanglular Parallelogram:

For a parallelogram with sides a and b and height h perpendicular to b, the actual
equation is $b \cdot h$. The ancient Egyptian formula gives
$\frac{1}{2}(a + a) \cdot \frac{1}{2}(b + b) = a \cdot b$, which overestimates
the area since $a > h$ (because of the slant of the sides).

Trapezoid:

Let's say the parallel sides are a and b and the other two sides are c and d.

The true area formula is $\frac{1}{2}(a + b) \cdot h$ where h is the height.
The ancient Egyptian formula is $\frac{1}{2}(a + b) \cdot \frac{1}{2}(c + d)$.

This does not generally equal the actual area formula since $\frac{1}{2}(c + d)$ is
not necessarily the height.

Other quadralaterals:

The ancient Egyptian formula does not generally result in correct values.
The formula assumes the shape can be approximated using a product of the average
of the opposite sides, but this property only holds true for rectangles.

Proof for non-rectangular cases:

For non-rectangular parallelograms, we know that h is always less than
a or b (whichever side is perpendicular to h). Since the ancient Egyptian formula
uses the full side lengths instead of the height, it results in an overestimate.

For trapezoids, The height h is independent of the non-parallel side lenghts c and d.
Thus, the ancient Egyptian formula, which approximates h using c and d does not
reflect the actual geometry of the trapezoid.


\end{document}
