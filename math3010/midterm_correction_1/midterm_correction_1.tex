\documentclass{article}

\usepackage{amsfonts}
\usepackage{graphicx}
\usepackage{amssymb}
\usepackage{amsmath}
\usepackage{listings}
\usepackage{hyperref}


\DeclareMathOperator{\sech}{sech}
\newcommand{\NN}{\mathbb{N}}
\newcommand{\RR}{\mathbb{R}}
\newcommand{\QQ}{\mathbb{Q}}
\newcommand{\ZZ}{\mathbb{Z}}
\newcommand{\dV}{\;\mathrm{d}V}
\newcommand{\dA}{\;\mathrm{d}A}
\newcommand{\dx}{\;\mathrm{d}x}
\newcommand{\dy}{\;\mathrm{d}y}
\newcommand{\dz}{\;\mathrm{d}z}
\newcommand{\cA}{\mathcal{A}}
\newcommand{\Bb}{\mathcal{B}}
\newcommand{\Ww}{\mathcal{W}}
\newcommand{\Dd}{\mathcal{D}}
\newcommand{\Ss}{\mathcal{S}}
\newcommand{\Ee}{\mathcal{E}}
\DeclareMathOperator{\im}{im}

\tolerance=1
\emergencystretch=\maxdimen
\hyphenpenalty=10000
\hbadness=10000

\setlength\parindent{10pt}

\begin{document}

\Large{Midterm 1 Correction}

\Large{Lincoln Sand}

\normalsize{Because I didn't have any significant errors in any of the problems
on my midterm (the largest amount of points I lost was from a
single true/false question), I decided to just do the problem I
didn't answer on the midterm.}

\textbf{Problem 2.}


\textbf{a} Use the Babylonian method and sexigesimal arthmeic to compute the quotient.
(Other methods receive zero credit.)

$1,50 \div 9 =$

We write this as $1,50 \times \frac{1}{9}$.

In sexigesimal, $\frac{1}{9} = 0;6,40$.

So, we have $1,50 \times 0;6,40$.

\Large{This gives $12,13$.}

\normalsize{\textbf{b.} Use the Egyptian method of doubling, find the product.
(Other methods receive zero credit.)}

$22 \times 2 \bar{6} =$

Doubling 22:

1: 22

2: 44

Note: It is impossible to calculate or represent $\frac{1}{6}$
using purely power of 2 fractions (without using an infinite sum).

Since $\frac{1}{6}$ cannot be computed exactly using doubling,
I will instead just calculate it directly since it is a unit fraction, but the Egyptians
likely would have instead approximated it using power of 2 unit fractions to the desired precision.

Computing it directly gives $3 + \frac{2}{3} = 3 \bar{\bar{3}}$.

\Large{The precise answer works out to $2 \times 22 + \frac{1}{3} \times 11$
= $44 + 3 \bar{\bar{3}} = 47 + \frac{2}{3} = 47 \bar{\bar{3}}$.}

\end{document}
