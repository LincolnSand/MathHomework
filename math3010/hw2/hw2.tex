\documentclass{article}

\usepackage{amsfonts}
\usepackage{graphicx}
\usepackage{amssymb}
\usepackage{amsmath}
\usepackage{listings}


\DeclareMathOperator{\sech}{sech}
\newcommand{\NN}{\mathbb{N}}
\newcommand{\RR}{\mathbb{R}}
\newcommand{\QQ}{\mathbb{Q}}
\newcommand{\ZZ}{\mathbb{Z}}
\newcommand{\dV}{\;\mathrm{d}V}
\newcommand{\dA}{\;\mathrm{d}A}
\newcommand{\dx}{\;\mathrm{d}x}
\newcommand{\dy}{\;\mathrm{d}y}
\newcommand{\dz}{\;\mathrm{d}z}
\newcommand{\cA}{\mathcal{A}}
\newcommand{\Bb}{\mathcal{B}}
\newcommand{\Ww}{\mathcal{W}}
\newcommand{\Dd}{\mathcal{D}}
\newcommand{\Ss}{\mathcal{S}}
\newcommand{\Ee}{\mathcal{E}}
\DeclareMathOperator{\im}{im}


\setlength\parindent{18pt}

\begin{document}

\textbf{B1:}

\textbf{28.2}:

Doubling 34: 34, 68, 136, 272, 544

$18 \cdot 34 = 2 \cdot 34 16 \cdot 34 = 68 + 544 = 612$.


Doubling 5: 5, 10, 20, 40, 80

80 + 10 = 90

16 + 2 = 18

So, we get that 5 divides into 93 19 times with remainder 3.


\textbf{28.4}:

$\left(2 + \frac{1}{14}\right) \cdot \left(1 + \frac{1}{2} + \frac{1}{4}\right) = \frac{29}{14} \cdot \frac{7}{4}$

Multiply 29 by 7:

Doubling 29: 29, 58, 116

116 + 58 + 29 = 203

Doubling 56: 56, 112

$3 \cdot 56 + 35 = 203$

So, $\frac{203}{56} = 3 + \frac{35}{56}$


\textbf{28.24}:

$1;24,51,10 = 1 + \frac{24}{60} + \frac{51}{60^2} + \frac{10}{60^3} \approx 1.414213$

$\sqrt{2} \approx 1.414214$

$1.414213 - 1.414214 \approx 5.99 \cdot 10^{-7}$


\textbf{28.25}:

$x_0 = 2$

$x_1 = \frac{1}{2} \left(2 + \frac{3}{2}\right) = 1.75 = 1;45$

$\frac{1}{1.75} \approx 0;34,17,8$ 

Recall:

$x_{n+1} = \frac{1}{2} \left(x_n + \frac{3}{x_n}\right)$

The approximation we get is approximately $1;37,30 \approx 1.625$


\textbf{28.28}:

We have:
\[x^2 + y^2 = 1525\]
\[y = \frac{2}{3}x + 5\]

\[x^2 + \left(\frac{2}{3}x + 5\right)^2 = 1525\]

This works out to $x = 30, y = 25$.


\textbf{28.30}:

We have:
\[\frac{2}{3}A + \frac{1}{2}B = 1100\]
\[A - B = 600\]

This works out to $A = 1200 \text{ sar}, B = 600 \text{ sar}$.


\end{document}
