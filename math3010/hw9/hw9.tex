\documentclass{article}

\usepackage{amsfonts}
\usepackage{graphicx}
\usepackage{amssymb}
\usepackage{amsmath}
\usepackage{listings}
\usepackage{hyperref}


\DeclareMathOperator{\sech}{sech}
\newcommand{\NN}{\mathbb{N}}
\newcommand{\RR}{\mathbb{R}}
\newcommand{\QQ}{\mathbb{Q}}
\newcommand{\ZZ}{\mathbb{Z}}
\newcommand{\dV}{\;\mathrm{d}V}
\newcommand{\dA}{\;\mathrm{d}A}
\newcommand{\dx}{\;\mathrm{d}x}
\newcommand{\dy}{\;\mathrm{d}y}
\newcommand{\dz}{\;\mathrm{d}z}
\newcommand{\cA}{\mathcal{A}}
\newcommand{\Bb}{\mathcal{B}}
\newcommand{\Ww}{\mathcal{W}}
\newcommand{\Dd}{\mathcal{D}}
\newcommand{\Ss}{\mathcal{S}}
\newcommand{\Ee}{\mathcal{E}}
\DeclareMathOperator{\im}{im}

\tolerance=1
\emergencystretch=\maxdimen
\hyphenpenalty=10000
\hbadness=10000

\setlength\parindent{18pt}

\begin{document}

\Large{Lincoln Sand}


\textbf{I1.}

\textbf{226.2}:

We look for a number $x$ such that $x^2$ is less than or equal to $1428$. Here, $x = 37$,
$37^2 = 1369$.
We have a remainder of $59$.

Bring down the next pair to get $5984$.

Double $37$ to get $p = 74$.
We will find an $x$ such that $(740 + x) \cdot x$ is less than or equal to $5984$.
$x = 8$ with $748 \cdot 8 = 5984$.
Subtract the result to get a remainder of $0$.

Append $8$ to $37$ to get $378$.

The square root of $142884$ is $378$.


\textbf{226.3}:

Find $x$ such that $x^2 \leq 12$, so $x = 3$.
Subtract $9$ from $12$ to get a remainder of $3$.

Bring down the next pair to get $381$.

Double $3$ to get $6$. Find $x$ such that $(60 + x) \cdot x \leq 381$.
So, $x = 5$, giving $65 \cdot 5 = 325$.
Subtract $325$ from $381$ to get a remainder of $56$.
Append $5$ to $3$ to get $35$.

Bring down the next pair to get $5629$.

Double $35$ to get $70$.
Find $x$ such that $(700 + x) \cdot x \leq 5629$.
So, $x = 7$ giving $707 \cdot 7 = 4949$.
Subtract $4949$ from $5629$ to get a remainder of $680$.
Append $7$ to $35$ to get $357$.

Bring down the last pair to get $68004$.

Double $357$ to get $714$. Find $x$ such that $(7140 + x) \cdot x \leq 68004$.
So, $x = 9$ because $7149 \cdot 9 = 64341$.
Subtract $64341$ from $68004$ to get a remainder of $3663$.
Append $9$ to $357$ to get $3579$.

Thus, the square root of $12812904$ is $3579$ with remainder $3663$.


\textbf{226.6}:

We will convert these into number of reservoirs filled per day.

$R_1 = 3$, $R_2 = 1$, $R_3 = \frac{2}{5}$, $R_4 = \frac{1}{3}$, $R_5 = \frac{1}{5}$.

Combined rate = $\frac{74}{15}$ reservoirs per day.

Thus, the time to fill one reservoir is $\frac{15}{74}$ days.


\textbf{226.8}:

\[s_6 = \sqrt{\frac{3}{4}}\]

\[s_{2n} = \sqrt{\frac{1}{2} - \sqrt{\frac{1}{4} - \left(\frac{s_n}{2}\right)^2}}\]
\[S_{2n} = 2n \cdot s_{2n}\]

So,
\[S_{6} = 5.196152422706632\]
\[S_{12} = 6.0\]
\[S_{24} = 6.2116570824605\]
\[S_{48} = 6.265257226562474\]
\[S_{96} = 6.278700406093744\]


\textbf{226.16}:

Let's denote the side of the square city as $s$. The diagonal $d$ of the square city can be calculated
using $d = s \sqrt{2}$.

So, the total path the person views the tree from can be expressed as $d + 20$ and using the Pythagorean theorem:
\[(s+14)^2 + 1775^2 = (d+20)^2\]
\[(s+14)^2 + 1775^2 = (s \sqrt{2} + 20)^2\]

Solving for $s$ gives $-20 \sqrt{2} + 14 + \sqrt{3151417 - 560 \sqrt{2}} \approx 1760.72$ pu.


\textbf{226.18}:

Consider the triangles $KEF$ and $KCD$, so $\frac{\text{KE}}{\text{KC}} = \frac{\text{FE}}{\text{DC}}$.

Plugging in the values gives:
\[\text{DC} = 10 + \frac{3}{4} \text{EC}\]

Consider the triangles $GEH$ and $GCD$, so $\frac{\text{GE}}{\text{GC}} = \frac{\text{FE}}{\text{DC}}$.

Plugging in the values gives:
\[600 \text{DC} = 453 \text{EC} + 2265\]

Plugging the first equation into the second gives:
\[\text{EC} = 1245\]

Plugging this back in gives:
\[\text{DC} = 943 \frac{3}{4}\]


\textbf{226.19}:

We can turn this into a system of equations.

Let $x$ be a unit of good grain, let $y$ be a unit of ordinary grain, and let $z$ be a unit of worst grain.

\[2x + 1y = 1\]
\[3y + 1z = 1\]
\[4z + 1x = 1\]

Solving this gives:
$x = \frac{9}{25}$, $y = \frac{7}{25}$, $z = \frac{4}{25}$.


\newpage


\textbf{I2.} Essay on Renaissance Mathematics

Lincoln Sand

Working title: Bhaskara II and the Foundations of Calculus

Essay topic description:

I want to write an essay about Bhaskaracharya's concepts of instantaneous
motion, especially since it predated Newton and Leibniz.
I want to explore his work with infinitesimals.


Interesting fact:

His work contained much of the elements and intuitions of modern
integral calculus, yet it predated its traditional discovery by Newton and Leibniz
by several centuries.
He invented an early form of Rolle's theorem.

Style manual I will use:

MLA

Two internet references:

1.

Bhāskara II: The great indian mathematician. Cuemath. (n.d.). https://www.cuemath.com/learn/bhaskara-ii/ 

2.

Encyclopædia Britannica, inc. (n.d.). Bhāskara II. Encyclopædia Britannica. https://www.britannica.com/biography/Bhaskara-II 

1.

Datta, B. and A. N. Singh. Use of Calculus in Hindu Mathematics. Indian Journal of History of Sciences 19.2 (1984): 95–104.

2.

Srinivasiengar, C. N. The History of Ancient Indian Mathematics. Calcutta: World Press, 1967.


NOTE: I may end up changing my book references later.



\end{document}
