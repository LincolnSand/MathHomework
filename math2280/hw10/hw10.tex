\documentclass{article}

\usepackage{amsfonts}
\usepackage{graphicx}
\usepackage{amssymb}
\usepackage{amsmath}
\usepackage{listings}


\DeclareMathOperator{\sech}{sech}
\newcommand{\NN}{\mathbb{N}}
\newcommand{\RR}{\mathbb{R}}
\newcommand{\QQ}{\mathbb{Q}}
\newcommand{\ZZ}{\mathbb{Z}}
\newcommand{\dV}{\;\mathrm{d}V}
\newcommand{\dA}{\;\mathrm{d}A}
\newcommand{\dx}{\;\mathrm{d}x}
\newcommand{\dy}{\;\mathrm{d}y}
\newcommand{\dz}{\;\mathrm{d}z}
\newcommand{\cA}{\mathcal{A}}
\newcommand{\Bb}{\mathcal{B}}
\newcommand{\Ww}{\mathcal{W}}
\newcommand{\Dd}{\mathcal{D}}
\newcommand{\Ss}{\mathcal{S}}
\newcommand{\Ee}{\mathcal{E}}
\DeclareMathOperator{\im}{im}


\setlength\parindent{18pt}

\begin{document}

Textbook Section 5.6:

2) The eigenvalues are $\lambda_1 = 0, \lambda_2 = 4$.
The corresponding eigenvectors are $v_1 = \begin{bmatrix}
    \frac{1}{2} \\
    1
\end{bmatrix}, v_2 = \begin{bmatrix}
    -\frac{1}{2} \\
    1
\end{bmatrix}$.

That means the fundamental matrix is
\[\phi(t) = \begin{bmatrix}
    \frac{1}{2} & -\frac{1}{2} e^{4t} \\
    1 & e^{4t}
\end{bmatrix}\]

\[\phi^{-1}(0) = \begin{bmatrix}
    \frac{1}{2} & -\frac{1}{2} \\
    1 & 1
\end{bmatrix}^{-1} = \begin{bmatrix}
    1 & \frac{1}{2} \\
    -1 & \frac{1}{2}
\end{bmatrix}\]

Thus,
\[x(t) = \begin{bmatrix}
    \frac{1}{2} & -\frac{1}{2} e^{4t} \\
    1 & e^{4t}
\end{bmatrix} \begin{bmatrix}
    1 & \frac{1}{2} \\
    -1 & \frac{1}{2}
\end{bmatrix} \begin{bmatrix}
    2 \\
    -1
\end{bmatrix}\]

6) The eigenvalues are $\lambda_1 = 5-4i, \lambda_2 = 5+4i$.
The corresponding eigenvectors are $v_1 = \begin{bmatrix}
    \frac{1}{2} - i \\
    1
\end{bmatrix}, v_2 = \begin{bmatrix}
    \frac{1}{2} + i \\
    1
\end{bmatrix}$.

That means the fundamental matrix is
\[\phi(t) = \begin{bmatrix}
    e^{5t} (\frac{1}{2} \cos(4t) + \sin(4t)) & e^{5t} (\frac{1}{2} \cos(4t) - \sin(4t)) \\
    e^{5t} (\frac{1}{2} \sin(4t) - \cos(4t)) & e^{5t} (\frac{1}{2} \sin(4t) + \cos(4t)) 
\end{bmatrix}\]

\[\phi^{-1}(0) = \begin{bmatrix}
    \frac{1}{2} & -1 \\
    -1 & \frac{1}{2}
\end{bmatrix}^{-1} = \begin{bmatrix}
    -\frac{2}{3} & -\frac{4}{3} \\
    -\frac{4}{3} & -\frac{2}{3}
\end{bmatrix}\]

Thus,
\[x(t) = \begin{bmatrix}
    e^{5t} (\frac{1}{2} \cos(4t) + \sin(4t)) & e^{5t} (\frac{1}{2} \cos(4t) - \sin(4t)) \\
    e^{5t} (\frac{1}{2} \sin(4t) - \cos(4t)) & e^{5t} (\frac{1}{2} \sin(4t) + \cos(4t)) 
\end{bmatrix} \begin{bmatrix}
    -\frac{2}{3} & -\frac{4}{3} \\
    -\frac{4}{3} & -\frac{2}{3}
\end{bmatrix} \begin{bmatrix}
    2 \\
    0
\end{bmatrix}\]

8) The eigenvalues are $\lambda_1 = -2, \lambda_2 = 1, \lambda_3 = 3$.
The corresponding eigenvectors are $v_1 = \begin{bmatrix}
    0 \\
    -1 \\
    1
\end{bmatrix}, v_2 = \begin{bmatrix}
    -1 \\
    1 \\
    0
\end{bmatrix}, v_3 = \begin{bmatrix}
    1 \\
    -1 \\
    1
\end{bmatrix}$.

That means the fundamental matrix is
\[\phi(t) = \begin{bmatrix}
    0 & -e^{t} & e^{3t} \\
    -e^{-2t} & e^t & -e^{3t} \\
    e^{-2t} & 0 & e^{3t}
\end{bmatrix}\]

\[\phi^{-1}(0) = \begin{bmatrix}
    -1 & -1 & 0 \\
    0 & 1 & 1 \\
    1 & 1 & 1
\end{bmatrix}\]

Thus,
\[x(t) = \begin{bmatrix}
    0 & -e^{t} & e^{3t} \\
    -e^{-2t} & e^t & -e^{3t} \\
    e^{-2t} & 0 & e^{3t}
\end{bmatrix} \begin{bmatrix}
    -1 & -1 & 0 \\
    0 & 1 & 1 \\
    1 & 1 & 1
\end{bmatrix} \begin{bmatrix}
    1 \\
    0 \\
    -1
\end{bmatrix}\]

10) The matrix A is $\begin{bmatrix}
    6 & -6 \\
    4 & -4
\end{bmatrix}$.

The eigenvalues are $\lambda_1 = 2, \lambda_2 = 0$.
The corresponding eigenvectors are $v_1 = \begin{bmatrix}
    1 \\
    1
\end{bmatrix}, v_2 = \begin{bmatrix}
    \frac{3}{2} \\
    1
\end{bmatrix}$.

$e^{At} = P e^{Dt} P{-1}$.

$P = \begin{bmatrix}
    1 & \frac{3}{2} \\
    1 & 1
\end{bmatrix}$

$e^{Dt} = \begin{bmatrix}
    e^{2t} & 0 \\
    0 & 1
\end{bmatrix}$

$P^{-1} = \begin{bmatrix}
    -2 & 3 \\
    2 & -2
\end{bmatrix}$

\[e^{At} = \begin{bmatrix}
    1 & \frac{3}{2} \\
    1 & 1
\end{bmatrix} \begin{bmatrix}
    e^{2t} & 0 \\
    0 & 1
\end{bmatrix} \begin{bmatrix}
    -2 & 3 \\
    2 & -2
\end{bmatrix}\]

17) The matrix A is $\begin{bmatrix}
    3 & 1 \\
    1 & 3
\end{bmatrix}$.

The eigenvalues are $\lambda_1 = 4, \lambda_2 = 2$.
The corresponding eigenvectors are $v_1 = \begin{bmatrix}
    1 \\
    1
\end{bmatrix}, v_2 = \begin{bmatrix}
    -1 \\
    1
\end{bmatrix}$.

$e^{At} = P e^{Dt} P{-1}$.

$P = \begin{bmatrix}
    1 & -1 \\
    1 & 1
\end{bmatrix}$

$e^{Dt} = \begin{bmatrix}
    e^{4t} & 0 \\
    0 & e^{2t}
\end{bmatrix}$

$P^{-1} = \begin{bmatrix}
    \frac{1}{2} & \frac{1}{2} \\
    -\frac{1}{2} & \frac{1}{2}
\end{bmatrix}$

\[e^{At} = \begin{bmatrix}
    1 & -1 \\
    1 & 1
\end{bmatrix} \begin{bmatrix}
    e^{4t} & 0 \\
    0 & e^{2t}
\end{bmatrix} \begin{bmatrix}
    \frac{1}{2} & \frac{1}{2} \\
    -\frac{1}{2} & \frac{1}{2}
\end{bmatrix}\]

24) $A = \begin{bmatrix}
    3 & 0 & -3 \\
    5 & 0 & 7 \\
    3 & 0 & -3
\end{bmatrix}, A^2 = \begin{bmatrix}
    0 & 0 & 0 \\
    36 & 0 & -36 \\
    0 & 0 & 0
\end{bmatrix}, A^3 = \begin{bmatrix}
    0 & 0 & 0 \\
    0 & 0 & 0 \\
    0 & 0 & 0
\end{bmatrix}$

This means that $e^{At} = I_n + At + \frac{A^2 t^2}{2}$.


Textbook Section 5.7:

1) Substituting $x_p(t) = A, y_p(t) = B$ into the original equation
gives:
\[A + 2B + 3 = 0\]
\[2A + B - 2 - 0\]

Solving this gives $A = \frac{7}{3}$
and $B = -\frac{8}{3}$.

That means our particular solution is:
\[x_p(t) = \frac{7}{3}\]
\[y_p(t) = -\frac{8}{3}\]

This solution is valid for any initial conditions since it represents
a constant solution to the non-homogeneous system.

5) We guess the particular solution
is of the form $x_p(t) = A + C t e^{-t}$
and $y_p(t) = B + D t e^{-t}$.

Through arithmetic after Substituting into the original equation,
we know that $A = -4$ and $B = -2$.

We can then solve the system and get that:
\[C = \frac{-10te^t - 14t - 10e^t}{6t - 1}\]
and
\[D = \frac{-10te^t - 14t + 2}{6t - 1}\]

13) Assume the particular solution is of the form:
\[x_p(t) = Ae^t\]
\[y_p(t) = Be^t\]

Substituting back into the original equation gives:
\[Ae^t = 2Ae^t + Be^t + 2e^t\]
\[Be^t = Ae^t + 2Be^t - 3e^t\]

After some solving, we get:
\[0 = Ae^t + Be^t + 2e^t\]
\[0 = Ae^t - Be^t - 3e^t\]

Solving this gives us that:
\[A = \frac{1}{2}\]
and
\[B = -\frac{5}{2}\]

Thus, the particular solution to the system is:
\[x_p(t) = \frac{1}{2} e^t\]
\[y_p(t) = -\frac{5}{2} e^t\]

17) The homogeneous solution is given by
\[x_h(t) = e^{At} x_0\].
That means that the homogeneous solution
is the zero vector $\begin{bmatrix}
    0 \\
    0
\end{bmatrix}$.

The particular solution is given by
\[x_p(t) = \int_{0}^{t} e^{A(t-s)} f(s) ds\].
\[x_p(t) = \begin{bmatrix}
    -7 e^{5t} + 102 - 95 e^{-t} \\
    -e^{5t} + 96 - 95 e^{-t}
\end{bmatrix}\]

Thus,
\[x(t) = x_p(t) = \begin{bmatrix}
    -7 e^{5t} + 102 - 95 e^{-t} \\
    -e^{5t} + 96 - 95 e^{-t}
\end{bmatrix}\]


\end{document}
