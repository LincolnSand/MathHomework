\documentclass{article}

\usepackage{amsfonts}
\usepackage{graphicx}
\usepackage{amssymb}
\usepackage{amsmath}
\usepackage{listings}


\DeclareMathOperator{\sech}{sech}
\newcommand{\NN}{\mathbb{N}}
\newcommand{\RR}{\mathbb{R}}
\newcommand{\QQ}{\mathbb{Q}}
\newcommand{\ZZ}{\mathbb{Z}}
\newcommand{\dV}{\;\mathrm{d}V}
\newcommand{\dA}{\;\mathrm{d}A}
\newcommand{\dx}{\;\mathrm{d}x}
\newcommand{\dy}{\;\mathrm{d}y}
\newcommand{\dz}{\;\mathrm{d}z}
\newcommand{\cA}{\mathcal{A}}
\newcommand{\Bb}{\mathcal{B}}
\newcommand{\Ww}{\mathcal{W}}
\newcommand{\Dd}{\mathcal{D}}
\newcommand{\Ss}{\mathcal{S}}
\newcommand{\Ee}{\mathcal{E}}
\DeclareMathOperator{\im}{im}


\setlength\parindent{18pt}

\begin{document}

Section 3.4:

2) Determine the period and frequency of the simple harmonic motion of a body of mass 0.75 kg on the
end of a spring with spring constant 48 N/m.

The period T of a simple harmonic oscillator, is given by:

\[T = 2 \pi \sqrt{\frac{m}{k}}\]

Plugging in the values, we get:

\[T = 2 \pi \sqrt{\frac{0.75}{48}} \approx 2 \pi \cdot 0.125 \approx 0.785s\]

Frequency f is given by:

\[f = \frac{1}{T}\]

So,

\[f = \frac{1}{0.785} \approx 1.27 Hz\]

Thus, the period is approximately 0.785 seconds and
the frequency is approximately 1.27 hertz.


5) Two pendulums are of lengths $L_1$ and $L_2$ and - when located at their respective distances $R_1$
and $R_2$ from the center of the earth - have periods $p_1$ and $p_2$. Show that
\[\frac{p_1}{p_2} = \frac{R_1 \sqrt{L_1}}{R_2 \sqrt{L_2}}\].

The period of a simple pendulum is given by:

\[p = 2 \pi \sqrt{\frac{L}{g}}\]

where L is the length of the pendulum
and g is the acceleration due to gravity.

\[g = \frac{G M}{R^2}\]

where G is the gravitational constant,
M is the mass of the earth,
and R is the distance from the center of the earth.

So,
\[p = 2 \pi \sqrt{\frac{L}{\frac{G M}{R^2}}}\]

So,
\[p_1 = 2 \pi \sqrt{\frac{L_1}{\frac{G M}{R_1^2}}}\]
\[p_2 = 2 \pi \sqrt{\frac{L_2}{\frac{G M}{R_2^2}}}\]

So,
\[\frac{p_1}{p_2} = \frac{2 \pi \sqrt{\frac{L_1}{\frac{G M}{R_1^2}}}}{2 \pi \sqrt{\frac{L_2}{\frac{G M}{R_2^2}}}}\]

Simplifying this gives us,

\[\frac{p_1}{p_2} = \sqrt{\frac{L_1 R_2^2}{L_2 R_1^2}}\]

Then, multiply both the numerator and denominator by $R_1 \sqrt{L_1}$:

\[\frac{p_1}{p_2} = \frac{R_1 \sqrt{L_1}}{R_2 \sqrt{L_2}}\]


15) A mass m is attached to both a spring (with given spring constant k) and a dashpot (with given damping constant c).
The mass is set in motion with initial position $x_0$ and initial velocity $v_0$. Find the position function x(t) and
determine whether the motion is overdamped, critically damped, or underdamped. If it is underdamped, write the
position function in the form $x(t) = C_1 e^{-pt} cos(\omega_1 t - \alpha_1)$. Also, find the undamped
position function $u(t) = C_0 cos(\omega_0 t - \alpha_0)$ that would result if the mass were set in
motion with the same initial position and velocity, but with the dashpot disconnected (so c = 0).
Finally, construct a figure that illustrates the effect of damping by comparing the graphs of x(t) and u(t).

$m = \frac{1}{2}$, c = 3, k = 4, $x_0 = 2$, $v_0 = 0$

The equation of motion for a damped harmonic oscillator is:

\[m \frac{d^2 x}{d t^2} + c \frac{dx}{dt} + kx = 0\]

Plugging in the given constants, we have:

\[\frac{1}{2} \frac{d^2 x}{d t^2} + 3 \frac{dx}{dt} + 4x = 0\]

First find the characteristic equation of the differential equation:

\[\frac{1}{2} r^2 + 3r + 4 = 0\]

Solving for r, we get:

\[r = -3 \pm \sqrt{5}\]

Since the roots are both real and distinct, the motion is overdamped.
The general solution for overdamped systems is:

\[x(t) = C_1 e^{r_1 t} + C_2 e^{r_2 t}\]

Now, solve for $C_1$ and $C_2$:

\[x(0) = 2 = C_1 + C_2\]
\[v(0) = 0 = C_1 r_1 + C_2 r_2\]

Substituting the values of $r_1$ and $r_2$, we get:

\[0 = C_1 (-3 + \sqrt{5}) + C_2 (-3 - \sqrt{5})\]

Since we have two equations and two unknowns, we can solve for $C_1$ and $C_2$.

We get $C_1 = \frac{2(\sqrt{5} - 1)}{4}$ and $C_2 = \frac{2(3-\sqrt{5})}{4}$.
Therefore, the position function is:

\[x(t) = \frac{2(\sqrt{5} - 1)}{4} e^{(-3 + \sqrt{5})t} + \frac{2(3-\sqrt{5})}{4} e^{(-3 - \sqrt{5})t}\]

Now, for the undamped case, the equation of motion is:
\[\frac{1}{2} \frac{d^2 u}{d t^2} + 4u = 0\]

Yielding the characteristic equation:
\[\frac{1}{2}r^2 + 4 = 0\]

So, $r = \pm 2i$, which gives a frequency of $\omega_0 = 2$.
The general solution for the undamped case is:
\[u(t) = C_0 cos(\omega_0 t - \alpha_0)\]

Solving for $C_0$ and $\alpha_0$:
\[u(0) = 2 = C_0 cos(\alpha_0)\]
\[v(0) = 0 = -C_0 \omega_0 sin(\alpha_0)\]

We get $C_0 = 2$ and $\alpha_0 = 0$ from the above equations.
So, the undamped equation is:

\[u(t) = 2cos(2t)\]


Graph: https://www.desmos.com/calculator/4bv9ucyrbo


21) A mass m is attached to both a spring (with given spring constant k) and a dashpot (with given damping constant c).
The mass is set in motion with initial position $x_0$ and initial velocity $v_0$. Find the position function x(t) and
determine whether the motion is overdamped, critically damped, or underdamped. If it is underdamped, write the
position function in the form $x(t) = C_1 e^{-pt} cos(\omega_1 t - \alpha_1)$. Also, find the undamped
position function $u(t) = C_0 cos(\omega_0 t - \alpha_0)$ that would result if the mass were set in
motion with the same initial position and velocity, but with the dashpot disconnected (so c = 0).
Finally, construct a figure that illustrates the effect of damping by comparing the graphs of x(t) and u(t).

m = 1, c = 10, k = 125, $x_0 = 6$, $v_0 = 50$

We have the equation:
\[\frac{d^2 x}{d t^2} + 10 \frac{dx}{dt} + 125x = 0\]

Assume a solution of the form:
\[x(t) = e^{rt}\]

Substituting in the equation, we get the characteristic equation:
\[r^2 + 10r + 125 = 0\]
The roots are $r = -5 \pm 10i$, so the motion is underdamped.

So, the solution will be of the form:
\[x(t) = e^{-5t} (C_1 cos(10t) + C_2 sin(10t))\]

Using the initial conditions of $x(0) = 6$ and $v(0) = 50$:
\[x(0) = 6 = C_1\]
\[v(0) = 50 = -5C_1 + 10C_2\]

Using the above equations, we get:
\[C_2 = 8\]

Thus, the solution is:
\[x(t) = e^{-5t} (6cos(10t) + 8sin(10t))\]

This can be written in our desired form as:
\[x(t) = C_1 e^{-pt} cos(\omega_1 t - \alpha_1)\]
where:
\[C_1 = \sqrt{6^2 + 8^2} = 10\]
\[p = 5\]
\[\omega_1 = 10\]
\[\alpha_1 = \arctan \left( \frac{8}{6} \right) = \arctan \left( \frac{4}{3} \right)\]

Now, for undamped motion:
\[\frac{d^2 u}{dt^2} + 125u = 0\]

The solution will be of the form:
\[u(t) = C_0 cos(\omega_0 t - \alpha_0)\]
where $\omega_0 = \sqrt{\frac{k}{m}} = \sqrt{125} = 5 \sqrt{5}$.

Using the initial conditions from before,
\[u(0) = 6 = C_0 cos(\alpha_0)\]
\[v(0) = 50 = -C_0 \omega_0 sin(\alpha_0)\]

Giving us:
\[\alpha_0 = 0; C_0 = 6\]

Thus, the undamped position function is:
\[u(t) = 6cos(5\sqrt{5}t)\]


Graph: https://www.desmos.com/calculator/cc0xenguxa


Section 3.6:

2) Equation 8: $x(t) = C cos(\omega_0 t - \alpha) + \frac{F_0/m}{\omega_0^2 - \omega^2} cos \omega t$.

Express the solution of the given initial value problem as a sum of two oscillation as in Equation 8.
Throughout, primes denote derivatives with respect to time t. Then graph the solution function x(t)
in such a way that you can identify and label its period.

x'' + 4x = 5 sin 3t; x(0) = x'(0) = 0

The homogenous equation $x'' + 4x = 0$ has the characteristic equation:
\[r^2 + 4 = 0\]
So, $r = \pm 2i$, so the homogenous solution is:
\[x_h(t) = C_1 cos(2t) + C_2 sin(2t)\]

For the particular solution, we assume it is of the form:
\[x_p(t) = A cos(3t) + B sin(3t)\]

Since the inhomogenous term is $5 sin(3t)$, then
$x_p(t) = B sin(3t)$.

\[B'' sin(3t) + 4B sin(3t) = 5 sin(3t)\]

\[\implies x_p(t) = -sin(3t)\]

So, the general solution is:
\[x(t) = C_1 cos(2t) + C_2 sin(2t) - sin(3t)\]

We can find $C_1$ and $C_2$ using the initial conditions:

\[x(0) = C_1 - 0 = 0; C_1 = 0\]
\[v(0) = 2C_2 - 3 = 0; C_2 = \frac{3}{2}\]

This gives a final solution to the initial value problem of:
\[x(t) = \frac{3}{2} sin(2t) - sin(3t)\]


Graph: https://www.desmos.com/calculator/irukqqk6ky

The period is $2 \pi$.


11) Find and plot both the steady periodic function $x_{sp}(t) = C cos(\omega t - \alpha)$ of the
given differential equation and the actual solution $x(t) = x_{sp} + x_{u}(t)$ that
satisfies the given initial conditions.

x'' + 4x' + 5x = 10 cos 3t; x(0) = x'(0) = 0

First we need to find the steady periodic function $x_{sp}$ of:
\[x'' + 4x' + 5x = 10cos(3t)\]

We assume a solution of the form:
\[x_{sp}(t) = C cos(3t - \alpha)\]

Plugging this back into the differential equation, we get:

\[-C 9 cos(3t - \alpha) - 4 C 3 sin(3t - \alpha) + 5 C cos(3t - \alpha) = 10 cos(3t)\]

Solving for the constants, we get:
\[C = \frac{5}{2}; \alpha = 0\]

So the steady periodic function is:
\[x_{sp}(t) = \frac{5}{2} cos(3t)\]

Now, we have to find the full solution x(t). The general solution of
$x'' + 4x' + 5x = 0$ is:
\[x_h(t) = e^{-2t} (C_1 cos(t) + C_2 sin(t))\]

So, the complete solution is:
\[x(t) = \frac{5}{2} cos(3t) + e^{-2t} (C_1 cos(t) + C_2 sin(t))\]

Using the initial conditions $x(0) = v(0) = 0$, we get:
\[C_1 = -\frac{5}{2}; C_2 = 5\]

So, the final solution to the initial value problem is:
\[x(t) = \frac{5}{2} cos(3t) + e^{-2t} (-\frac{5}{2} cos(t) + 5 sin(t))\]


Plot: https://www.desmos.com/calculator/vtcnlhhsqd



15) This problem gives the parameters for a forced mass-spring-dashpot system with equation
$mx'' + cx' + kx = F_0 cos \omega t$. Investigate the possibility of practical
resonance of this system. In particular, find the amplitude $C(\omega)$ of
steady periodic forced oscillations with frequency $\omega$. Sketch the
graph of $C(\omega)$ and find the practical resonance frequency $\omega$ (if any).

m = 1; c = 2; k = 2; $F_0 = 2$

We have:
\[x'' + 2x' + 2x = 2 cos \omega t\]

We can find the amplitude of the steady periodic forced oscillations
if we assume a particular solution of the form:
\[x_{sp}(t) = C(\omega) cos(\omega t - \alpha)\]

where $C(\omega)$ is the amplitude we want to find.

Plugging the above steady state form back into the differential equation gives us:
\[-C(\omega) \omega^2 cos(\omega t - \alpha) - 2 C(\omega) \omega sin(\omega t - \alpha) + 2 C(\omega) cos(\omega t - \alpha) = 2 cos(\omega t)\]

Solving that equation gives us:
\[C(\omega) = \frac{F_0}{\sqrt{(k - m \omega^2)^2 + (c \omega)^2}}\]

If we substitute in the given values, we have:
\[C(\omega) = \frac{2}{\sqrt{(2 - \omega^2)^2 + (2 \omega)^2}}\]

To find the practical resonance frequency, we have to get the maximum
of $C(\omega)$. So we have to find when the derivative of $C(\omega)$
with respect to $\omega$ is 0.

\[\frac{d C(\omega)}{d \omega} = \frac{-4 \omega (2 - \omega^2) + 8 \omega}{((2 - \omega^2)^2 + 4 \omega^2)^\frac{3}{2}}\]

So,
\[-4 \omega (2 - \omega^2) + 8 \omega = 0\]
\[\omega^3 - 2 \omega = 0\]

The roots are $\omega = 0, \pm \sqrt{2}$.
But $\omega$ must be non-negative, so we will discard $-\sqrt{2}$
as a valid root.

\[C(\sqrt{2}) = \frac{1}{\sqrt{2}}\]
\[C(0) = 1\]

Therefore, the maximum amplitude occurs at $\omega = 0$
and the practical resonance frequency is $\omega = 0$.


Sketch: https://www.desmos.com/calculator/aevlaufb5n


24)

For small angles, the restoring force from the pendulum is:
\[F_p = m g \theta\]

The restoring force from the spring is:
\[F_s = kx\]

The equation of motion for the pendulum:
\[m \frac{d^2 \theta}{d t^2} = -m g \theta - k x\]

The equation of motion for the spring:
\[m \frac{d^2 x}{d t^2} = -kx -m g \theta\]

Because the oscillations are small, x and $\theta$
can be related via arclength:
\[x = L \theta\]

If we substitute these back into the above equations of motion, we get:
\[m \frac{d^2 \theta}{d t^2} = -m g \theta - k L \theta\]
\[m \frac{d^2 x}{d t^2} = - k L \theta -m g \theta\]

The general form for the equation for simple harmonic motion is given by:
\[\frac{d^2 y}{d t^2} = -\omega_0^2 y\]


So, we have:
\[-\omega_0^2 m \theta = -m g - k L \theta\]
\[\omega_0^2 m \theta = g \theta + \frac{k L \theta}{m}\]

If we solve for $\omega_0^2$, we have:
\[\omega_0^2 = \frac{g}{L} + \frac{k}{m}\]

Therefore, the natural circular frequency of the mass's motion is:
\[\omega_0 = \sqrt{\frac{g}{L} + \frac{k}{m}}\]


\end{document}
