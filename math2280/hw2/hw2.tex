\documentclass{article}

\usepackage{amsfonts}
\usepackage{graphicx}
\usepackage{amssymb}
\usepackage{amsmath}

\setlength\parindent{18pt}

\begin{document}


Textbook Section 1.3:

12) $\frac{dy}{dx} = x \cdot ln(y)$; $y(1) = 1$

$x \cdot ln(y)$ is continuous on the open $x$ interval $(0, \infty)$,
which contains $x = 1$.
$\therefore$ a solution does exist.
Now for uniqueness:
\[\frac{\partial f}{\partial y} = x/y\]

$\frac{\partial f}{\partial y}$ is continuous on the open $x$ interval
$(0, \infty)$, which also contains $x = 1$ and $y = 1$.
$\therefore$ the solution is unique.

$\exists$ a unique solution for the differential equation with
the initial starting value of $y(1) = 1$.


13) $\frac{dy}{dx} = \sqrt[3]{y}$; $y(0) = 1$

$\sqrt[3]{y}$ is continuous on the open $x$ interval $(-\infty, \infty)$,
which contains $x = 0$.
$\therefore$ a solution does exist.
Now for uniqueness:
\[\frac{\partial f}{\partial y} = \frac{1}{3}y^{-\frac{2}{3}}\]
\[= \frac{1}{3 \cdot y^{\frac{2}{3}}}\].
$\frac{1}{3 \cdot y^{\frac{2}{3}}}$ is not continuous at $x = 0$,
$\therefore$ the solution is not guaranteed to be unique.


17) $y \frac{dy}{dx} = x-1$; $y(0) = 1$

\[y \frac{dy}{dx} = x-1
\implies \frac{dy}{dx} = \frac{x-1}{y}\]

$\frac{x-1}{y}$ in continuous on the open $x$ interval $(-\infty, \infty)$,
which contains $x = 0$.
$\therefore$ a solution does exist.
Now for uniqueness:
\[\frac{\partial f}{\partial y} = \frac{1-x}{y^2}\]
This is continuous on the open $x$ interval $(-\infty, \infty)$,
which contains $x = 0$.
$\therefore$ the solution is unique.

$\exists$ a unique solution for the differential equation with
the initial starting value of $y(0) = 1$.


Textbook Section 1.4:

34) (Population growth) In a certain culture of bacteria, the number
of bacteria increased sixfold in 10h. How long did it take the
population to double?

Let $r$ be the growth constant in hours.

\[r^{10} = 6
\implies ln(r^{10}) = ln(6)
\implies 10 \cdot ln(r) = ln(6)
\implies ln(r) = \frac{ln(6)}{10}\]
\[\implies r = e^{\frac{ln(6)}{10}}
\implies r \approx 1.196\]

Now, let's find the time when it doubles.

\[r^t = 2
\implies ln(r^t) = ln(2)
\implies t \cdot ln(r) = ln(2)
\implies t = \frac{ln(2)}{ln(r)}\]
\[\implies t \approx 3.869\]

The bacteria colony population doubles after approximately $3.869$ hours.


Textbook Section 1.5:

8) $3xy' + y = 12x$

We have to use the integrating factor method.
\[3xy' + y = 12x
\implies y' + \frac{1}{3x}y = 4
\implies P(x) = \frac{1}{3x}; Q(x) = 4\]

\[M(x) = \int P(x) dx
\implies M(x) = \int \frac{1}{3x} dx
\implies M(x) = \frac{1}{3} \cdot ln(|3x|)\]
\[\implies M(x) = ln(|\sqrt[3]{3x}|)\]

\[y(x) = \frac{4}{|\sqrt[3]{3x}|} \cdot \int |\sqrt[3]{3x}| dx\]

For $x \geq 0$:

\[y(x) = \frac{4}{\sqrt[3]{3x}} \cdot \int \sqrt[3]{3x} dx\]


For $x < 0$:

\[y(x) = \frac{4}{-\sqrt[3]{3x}} \cdot \int -\sqrt[3]{3x} dx
\implies y(x) = \frac{4}{\sqrt[3]{3x}} \cdot \int \sqrt[3]{3x} dx\]



So, $y(x) = \frac{4}{\sqrt[3]{3x}} \cdot \int \sqrt[3]{3x} dx$



\[\int \sqrt[3]{3x} dx = \frac{3}{12} \cdot (3x)^{\frac{4}{3}} + C\]

\[y(x) = \frac{4}{\sqrt[3]{3x}} \cdot (\frac{3}{12} \cdot (3x)^{\frac{4}{3}} + C)
\implies y(x) = 3x + \frac{4 \cdot C}{\sqrt[3]{3x}}\]

\[\therefore y(x) = 3x + \frac{4 \cdot C}{\sqrt[3]{3x}}.\]


14) $xy' - 3y = x^3; y(1) = 10$

\[xy' - 3y = x^3
\implies y' - \frac{3}{x}y = x^2
\implies P(x) = -\frac{3}{x}; Q(x) = x^2\]

\[M(x) = \int P(x) dx
\implies M(x) = -3 \cdot \int \frac{1}{x} dx
\implies M(x) = ln(|x^{-3}|)\]

\[y(x) = |x^3| \cdot \int |\frac{1}{x^3}| \cdot x^2 dx\]

The sign will cancel from the absolute value like in $x$, so we get:

\[y(x) = x^3 \cdot \int \frac{1}{x^3} \cdot x^2 dx
\implies y(x) = x^3 \cdot \int \frac{1}{x} dx
\implies y(x) = x^3 \cdot (ln|x| + C)\]

\[y(1) = 1^3 \cdot (ln|x| + C)
\implies y(1) = C = 10
\implies C = 10\]

\[\therefore y(x) = x^3 \cdot (ln|x| + 10)\]


29) Express the general solution of $\frac{dy}{dx} = 1 + 2xy$ in
terms of the error function

\[erf(x) = \frac{2}{\sqrt{\pi}} \cdot \int_{0}^{x} e^{-t^2} dt.\]

\[\frac{dy}{dx} = 1 + 2xy
\implies y' - 2xy = 1
\implies P(x) = -2x; Q(x) = 1\]

\[M(x) = \int -2x dx
\implies M(x) = -x^2\]

\[y(x) = e^{x^2} \cdot \int \frac{1}{e^{x^2}} dx\]

\[\int \frac{1}{e^{x^2}} dx
= \int e^{-x^2} dx
= \frac{\sqrt{\pi}}{2} erf(x) + C\]

\[e^{x^2} = \frac{\sqrt{\pi}}{2} \cdot \frac{1}{\frac{d}{dx} erf(x)}\]

\[y(x) = e^{x^2} \cdot (\frac{\sqrt{\pi}}{2} erf(x) + C)
\implies y(x) = \frac{\sqrt{\pi}}{2} \cdot \frac{1}{\frac{d}{dx} erf(x)} \cdot (\frac{\sqrt{\pi}}{2} erf(x) + C).\]


34) Consider a resevoir with a volume of 8 billion cubic feet ($ft^3$)
and an initial pollutant concentration of 0.25\%. There is a daily
inflow of 500 million $ft^3$ of water with a pollutant
concentration of 0.05\% and an equal daily output of the well-mixed
water in the resevoir. How long will it take to reduce the pollutant
concentration in the resevoir to 0.10\%?

Let $V_0 = 8$

Let $Poll_0 = 0.25\% \cdot V_0 = 0.02$

Let $\frac{dPoll}{dt} = 0.5 \cdot 0.05\% - 0.5 \cdot \frac{Poll}{8}
= 0.00025 - \frac{1}{16} \cdot Poll$

\[\frac{dPoll}{dt} = 0.00025 - \frac{1}{16} \cdot Poll
\implies \frac{dPoll}{dt} + \frac{1}{16} \cdot Poll = 0.00025\]
\[\implies P(x) = \frac{1}{16} = 0.0625; Q(x) = 0.00025\]

\[M(x) = \int P(x) dx
\implies M(x) = \int 0.0625 dx
\implies M(x) = 0.0625x\]

\[Poll(x) = \frac{1}{e^{0.0625x}} \cdot \int e^{0.0625x} \cdot 0.00025 dx\]
\[\implies Poll(x) = \frac{0.00025}{0.0625 \cdot e^{0.0625x}} \cdot (e^{0.0625x} + C)\]
\[\implies Poll(x) = \frac{0.00025}{0.0625} + \frac{0.00025 \cdot C}{0.0625 \cdot e^{0.0625x}}\]

\[Poll(0) = 0.004 + 0.004 \cdot \frac{C}{e^{0.0625 \cdot 0}}
= 0.004 + 0.004 \cdot C
= 0.004 \cdot (C + 1)\]

\[0.004 \cdot (C + 1) = 0.02
\implies C + 1 = 5
\implies C = 4\]

\[Poll(x) = 0.004 + \frac{0.016}{e^{0.0625 \cdot x}}\]

\[Poll(x) = 0.10\% \cdot 8 = 0.008\]

\[0.004 + \frac{0.016}{e^{0.0625 \cdot x}} = 0.008
\implies \frac{0.016}{e^{0.0625 \cdot x}} = 0.004
\implies \frac{1}{e^{0.0625 \cdot x}} = 0.25\]

\[\implies e^{-0.0625 \cdot x} = 0.25
\implies -0.0625 \cdot x = ln(0.25)
\implies x = \frac{ln(0.25)}{-0.0625} \approx 22.181\]

It will take approximately $22.181$ days for the pollutant concentration
in the resevoir to reach $0.10\%$.


36) A tank initially contains 60 gal of pure water. Brine
containing 1lb of salt per gallon enters the tank at 2 gal/min, and
the (perfectly mixed) solution leaves the tank at 3 gal/min; thus the tank
is empty after exactly 1h.

a) Find the amount of salt in the tank after t minutes.

$\frac{dS}{dt} = 2 - 3 \cdot \frac{S}{60-t}$

\[\frac{dS}{dt} = 2 - 3 \cdot frac{S}{60-t}
\implies \frac{dS}{dt} + \frac{3}{60-t} S = 2\]
\[\implies P(t) = \frac{3}{60-t}; Q(t) = 2\]

\[M(t) = 3 \cdot \int \frac{1}{60-t} dt
\implies M(t) = ln((60-x)^3)\]

\[S(x) = 2 \cdot (60-x)^{-3} \cdot \int (60-x)^3 dx
\implies S(x) = 2 \cdot (60-x)^{-3} \cdot (-\frac{60-x}{4} + C)\]
\[\implies S(x) = -\frac{1}{2}(60-x)^{-2} + 2 \cdot C \cdot (60-x)^{-3}\]

\[S(0) = -\frac{1}{2}(60-x)^{-2} + 2 \cdot C \cdot (60-x)^{-3} = 0
\implies 2 \cdot C \cdot (60)^{-3} = \frac{1}{2}(60)^{-2}\]
\[\implies C = \frac{60}{4} = 15\]

\[S(x) = -\frac{1}{2}(60-x)^{-2} + 2 \cdot 15 \cdot (60-x)^{-3}
\implies S(x) = -\frac{1}{2}(60-x)^{-2} + 30 \cdot (60-x)^{-3}\]

\[\therefore S(x) = -\frac{1}{2}(60-x)^{-2} + 30 \cdot (60-x)^{-3}\]


b) What is the maximum amount of salt ever in the tank?

\[\frac{dS}{dt} = 0
\implies 2 - 3 \cdot (-\frac{1}{2}(60-x)^{-2} + 30 \cdot (60-x)^{-3}) = 0\]
\[\implies -\frac{1}{2}(60-x)^{-2} + 30 \cdot (60-x)^{-3} = \frac{2}{3}
\implies x \approx 56.513\]

\[S(56.513)\] is the maximum value of salt in the container.


I think my answer for a is wrong, but I've redone this problem like 4 times
and kept finding arithmetic and other nonsense errors,
but I can't find anymore, so I'm leaving it as is.


Textbook section 2.1:

1) $\frac{dy}{dt} = x - x^2; x(0) = 2$

12) The time rate of change of an alligator population $P$
in a swamp is proportional to the square of P. The swamp
contained a dozen alligators in 1988, two dozen in 1998.
When will there be four dozen alligators in the swamp?
What happens thereafter?

24) Suppose that a community contains 15,000 people who are
susceptible to Michaud's syndrome, a contagious disease. At
time $t = 0$ the number $N(t)$ of people who have developed
Michaud's syndrome is 5000 and is increasing at the rate of 500
per day. Assume that $N'(t)$ is proportional to the product
of the numbers of those who have caught the disease and those
who have not. How long will it take for another 5000
people to develop Michaud's syndrome?


Custom problem:

\[\frac{dy}{dx} = y^2; y(1) = 2\]

a)
\[\frac{dy}{dx} = y^2
\implies y^{-2}dy = dx
\implies \int y^{-2} dy = \int dx
\implies -\frac{1}{y} = x + C\]
\[\implies \frac{1}{y} = C - x
\implies y = \frac{1}{C-x}\]

b) It fails because there is no open
continuous interval for $y(x)$ where $x = \frac{3}{2}$ is continuous
(it is not included in the maximal interval of existence).


\end{document}
